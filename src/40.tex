%!TEX root = ../report.tex"
\section{Вопрос 40: Конечные расширения полей и расширения конечной степени, примеры. Теорема о башне полей. Расширения поля корнем неприводимого многочлена, Поле разложения многочлена, теорема существования. Минимальное поле разложения.
}

\begin{defs}[Поле]
  \textit{Полем} называется коммутативное кольцо с единицей, отличной от нуля, в котором каждый ненулевой элемент обратим.
\end{defs}

\begin{defs}[Подполе]
  Пусть $P$ - поле, тогда подмножество $K \subset P$ называется \textit{подполем поля $P$}, если оно замкнуто отностительно операций и является полем.
  Обозначается как $K < P$.
  Поле $P$ назывется \textit{расширением поля $K$}.
\end{defs}

\begin{defs}[Алгебраический элемент]
  Пусть $P < P'$, тогда $a \in P'$ называется \textit{алгебраическим элементом над полем $P$}, если $\exists\; f(x) \in P[x] < P'[x], f(x) \neq 0$, т.что в $P':f(a) = 0$. В противном случае элемент называется \textit{трансцендентным}.
\end{defs}

\begin{example}[Примеры алгебраических элементов]
    $\mathbb{Q}<\mathbb{C}, i = \sqrt{-1}$ - является корнем $x^2 - 1 \in \mathbb{Q} \Rightarrow i$ алгебраичен над $\mathbb{Q}$.
\end{example}

\begin{defs}[Минимальный многчлен элемента]
  Пусть $P < P'$,  $a \in P'$ алгебраичен над полем $P$, тогда единственный унитарный неприводимый многочлен
  $m(x) \in P[x]$, т.что $ m(a) = 0$ и $\forall \; f(x) \in P[x]$ т.что $f(a) = 0 $ верно $m(a) | f(x) $ называется \textit{минимальным многочленом элемента $a$}.
  Обозначается $m_{a,P}(x)$
\end{defs}

\begin{proofs}[Теорема о существовании минимального многочлена]
	Пусть $P < P'$,  $a \in P'$ алгебраичен над полем $P$, тогда $m_{a,P}$ существует.
\end{proofs}

\begin{defs}[Степень расширения поля]
  Пусть $P < P'$, тогда $(P, +) < (P', +), (P, \cdot) < (P', \cdot)$ и $\forall\; a,b \in P$ и $\alpha,\beta \in P'$ верно:
  \begin{enumerate}
    \item $\alpha(ab) = (\alpha a)b$
    \item $\alpha(a+b) = (\alpha a) + (\alpha b)$
    \item $(\alpha+\beta)a = (\alpha a) + (\beta a)$
    \item $\alpha\cdot(e) = \alpha$
  \end{enumerate}
  Можно рассматривать $P'$ как векторное пространство $P'_{P}$
  Величина $[P':P] = dimP'_{P}$ - называется \textit{стпенью расширения поля $P'$ над полем $P$}.
\end{defs}

\begin{proofs}[Критерий быть подполем]
	Пусть $P$ - поле, $K \subset P$, $K \neq \emptyset$, $K \neq \{0\}$. Тогда $K$ является подполем ТИТТК $\forall a,b \in K$ верно: $(a - b) \in K$, $(a \cdot b) \in K$ и, если $a \neq 0$, то $a^{-1} \in K$.
\end{proofs}

\begin{example}[Примеры подполей]
  \begin{enumerate}
    \item $P = \mathbb{R}, P' = \mathbb{C}, \{1,i\}$ - базис. $[\mathbb{C}:\mathbb{R}] = 2$
    \item $[\mathbb{R}:\mathbb{Q}] = \infty$
    \item Пусть $P$ - поле, $P' = P[x]_{/f(x)}$, где $f(x)$ - унитарный неприводимый многочлен. Тогда $P'$ содержит подполе $P_{o} < P'$,
    которое изоморфно полю $P \Rightarrow [P':P_{o}] = degf(x)$. Базис: $\{[e]_{f}, [x]_{f}, ... , [x^{degf(x)-1}]_{f}\}$
  \end{enumerate}
\end{example}

\begin{defs}[Башня полей]
  Несколько последовательных расширейний называются \textit{башней полей}.
\end{defs}

\begin{proofs}[Теорема о башне полей]
	Пусть $P < P' < P''$, тогда $[P'' : P] < \infty$ ТИТТК $[P' : P] < \infty$, $[P'' : P'] < \infty$ и выполняется равенство $[P'' : P] = [P'' : P'] \cdot [P' : P]$.
  \begin{dokvo}
    \zagolovok{Необходимость:} Пусть $[P'' : P] < \infty$. Заметим, что тк $P' < P''$, пространство $P'_{P} < P''_{P}$, если при этом $[P' : P] = \infty \Rightarrow dimP''_{P} \geq dimP'_{P} = \infty \Rightarrow [P'' : P] = \infty$ - противоречие. Значит $[P' : P] < \infty$.
    Если $[P'' : P'] = \infty$, то в $P'' \exists$ бесконечный базис над $P'$(бесконечная ЛНЗ сист.) $\Rightarrow$ эта сист независима на $P$, тк $P < P'$, но мы доказали, что $[P' : P] < \infty$, значит $[P'' : P'] < \infty и равенство выполняется$.
    \zagolovok{Достаточность:} Пусть $[P'' : P'] = m, [P' : P] = n$. Выделим 2 базиса $\{\alpha_{1} ... \alpha_{m}\}$ - базис $P''_{P'}$ и $\{\beta_{1} .. \beta_{n}\}$ - базис $P'_{P}$.
    Положим $i = \overline{1,m}$ и $j = \overline{1,n}$. Покажем, что $\gamma_{ij} = \alpha_{i}\beta_{j}$ - базис $P''_{P}$.
    $\{\gamma_{ij}\}$ - ЛНЗ над $P$, тк если $\exists d_{ij} \in P$ не все равные нулю, т.что $\sum_{i=1,j=1}^{m,n}d_{ij}\gamma_{ij} = 0$, то
    $\sum_{i=1,j=1}^{m,n}d_{ij}\alpha_{i}\beta_{j} = \sum_{i=1}^{m}\alpha_{i}(\sum_{j=1}^{n}d_{ij}\beta_{j}) = 0 \Rightarrow$
    тк $\{\alpha_{1} ... \alpha_{m}\}$ - базис $P''_{P'}$ полагаем,что $\sum_{j=1}^{n}d_{ij}\beta_{j} = 0 \;\forall i = \overline{1,m} \Rightarrow d_{ij} = 0 \Rightarrow \{\gamma_{ij}\}$ - ЛНЗ.
    Покажем, что любой вектор выражается через систему. Пусть $c \in P'', c = \sum_{i=1}^{m}c_{i}\alpha_{i} \Rightarrow c_{i} \in P'$,
    $c_{i} = \sum_{j=1}^{n}d_{ij}\beta_{j} \Rightarrow c = \sum_{i=1}^{m}\alpha_{i}(\sum_{j=1}^{n}d_{ij}\beta_{j}) = \sum_{i=1,j=1}^{m,n}d_{ij}\alpha_{i}\beta_{j} = \sum_{i=1,j=1}^{m,n}d_{ij}\gamma_{ij}
    \Rightarrow$ линейно выражается $\Rightarrow$ базис $P''_{P}$ $\Rightarrow [P'' : P'] = mn < \infty$.

	\end{dokvo}
\end{proofs}

\begin{sledsv}
  Если $P_{1} < ...' < P_{n}$ - башня полей, то $[P_{n} : P_{1}] < \infty$ ТИТТК $[P_{i} : P_{i-1}] < \infty$ и в этом случае $[P_{n} : P_{1}] = [P_{n} : P_{n-1}] \cdot ... \cdot [P_{2} : P_{1}]$.
\end{sledsv}

\begin{defs}[Расширение, порожденное множеством]
  Пусть $P < P', M \subset P'$. \textit{Расширением поля $P$, порожденным множеством $M$} называют поле $P(M) = \cap P''$, где $P'' < P'$
   и $(P \cup M) \subset P''$. Если $|M| = \infty$, то $P(M)$ - бесконечое расширение. Если $|M| < \infty$, то $P(M)$ - конечное расширение
\end{defs}

\begin{example}[Примеры расширений полей порожденных множествами]
  \begin{enumerate}
    \item $\mathbb{R}(i) = \mathbb{C}$
    \item $\mathbb{Q}](\sqrt{2}, \square{3})$
    \item Пусть $P$ - поле, $P' = P[x]_{/f(x)}$, где $f(x)$ - унитарный неприводимый многочлен степени $m$.
    $a = [x]_{f}$, тогда $P(a)$ содержит элементы $e,a,a^2,...,a^{m-1}. \; a \in P' \Rightarrow P(a) < P'.$ В то же время $\forall\; g(x) \in P[x] [g(x)]_{f} \in P(a) \Rightarrow P' = P(a).$
    Таким образом $a$ - алгебраичен и $m_{a,P}(x) = f(x)$.
  \end{enumerate}
\end{example}

\begin{defs}[Расширение поля корнем неприводимого многочлена]
  Построенное в примере 3 поле $P(a)$ называется расширением поля $P$ корнем неприводимого многочлена $f(x)$.
\end{defs}

\begin{defs}[Поле разложения многочлена]
  Пусть $P < P'$, тогда поле $P'$ называется \textit{полем разложения многочлена $f(x) \in P[x]$}, если $f(x)$ раскладывается в поле
  $P'$ на линейные множители.
\end{defs}

\begin{example}[Пример поля разложения]
  $f(x) = x^2 - 2 \in \mathbb{Q}[x], \mathbb{R}$ - поле разложения.
\end{example}


\begin{proofs}[Теорема о существовании расширений, обнуляющих многчлен]
	Пусть $f(x) \in P[x]$, тогда $\exists P', P < P'$, т.что $\exists a \in P'$, т.что $f(a) = 0$ в $P'$.
  \begin{dokvo}
    Пусть $P' = P[x]_{/f(x)}, f(x) = x^m - c_{m-1}x^{m-1} - ... - - c_1x - c_0,\;a = [x]_{f}$.
    Рассмотрим $f'(x) = [e]_f[x]_f^m - [c_{m-1}]_f[x]_f^{m-1} - ... - - [c_1]_f[x]_f - [c_0]_f \in P'$, тогда
    $f'(a) = f'([x]_f) = [x^m - c_{m-1}x^{m-1} - ... - - c_1x - c_0]_f = [f(x)]_f = [0]_f \Rightarrow f'(x)$ - имеет корень.
	\end{dokvo}
\end{proofs}

\begin{proofs}[Теорема о существовании поля разложения]
	Пусть $f(x) \in P[x],\; degf(x) = m \geq 1$, тогда существует поле разложения $f(x)$.
  \begin{dokvo}
    Доказывается индукцией по $m.\;$
    \begin{enumerate}
    \item $m = 1 \Rightarrow f(x) = x-a,\; f(a) = 0 \Rightarrow P$ - поле разложения.
    \item Пусть теорема верна для $m \leq n$.
    \item Докажем для $m = n + 1$. Если $f(x)$ - неприводим в $P[x]$, то, согласно теореме о существовании расширений, обнуляющих многчлен,
    существует поле $P'$, в кот $f(x)$ имеет вид $f(x) = (x-a)g(x) \in P'[x] \Rightarrow degg(x) \leq n \Rightarrow \exists\; P'', P' < P''$
    - поле разложения $g(x)$ (по предположению индукции). Таким образом $P < P' < P''$, значит $P''$ - поле разложения
    $f(x)$ над $P$. Если $f(x)$ - приводим в $P[x]$, то $f(x) = f_1(x)f_2(x), degf_i(x) \leq n \Rightarrow \exists\; P'$ - поле разложения
    $f_1()$ над $P$. По предположению индукции $\exists\; P''$ - поле разложения $f_2(x)$ над $P'\Rightarrow P''$ - поле разложения
    $f(x)$ над $P$.
  \end{enumerate}
	\end{dokvo}
\end{proofs}

\begin{defs}[Минимальное поле разложения многочлена]
  Поле разложения $P'$ многочлена $f(x)$ называется минимальным, если оно порождено корнями $f(x)$.
\end{defs}

\begin{example}[Пример минимального поля разложения]
  $\mathbb{Q}(\sqrt{2})$ - минимальное поле разложения многочлена $f(x) = ^2 - 2$.
\end{example}

\begin{proofs}[Теорема о существовании минимального поля разложения]
	$\forall\; P$ и $f(x) \in P[x], degf(x) \geq 1\; \exists!$ минимальное поле разложения.
  \begin{dokvo}
    Пусть $P'$ - произвольное поле разложения $f(x)$ и $(a_1, ... , a_n)$ - все различные корни $f(x)$ в $P'$.
    Тогда $P(a_1, ... , a_n)$ - минимальныое поле разложения. Единственное, тк любое другое поле разложения содержит
    $(a_1, ... , a_n) \Rightarrow $ содержит $P(a_1, ... , a_n)$.
	\end{dokvo}
\end{proofs}

\newpage
