%!TEX root = ../report.tex"
\section{Вопрос 4: Формула Тейлора для действительных функций одного и многих действительных переменных. Экстремум действительной функции одного и многих действительных переменных, достаточные условия его существования.}

\begin{defs}
	Точка $x \in E$ называется локального максимума или локального минимума функции $f: E \to \Real$, если $\exists U(x): \ \forall z \in U(x), f(z) \leq (\geq) f(x)$
\end{defs}

\begin{defs}[Формула Тейлора]
	$f$ - $n$-раз дифференцируема в \vtochke{x_0}, рассмотрим
	$$T_n(x,x_0) = f(x_0) + \frac{f^{\shtrih}(x_0)}{1!}\cdot (x-x_0)^{1} + \frac{f^{\shtrih\shtrih}(x_0)}{2!}(x-x_0)^{2} + \cdots + \frac{f^{n}(x_0)}{n!}(x-x_0)^{n} \text{-- n-й многочлен Тейлора}$$
	$\sledue f(x) = T_n(x,x_0) + r_n(x,x_0)$ -- \textbf{формула Тейлора}, где $r_n$ остаточный член формулы Тейлора
\end{defs}

\begin{proofs}[Формула Тейлора с остаточным членом в форме Шлемильха-Роша]
	$\pust$ ф-я $f$ $n$ - раз дифференцируема и непрерывна на $\kskobk{x_0,x}, \exists$ $(n+1)$-я производная $f$ на $\skobk{x_0,x}$; $\phi$ - функция непрерывная на $\skobk{x_0,x}$, диф-ма на $\skobk{x_0,x}$ и $\phi^{\shtrih} \neq 0 \sledue \exists \ \xi \in \skobk{x_0,x} :$
	$$ r_n(x,x_0)= \frac{f^{n+1}(\xi)\cdot (x-\xi)^{n}\cdot (\phi(x) - \phi(x_0)}{n!\phi^{\shtrih}(\xi)}$$

\begin{dokvo}
	$\rassmotr \ F(t) = T(x,t) \ t \in \kskobk{x_0,x}$

	\begin{gather*}
		F^{\shtrih}(t) = \skobk{f(t) + \frac{f^\shtrih(x)}{1!}\cdot (x-t) + \frac{f^{\shtrih\shtrih}(t)}{2!}\cdot (x-t)^{2} + \cdots + \frac{f^{(n)}(t)}{n!}\cdot (x-t)^{n}}^{\shtrih} = \\
		= \skobk{\cancel{f^{\shtrih}(t)} + \cancel{f^{\shtrih\shtrih}(t)\cdot (x-t)} - \cancel{f^{\shtrih}(t)} + \frac{f^{\shtrih\shtrih\shtrih}(t)}{2!}\cdot(x-t)^{2} - \cancel{f^{\shtrih\shtrih}(t)\cdot (x-t)} + \cdots + \frac{f^{(n+1)}(t)}{n!}\cdot (x-t)^n - \cancel{\cdots}} = \\
		= \frac{f^{(n+1)}(t)}{n!}\cdot (x-t)^{n}
	\end{gather*}
	То теореме Коши (т.к $F$ и $\phi$ непрерывны на $\kskobk{x_0,x}$, дифф-мы на $\skobk{x_0,x}$ и $\phi^\shtrih$ не обращается в нуль на $\skobk{x_0,x} \sledue \exists \ \xi \in \skobk{x_0,x}:$)
	\begin{gather*}
		\frac{\mathnaverh{=f(x)}{F(x)}-\mathnaverh{=T(x,x_0)}{F(x_0)}}{\phi(x)-\phi(x_0)} = \frac{F^{\shtrih}(\xi)}{\phi^{\shtrih}(\xi)}, \xi \in \skobk{x_0,x} \sledue f(x) - T(x,x_0) = r_n(x,x_0) \sledue \frac{r_n(x,x_0)}{\phi(x)-\phi(x_0)} = \\
		= \frac{f^{(n+1)}(\xi)}{\phi^{\shtrih}(\xi)\cdot n!}\cdot (x-\xi)^{n} \sledue r_n(x,x_0) = \frac{f^{(n+1)}(\xi) \cdot (x-\xi)^{n} \cdot (\phi(x)-\phi(x_0))}{\phi^{\shtrih}(\xi) \cdot n!}
	\end{gather*}
\end{dokvo}

\end{proofs}

\begin{sledsv}[Для условии теоремы выше справедливо]
	\begin{enumerate}
		\item $r_n(x,x_0) = \frac{f^{(n+1)}(\xi)}{(n+1)!}\cdot (x-x_0)^{n+1}$, $\xi \in \skobk{x_0,x}$ -- \textbf{Остаточный член в форме Лагранжа}
		\item $r_n(x,x_0) = \frac{f^{(n+1)}(\xi)}{n!}\cdot (x-\xi)^{n}\cdot (x-x_0)$, $\xi in \skobk{x_0, x}$ -- \textbf{Остаточный член в форме Коши}
		\item Если функция $f$ дополнение к условию теоремы дифф-ма $n+1$ раз на $\kskobk{x_0,x}$ и $n+1$ - непрерывна в \vtochke{x_0}, то $r_{n+1}(x,x_0) = \omaloe{(x-x_0)^{n+1}}$, при $x \to x_0$  -- \textbf{Остаточный член в форме Пеано}
	\end{enumerate}

	\begin{dokvo}
		остаточный член в форме Шлемильха-Роша $r_n(x,x_0) = \frac{f^{(n+1)}(\xi)  \cdot (\phi(x)-\phi(x_0))}{\phi^{\shtrih}(\xi) \cdot n!}\cdot (x-\xi)^{n}$
		\begin{enumerate}
			\item при $\phi(t) = (x-t)^{n+1}$
			$$\frac{f^{(n+1)}(\xi) \cdot (-(x-x_0)^{n+1})}{n! \cdot (-(n+1)\cdot \cancel{(x - \xi)^{n}})}\cdot \cancel{(x-\xi)^{n}}$$
			\item при $\phi(t) = x - t:$
			$$\frac{f^{(n+1)}(\xi)\cdot (-x + x_0)}{n! \cdot (-1)}\cdot (x - \xi)^n = \frac{f^{n+1}(\xi)}{n!}\cdot(x-\xi)^{n}\cdot (x - x_0)$$
			\item $f(t) = T_n\skobk{x,x_0} + \frac{f^{(n+1)}(\xi)}{(n+1)!}\cdot (x-x_0)^{n+1} =$
			$$T_n\skobk{x,x_0} + \frac{f^{(n+1)}(x_0)}{(n+1)!}\cdot (x-x_0)^{n+1} + \underbrace{\frac{f^{(n+1)}(\xi)}{(n+1)!}\cdot (x-x_0)^{n+1} - \frac{f^{(n+1)}(x_0)}{(n+1)!}\cdot (x-x_0)^{n+1}}_{\omaloe{(x-x_0)^{n+1}}} \oeq$$
			$\oeq T_{n+1}(x,x_0) + \omaloe{ (x-x_0)^{n+1}} f^{(n+1)}(\xi)-f^{(n+1)}(x_0)=\omaloe{1}$ при $x\to x_0$ так как $\xi \to x$ и $f^{(n+1)}$ непрерывная в точке $x_0$
		\end{enumerate}
	\end{dokvo}
\end{sledsv}

\begin{proofs}[Достаточные условия экстремума]
\zagolovok{Лемма} $\pust f$ - определена и непрерывна в некоторой окрестности $U(x_0)$, дифф-ма в $\prokol{U}(x_0)$, тогда, если $\forall x \in \fskobk{z \ | \ z \in U(x_0) : z < x_0}$ выполняется $f^{\shtrih}(x) > 0$, а $\forall \ x \in \fskobk{z \ | \ z \in U(x_0): z > x_0}$
$f^{\shtrih}(x) < 0$, то \vtochke{x_0} точка локального максимума (при обратных неравенствах минимума)
\begin{dokvo}
	$\pust$ выполняется $f^{\shtrih}(x)>0 \ \forall x \in \overbrace{\fskobk{z \ | \ z \in U(x_0) : z > x_0}}^{Z}$, $\rassmotr \ z_1, z_2 \in Z: z_1 < z_2$. По теореме $(\forall x \in \skobk{a,b}, f^{\shtrih}(x) \geq 0 \tittg$ функция $f$ - не убывает на $\skobk{a,b}) \sledue \ f(z_1) < f(z_2)$, далее  при $z_2 \to z_0 - 0$ из-за непрерывности функции \fx, $\predel{z_2 \to z_0 - 0}f(z_2) = f(z_0) \sledue$ $f(z_2) < f(z_0) \sledue$ $f(z_1)< f(x_0)$, аналогично, $\forall \ z_1, z_2 \in \fskobk{z \ | \ z \in U(x_0) : z < x_0}: z_1 > z_2, f(z_1) > f(x_0) \sledue \ x_0$ локальный максимум.
\end{dokvo}

\zagolovok{Лемма} $f$ $n$-раз непрерывна и диф-ма на $\skobk{a,b}$, $x_0 \in (a,b)$ - стационарная точка\footnote{в ней производная обращается в нуль или ноль} и $f^{\shtrih}(x_0)= f^{\shtrih\shtrih}(x_0) = \ldots = f^{(n-1)}(x_0)= 0$, а $f^{n}(x_0) \neq 0 \sledue$

если $n$ - нечетное, то $x_0$ - не является точкой локального экстремума

если $n$ - четное, то $x_0$ точка локального экстремума, при этом если $f^{(n)}(x_0) > 0$, то \vtochke{x_0} - точка локального минимума, а если $f^{n}(x_0) < 0$, то $x_0$ - точка локального максимума.
\begin{dokvo}
	По теореме о формуле Тейлора с остаточным членом в форме Пеано:
	\begin{gather*}
		f(x) = f(x_0) + \cancelto{= 0}{\frac{f^{\shtrih}(x_0)}{1!}}\cdot (x-x_0) + \cdots + \cancelto{= 0}{\frac{f^{(n)}(x_0)}{n!}}\cdot (x-x_0)^{n} + \omaloe{(x-x_0)^{n}}, x \to x_0 \\
		f(x) - f(x_0) = \frac{f^{(n)}(x_0)}{n!}\cdot (x - x_0)^{n} + \omaloe{(x - x_0)^{n}}, x \to x_0 \\
		f(x) - f(x_0) = (x - x_0)^{n} \cdot \skobk{\frac{f^{(n)}(x_0)}{n!} + \frac{\omaloe{(x-x_0)^{n}}}{(x-x_0)^{n}}}, x \to x_0
	\end{gather*}
	$\exists \delta > 0: \forall x: \Modul{x- x_0} < \delta, \Modul{\omaloe{1}} < \Modul{\frac{f^{(n)}(x_0)}{n!}}\cdot \frac{1}{2} \sledue$ знак выражения в скобках совпадает со знаком $\frac{f^{n}(x_0)}{n!}$

	\begin{enumerate}
		\item \zagolovok{$n$ - нечетное} $\sledue \ f(x) - f(x_0)$ меняет зна при прохождении $x$ через $x_0$ $\sledue x_0$ не точка экстремума
		\item \zagolovok{$n$ - четное} $\sledue$ не меняет, при этом имеет знак, совпадающий с $\frac{f^{n}(x_0)}{n!}$
		\begin{itemize}
			\item $f(x) - f(x_0)>0$,если $x \neq x_0, \Modul{x- x_0} < \delta$ и $f^{(n)}(x_0)>0 \sledue$ $x_0$ точка строгого локального минимума
			\item $f(x) - f(x_0)<0$,если $x \neq x_0, \Modul{x- x_0} < \delta$ и $f^{(n)}(x_0)<0 \sledue$ $x_0$ точка строгого локального максимума 
		\end{itemize}
	\end{enumerate}
\end{dokvo}
\end{proofs}
\newpage
