%!TEX root = ../report.tex"
\section{Вопрос 53: Алгоритм нахождения кратчайших расстояний от выделенной вершины до всех остальных вершин графа(Алгоритм Дейкстры), обоснование корректности и оценка вычислительной сложности.}

$\pust$ $G = (V, E)$ - связанный граф. Для каждого $e \in E$ определена функция $l(e) \geq 0$ - вес ребра. Доопределим $l$ для любой пары вершин $(u; v) \in  V$. Если $e= \fskobk{u;v} => l(u; v) = l(e)$ - если вершины смежные. Если $u, v$ - не смежные $=> l(u; v) = \infty. l(v; v) = 0.$
\\$\pust P$ - произвольная вершина, $(u, v)$ - цепь.
\\$P = u = v_0 e_1 v_1 ... e_t v_t = v$
\\Положим $\widetilde{l}(P)=\summa{i=1}{n}l(e_i)$ и $\widetilde{l}(u; v) = \min\limits_{\text{по всем (u,v)-цепям P}}\widetilde{l}(P)$
\\Требуется для заданной вершины $u \in V$ найти все значения $\widetilde{l}(u; v), v \in V$
\\Положим $v_0=u.
\\S \subset V$ - множество вершин, для которых найдены $\widetilde{l}$
\\$D$ - массив текущих расстояний от $u$ до $v; v \in V$ 

\zagolovok{Алгоритм Дейкстра}
\begin{enumerate*}
	\item $S=\fskobk{v_0}\\
	D(v_0) = 0\\
	D(v) = l(v_0; v); v \in V$

	\item Выберем ведущую вершину $w \in V\setminus S$, такую, что $D(w)=\min\limits_{v \in V\setminus S}D(v); D(w) < \infty$ и, добавляя $w$ к $S$, получим $S'=S \cup \fskobk{w}$

	\item Для $v \in V \setminus S$ вычислим $D'(v)$ следующим образом:\\
	Если $D(v) \leq D(w) + l(w; v)$, то $D'(v)=D(v)$\\
	Если $D(v) > D(w) + l(w; v)$, то $D'(v)=D(w) + l(w; v)$
	
	\item Если $S' = V =>$ СТОП\\
	Если $S' \neq V => S = S'; D = D'$ и возвращаемся к п. 2.
\end{enumerate*}

\begin{proofs}[Корректность алгоритма Дейкстры]
	Алгоритм Дейкстры корректно вычисляет расстояние от $v_0$ до остальных вершин графа со сложностью $O(|V|^2) = O(n^2)$
	\begin{dokvo}
		\begin{enumerate*}
			\item Алгоритм совершает $n$ итераций с номерами $i = 0; 1; ...; n-1$ Для каждой итерации обозначения: $S = S_i; D = D_i$\\
			$S_0; ...; S_{n-1}\\
			D_0; ...; D_{n-1}$
			\item Индукцией по $i$ докажем, что выполняется одновременно 2 условия:
			\begin{itemize}
				\item[a)] $\forall v \in S: D_i(v)$ - длина кратчайшей $(v_0; v)$-цепи и эта цепь вся целиком содержится в $S_i$
				\item[b)] $\forall v \in V\setminus S$, если $D_i(v) < \infty$, то $D_i(v) = $наименьшая длина $(v_0; v)$-цепи, которая целиком содержится в $S_i$, кроме последней вершины $(v)$
			\end{itemize}
			\item $i=0, S_i=S_0=\fskobk{v_0}, D_i=D_0, D_i(v)=l(v_0; v)$\\
			Если $v \in S => v = v_0, D_0(v_0) = 0$ и пункт а) выполняется\\
			$v \in V\setminus S_0$ и $D_0(v) < \infty => v$ - смежна с $v_0$ и $D_0(v)=l(v_0;v)$ - длина единственной цепи $(v_0;v)$, удовлетворяющей пункту b) $=>$ b) - выполняется
			\item Предположим, что a) и b) выполняются для i. Докажем для i+1.\\
			На i-й итерации: $S_i, D_i\\
			w: D_i(w) = \min\limits_{v \in V\setminus S_i}D(v), S_{i+1} = S_i \cup \fskobk{w}$\\
			Для $v \in V\setminus S_{i+1}: D_{i+1}(v) = min\fskobk{D_i(v); D_i(w) + l(w; v)} $\\
			Докажем пункт а):\\
			$\pust v \in S_{i+1}$. Тогда либо $v \in S$, либо $v=w. V\in S_i => D_{i+1}(v) = D_i(v)$ и значение $D_i(v)$ - не менялось.\\
			Тогда по предположению пункта а): $D_{i+1}(v) = D_i(v)$ - длина кратчайшей $(u;v)$-цепи и эта цепь лежит в $S_i =>$ лежит в $S_{i+1} =>$ для такой вершины $v$ пункт а) выполняется\\
			$\pust P$ - кратчайшая $(v_0;w)$-цепь; $l(P) < D_{i+1}(w). P = v_0v_1...v_t=w$ тогда существует 2 случая:
			\begin{itemize}
			 	\item $P$ целиком, кроме $w$ содержится в $S_i$
			 	\item  $P$ не содержится в $S_i$
			\end{itemize}
			Расмотрим первый случай: $v_0;...v_{t+1} \in S_i$\\
			По предположению индукции в пункте b): $D_i(w)$ - минимальная длина $(v_0;w)$-цепи, лежащей в $S_i$, кроме $w$ и $D_{i+1}(w) = D_i(w)$ - противоречие с выбором $P. => P$ - содержит вершины не из $S_i$\\
			$P=v_0v_1...y...v_t=w, y$ - первая вершина $P$, не лежащая в $S_i$. Рассмотрим подцепь $P'= v_0v_1...y$ - целиком в $S_i$, кроме $y$. Тогда $D_i(y) \leq \widetilde{l}(P') \leq \widetilde{l}(P) < D_i(w)$ - противоречие с выбором $w$ в соответствии с предположением пункта b) индукции. => п. а) - доказан.
			\item Предположим, что a) и b) верны для i. Докажем пункт b) для i+1.\\
			$\pust v \notin S_{i+1}$ и $D_{i+1}(v) < \infty. => D_{i+1}(v)$ - длина некоторой $(v_0;v)$-цепи $P$. Рассмотрим все кратчайшие цепи из $v_0$ в $v$. Эти цепи делятся на 3 класса:
			\begin{itemize}
				\item[1)] цепи, не содержащие $w. S_{i+1} = S_i\cup\fskobk{w}$
				\item[2)]  цепи $P$ в которых $w$ - предпоследняя вершина
				\item[3)]  цепи $P$ в которых $w$ - не предпоследняя вершина
			\end{itemize}
			Поскольку $D_{i+1}(v) < \infty =>$ все рассматриваемые цепи 1) - 3) типов обязаны целиком, за исключением $v$, содержаться в $S_{i+1}$
			\item $\pust P$- цепь 1) типа. Тогда $P$ не содержит $w$ и содержится в $S_{i+1}=S_i\cup\fskobk{w}$ (за исключением $v$) $=> P$ содержится в $S_i => D_{i+1}(v) = D_i(v)$, так как $w$ не участвует в переоценке $D_i(v) =>$ применяя пункт b) предположения индукции, получим то, что требовалось доказать.
			\item $\pust P$ - цепь 2 типа. Тогда $D_{i+1}(v) = D_i(w) + l(w; v) < D_i(v)$. При этом $D_{i+1}(w) = D_i(w)$ - длина кратчайшей $(v_0; w)$-цепи среди цепей, содержащихся в $S_{i+1}$ целиком, кроме вершины $v$
			\item $\pust P$ - цепь 3) типа. $P = v_0...w...uv. D_{i+1}(u) = D_i(u); u \in S_i => D_i(u)$ - длина кратчайшей $(v_0; u)$-цепи (по пункту a)) и эта цепь целиком содержится в $S_i. =>$ есть цепь из $v_0$ до $v$ 1 типа $=>$ пункт b) - верно.
			\item $n$ - шагов; на каждом шаге $n$ вершин $=> O(n^2)$ 
		\end{enumerate*}
	\end{dokvo}	
\end{proofs}
\newpage
