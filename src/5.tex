%!TEX root = ../report.tex"
\section{Вопрос 5: Числовой ряд. Сходящиеся ряды и их простейшие свойства. Признаки сходимости рядов с положительными членами (признаки сравнения, Даламбера, Коши). Абсолютно и неабсолютно сходящиеся ряды. Признаки
Дирихле, Абеля и Лейбница. Переместительное свойство абсолютно сходящихся рядов.}

\begin{defs}
	$\fskobk{a_n}$ - последовательность чисел из $\Complex$, $S_n = \summa{i=1}{n}a_i$ - $n$-я частичная сумма ряда. Числовой ряд - пара $\fskobk{a_n, S_n}, n \in \Natural$ Будем говорить, что ряд $\fskobk{a_n, S_n}$ \textbf{сходится}, если $\exists$ предел последовательности частичных сумм:$\exists \predel{n\to \infty}S_n$ - в этом случае данный предел - сумма ряда. Если предел $\nexists \predel{n\to \infty}S_n$, то ряд \textbf{расходится}, а $a_n$ - $n$-й  чден ряда;(обозначается $\summa{n=1}{\infty}a_n$)
\end{defs}

\begin{proofs}[Простейшие свойства сходящихся и расходящихся рядов]
\begin{enumerate*}
	\item Пусть $z_n \in \Complex$, $z_n = x_n + \imag{y_n}$, где $x_n, y_n \in \Real$, $\summa{n=1}{\infty}z_n$ -ряд, то он сходится $\tittg$ сходится $\summa{n=1}{\infty}x_n$ и $\summa{n=1}{\infty}y_n$
	\begin{dokvo}
		Следствие теоремы о том, что $\exists \predel{n \to \infty}z_n \tittg \exists \predel{n \to \infty}x_n \I \exists \predel{n \to \infty}y_n$
	\end{dokvo}

	\item $\summa{n=1}{\infty}a_m$ - числовой ряд и $\rassmotr$ $\summa{n=1}{\infty}b_n: \ b_n = a_{m+n}$, тогда эти ряды сходятся и расходятся одновременно
	\begin{dokvo}
	Следствие теоремы о возможности удаления из последовательности сходящегося конечного числа членов
	\end{dokvo}

	\item $\summa{n=1}{\infty}a_n$,$\summa{n=1}{\infty}b_n$ - сходящиеся ряды; $c$ -константа, тогда ряды $\summa{n=1}{\infty}(a_n+b_n)$ и $\summa{n=1}{\infty}c \cdot a_n$ - сходятся и $\summa{n=1}{\infty}(a_n+b_n) = \summa{n=1}{\infty}a_n + \summa{n=1}{\infty}b_n$, а $\summa{n=1}{\infty}c\cdot a_n = c \cdot \summa{n=1}{\infty}a_n$

	\item \textbf{Необходимый признак сходимости} Если ряд $\summa{n=1}{\infty}a_n$ сходится, то $\predel{n \to \infty}a_n = 0$
	\begin{dokvo}
		т.к $a_n = S_n - S_{n-1}$ и при этом $\predel{n\to \infty}S_n = \predel{n\to \infty}S_{n-1}$ т.к как ряд $\summa{n=1}{\infty}a_n$ сходится $\sledue \predel{n \to \infty}a_n=0$
	\end{dokvo}

	\item \textbf{Критерий сходимости Коши} Ряд $\summa{n=1}{\infty}a_n$ сходится $\tittg$ $\forall \ \epsilon > 0 \exists N \in \Natural: \forall n > N, n,p\in \Natural$ выполняется $\Modul{a_{n+1}+ \cdots + a_{n+p}} < \epsilon$
	\begin{dokvo}
		$\exists \ \predel{n\to \infty}S_n \tittg \ \forall \ \epsilon > 0 \exists \ N\in \Natural: \forall m,n > N \ \Modul{S_m - S_n} < \epsilon, \pust m > n \sledue m = n + p \sledue \Modul{S_m - S_n} = \Modul{a_{n+1}+ \cdots + a_{n+p}}$
	\end{dokvo}
\end{enumerate*}
\end{proofs}

\begin{proofs}[Ряды с неотрицательными членами]
	\begin{enumerate*}
		\item \textbf{Признак сравнения}:  $\summa{n=1}{\infty}a_n$, $\summa{n=1}{\infty}b_n$ ряды с неотрицательными членами. Начиная с некоторого $N$ выполняется $a_n \leq b_n \ \forall n > N \ \sledue$ из сходимости ряда $\summa{n=1}{\infty}b_n$ следует сходимость ряда $\summa{n=1}{\infty}a_n$, а из расходимости ряда $\summa{n=1}{\infty}a_n$ следует расходимость $\summa{n=1}{\infty}b_n$
		\begin{dokvo}
			$\pust$ $S_n$ -- $n$-я частичная сумма ряда $\summa{n=1}{\infty}a_n$,$S_{n}^{\shtrih}$ -- $n$-я частичная сумма ряда $\summa{n=1}{\infty}b_n$.

			Также отбросим конечное число членов $\summa{n=1}{\infty}a_n$ для которых не выполняется: $a_n \leq b_n$ (это можно сделать, согласно теореме выше (теорема о сходимости)) $\sledue S_n \leq S_{n}^{\shtrih}$ $\sledue$ считаем, что $a_n \leq b_n$ - выполняется $\forall n$

			Далее, т.к $\summa{n=1}{\infty}b_n$ сходится $\sledue \ \fskobk{S_{n}^{\shtrih}}$ -- ограничена $\sledue \ \fskobk{S_{n}}$ -- ограничена и следовательно $\summa{n=1}{\infty}a_n$ - сходится, а если $\summa{n=1}{\infty}a_n$ - расходится $\sledue \fskobk{S_n}$ -- неограничена $\sledue \fskobk{S_{n}^{\shtrih}}$ не ограничена $\sledue$ $\summa{n=1}{\infty}b_n$ -- расходится
		\end{dokvo}

		\item  \textbf{Признак сходимости Коши(радикальный)}: $\pust$ $\summa{n=1}{\infty}a_n$ ряд с неотрицательными членами и $\rassmotr$ $\overline{\predel{n\to \infty}}\sqrt[n]{a_n} = q \sledue$
		\begin{itemize}
			\item если $q < 1 \ \sledue \ \summa{n=1}{\infty}a_n$ -- сходится
			\begin{dokvo}
				$\rassmotr$ $\overline{\predel{n\to \infty}}\sqrt[n]{a_n} = q$ и $q < 1$ Выберем $\epsilon >0: q + \epsilon < 1 \sledue$ т.к вернхний предел - наибольший из частичных пределов $\sledue$ начиная с некоторого номера, $\sqrt[n]{a_n} < q + \epsilon \ \tittg \ a_n < \skobk{q + \epsilon}^{n}$, при этом $\summa{n=1}{\infty}\skobk{q + \epsilon}^{n}$ - сходится к бесконечно убывающей геометрической прогрессии $\sledue$ ряд $\summa{n=1}{\infty}a_n$ сходится по признаку сравнения.
			\end{dokvo}

			\item если $q > 1 \ \sledue \ \summa{n=1}{\infty}a_n$ -- расходится
			\begin{dokvo}
				$q > 1$, $\rassmotr \ \epsilon > 0: \ q - \epsilon > 1$, т.к $q$-верхний предел $\sledue \ \exists \fskobk{n_k}: \ \overline{\predel{k\to\infty}}\sqrt[n_k]{a_{n_k}} = q \ \sledue$ начиная с некоторого номера $a_{n_k} > q - \epsilon > 1 \ \sledue \ \predel{k \to \infty}a_{n_k} >1 \ \sledue$ не выполняется необходимый признак сходимости.
			\end{dokvo}

			\item если $q = 1 \ \sledue \ \summa{n=1}{\infty}a_n$ -- может как расходится так и сходиться
			\begin{dokvo}
				$q = 1: $ $\summa{n=1}{\infty}\frac{1}{n}$ --  не сходится, хотя $\predel{n \to \infty}\sqrt[n]{\frac{1}{n}} = \predel{n \to \infty}n^{- \cancelto{0}{\frac{1}{n}}} = \predel{n\to\infty}\frac{1}{n}^{\frac{1}{n}}= 1$

				$\summa{n=1}{\infty}\frac{1}{n^{2}}$ -- сходится, хотя $\predel{n \to \infty}\sqrt[n]{\frac{1}{n^{2}}} = \predel{n \to \infty}\skobk{\frac{1}{n}}^{\cancelto{0}{\frac{2}{n}}} = 1$
			\end{dokvo}
		\end{itemize}

		\item \textbf{Признак Даламбера}:
		$\pust$ $\summa{n=1}{\infty}a_n$ - ряд с положительными($\neq 0$) членами и начиная с некоторого номера выполняется $\frac{a_{n+1}}{a_n} < q < 1$, где $q = const \ \sledue$ $\summa{n=1}{\infty}a_n$ -- сходится. Если же начиная с некоторого номера $\frac{a_{n+1}}{a_n} > 1 \sledue \summa{n=1}{\infty}a_n$ -- расходится
		\begin{dokvo}
			$\pust$ начиная с некоторого номера $\frac{a_{n+1}}{a_n}\leq q < 1 \ \tittg \ \frac{q^{n+1}}{q^{n}} \sledue \ \rassmotr b_n = \summa{n=1}{\infty}q^{n}$ - бесконечно убывающая геометрическая прогрессия $\skobk{q < 1} \sledue$ по признаку, так как $\summa{n=1}{\infty}b_n$ - сходится $\sledue$ $\summa{n=1}{\infty}a_n$ также сходится $\sledue$ этот случай доказан

			$\frac{a_{n+1}}{a_n} > 1 \tittg \ \frac{a_{n+1}}{a_n} > \frac{1^{n+1}}{1^{n}} \ \sledue \ \rassmotr \ \summa{n=1}{\infty}b_n$, где $b_n = 1 \ \forall n \ \sledue$ т.к $\summa{n = 1}{\infty}b_n$ -- расходится, то и $\summa{n = 1}{\infty}a_n$ -- расходится.
		\end{dokvo}
	\end{enumerate*}
\end{proofs}

\zagolovok{Абсолютно и неабсолютно сходящиеся ряды}

$a_n \in \Real \ILI \Complex$

\begin{defs}
	Числовой ряд $\summa{n=1}{\infty}a_n$ называется \textbf{АСР(абсолютно сходящимся рядом)}, если сходится ряд $\summa{n=1}{\infty}\Modul{a_n}$
\end{defs}

\begin{defs}
	Числовой ряд $\summa{n=1}{\infty}a_n$ называется \textbf{УСР(условно сходящимся рядом)}, если сходится ряд $\summa{n=1}{\infty}a_n$, а $\summa{n=1}{\infty}\Modul{a_n}$ -- расходится
\end{defs}

\begin{proofs}[Признак Лейбница]
	$\pust$ $\fskobk{a_n}$ - последовательность: $\levfigurn{ a_n \geq 0 \ \forall n \in \Natural \\ \predel{n\to \infty}a_n = 0 \\  \fskobk{a_n}\text{ не возрастает}} \sledue$ ряд $\summa{n=1}{\infty}\skobk{-1}^{n-1}\cdot a_n$ -сходится
	\begin{dokvo}
		$\rassmotr$ $S_{2n} = \mathvniz{\geq 0}{\skobk{a_1 - a_2}} + \mathvniz{\geq 0}{\skobk{a_3 - a_4}}+ \cdots + \mathvniz{\geq 0}{\skobk{a_{2n-1} - a_{2n}}} \geq 0$ При этом $S_{2n + 2} \geq S_{2n} \geq 0 \ \sledue$ $S_{2n}$ не убывает. (игра со скобками) $S_{2n} = a_1 - \mathvniz{\geq 0}{\skobk{a_2 - a_3}} - \cdots - \mathvniz{\geq 0}{\skobk{a_{2n-2} - a_{2n-1}}} - \mathvniz{\geq 0}{a_{2n}} \sledue$ $S_{2n} \geq a_1 \ \sledue$ $S_{2n}$ - не убывает и ограничена сверху $\sledue \ \exists \predel{n\to \infty}S_{2n}$ т.к $S_{2n + 1} = S_{2n} + a_{2n+1}$ и т.к $\predel{n\to \infty}a_n = 0 \ \sledue \ \predel{n\to \infty}S_{2n+1} = \predel{n\to \infty}S_{2n} \ \sledue \ \exists\predel{n\to \infty}S_{n} \ \sledue$
	\end{dokvo}
\end{proofs}

\begin{lemma}[Преобразование Абеля]
 $$\summa{i=1}{n}a_i \cdot b_i = \summa{i=1}{n=1}\skobk{a_i - a_{i+1}}\cdot B_i + a_n\cdot B_n,\text{ где } B_i = \summa{k=1}{i}b_k$$
\end{lemma}

\begin{proofs}[Признак Дирихле]
	$\rassmotr$ $\summa{n=1}{\infty}a_{n}b_{n}$, где $\fskobk{a_n}$ - монотонная последовательность, $\predel{n\to \infty}a_n = 0$, Частичные суммы ряда $\summa{n=1}{\infty}b_{n}$ ограничены $\sledue$ ряд $\summa{n=1}{\infty}a_{n}b_{n}$ сходится.
	\begin{dokvo}
		$\rassmotr$ выполнение критерия Коши ($\forall  \epsilon > 0 \ \exists  N \in \Natural:  \forall  n > N \I  \forall  p > 0,$ $\Modul{a_{n+1}+ \cdots + a_{n+p}} < \epsilon$)

		$\Modul{\summa{i=1}{p}{a_{n+i} \cdot b_{n+i}}} \naverh{пр-я Абеля}= \Modul{\summa{i=1}{p-1}\skobk{a_{n+i}-a_{n+i+1}}B_i + a_{n+p}B_{n+p}} \leq \Modul{\summa{i=1}{p-1}\skobk{a_{n+i}-a_{n+i+1}}B_i} + \Modul{a_{n+p}B_{n+p}} \oleq$
		Так как $\Modul{B_i} = \Modul{\summa{k=1}{n+i}b_k - \summa{k=1}{n}b_k} \leq \Modul{\summa{k=1}{n+i}b_k} - \Modul{\summa{k=1}{n}b_k} \leq 2M$,
		а $\Modul{\summa{i=1}{p-1}\overbrace{\skobk{a_{n+i} - a_{n+i+1}}}^{\text{Одного знака}}B_i} \leq \Modul{\summa{i=1}{p-1}\overbrace{\skobk{a_{n+i} - a_{n+i+1}}}^{\vniz{вз-е унич-е членов}{\Modul{a_{n+1} - a_{n+p}}}}B_i} \cdot 2M$
		$\oleq \ 2M \cdot \skobk{\Modul{a_{n+1} + 2\cdot \Modul{a_{n+1}}}} < 6 \cdot M \cdot \epsilon$
	\end{dokvo}
\end{proofs}

\begin{proofs}[Признак сходимости Абеля]
	Дан ряд $\summa{n=1}{\infty}\skobk{a_n \cdot b_n}: \fskobk{a_n}$ -- монотонная ограниченная последовательность, а $\summa{n=1}{\besk}$ -- сходится $\sledue$ $\summa{n=1}{\besk}a_n b_n$ -- сходится
	\begin{dokvo}
		Выполнение критерия Коши:
		\begin{gather*}
			\Modul{\summa{i=1}{p}a_{n+1}b_{n_i}} \naverh{пр-е Абеля}= \Modul{\summa{i=1}{p-1}\skobk{a_{n+i} -a_{n+i+1}}B_i + a_{n+p}B_p} \leq \Modul{\summa{i=1}{p-1}\underbrace{\skobk{a_{n+i} -a_{n+i+1}}}_{\vniz{от мон-ти крит Коши}{\text{один-й знак в зав-ти}}}B_i}  + \Modul{ a_{n+p}B_p} \oleq \\
			\summa{n=1}{\besk}b_n - \text{сходится } \tittg \ \forall \epsilon > 0 \ \exists N \in \Natural: \forall n > N, p \in \Natural \ \Modul{\underbrace{b_{n+1} + \cdots + b_{n+p}}_{B_p}} < \epsilon \ \sledue \Modul{a_{n+p}B_p} \leq \Modul{a_{n+p} \epsilon}\\
			\Modul{\summa{i=1}{p-1}\skobk{a_{n+i} - a_{n+i+1}}B_i} \leq \epsilon \cdot \Modul{\underbrace{\summa{i=1}{p-1}\skobk{a_{n+i} - a_{n+i+1}}}_{a_{n+1} + a_{n+p}}} \leq \epsilon \cdot \skobk{\Modul{a_{n+1}} + \Modul{a_{n+p}}}, \\
			\text{т.к.} \fskobk{a_n} \text{ ограничена } \sledue \Modul{a_n} \leq M \ \forall n \in \Natural \ \sledue \\
			\oleq \ 2M\epsilon + M\epsilon = 3M\epsilon \sledue \ \text{по критерию Коши ряд расходится}
		\end{gather*}
	\end{dokvo}
\end{proofs}

\begin{lemma}[Переместительное свойство абсолютно сходящихся рядов]

	$\pust$ ряд $\summa{n=1}{\besk}a_n$ - АСР, $\sigma: \Natural \to \Natural$ -- биективное отображение $\sledue$ ряд $\summa{n=1}{\besk}a_{\sigma(n)}$ - АСР и $\summa{n=1}{\besk}a_n = \summa{n=1}{\besk}a_{\sigma(n)}$
\end{lemma}
\newpage
