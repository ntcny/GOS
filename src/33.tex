%!TEX root = ../report.tex"
\section{Вопрос 33:
Основные понятия математической статистики.
Точечные оценки неизвестных значений параметров распределений.
Несмещенность и асимптотически несмещенность, состоятельность оценок.
Пример: проверить оценку $\theta^*(x_1,\ldots,x_n) = (x_1+x_n)/2$ параметра $\theta=\bf{M}\xi$ на
несмещенность и состоятельность.
Свойства статистик $\overline{x}$ и $S^2, \ S_0^2$ как оценок неизвестных значений, соответственно,
математического ожидания и дисперсии СВ.
}

\begin{defs}[Статистическая ситуация (структура)]
  Говорят, что \textit{имеет место статистическая ситуация (или задана статистическая структура)},
  если выполняются следующие условия:
  \begin{enumerate}
    \item Для некоторого фиксированного $n \in N$ проводится серия из $n$ независимых экспериментов,
    в результате которых получаются исходы $\omega_1,\ldots,\omega_n$. (они известны, доступны).
    \item На пространстве возможных исходов (элементарных событий) $\Omega$ этих
    экспериментов совместно с некоторой выбранной естесственной $\sigma$-алгеброй событий $\mathfrak{A}$
    определена (и полностью нам известна как функция) некоторая СВ $\xi \ : \ \Omega \to R$ и, следовательно,
    нам известны измереннные (наблюдённые) значения этой СВ: $x_1=\xi(\omega_1),\ldots,x_n=\xi(\omega_n) \ (x_i \in R)$
    \item Вероятностная функция $P(.)$ на измеримом пространстве $(\Omega, \mathfrak{A})$ полностью или хотя бы частично
    неизвестна, и, следовательно, неизвестно (полностью или частично) распределение вероятностей измеряемой СВ $\xi$.
    \item По результатам $x_1,\ldots,x_n$ измерений СВ $\xi$ требуется оценить неизвестное распределение СВ $\xi$
    и (или) решить иную задачу мат.статистики.
  \end{enumerate}
\end{defs}

\begin{defs}[Независимость экспериментов]
  Эксперименты, проводимые в рамках статистической структуры, называются независимыми,
  если вероятностно-статистические свойства вектора $\overrightarrow{x}=(x_1,\ldots,x_n)$ результатов
  $n$ измерений СВ $\xi$ совпадают с соответствующими вероятностно-стохастическими свойствами результата
  $\overrightarrow{x} = \overrightarrow{\xi}(\omega)=(\xi_1(\omega, \ldots, \xi_n(\omega)))$ одного наблюдения за
  случайным вектором $\overrightarrow{\xi}=(\xi_1,\ldots,\xi_n)$ с независимыми и одинаково распределенными
  (одинаково между собой и с измеряемой СВ $\xi$) распределенным компонентами $\xi_1,\ldots,\xi_n$.
\end{defs}

Следовательно, вектор результатов $n$ независимых наблюдений за СВ $\xi$ можно считать (в мат.статистике так всегда и поступают)
результатом одного наблюдения за случайным вектором с независимыми, одинаково распределенными координатами.

\begin{defs}[Генеральная совокупность]
  Множество $E=\xi(\Omega)$ возможных значений измеряемой СВ $\xi$ называется \textit{генеральной совокупностью}.
\end{defs}

\begin{defs}[Выборочное пространство]
  Множество $E^n=\overrightarrow{\xi}(\Omega)$ возможных значений случайного вектора $\overrightarrow{\xi}$
  независимыми, одинаково распределенными с $\xi$ координатами называется \textit{выборочным пространством}.
  ($E^n \subseteq R$).
\end{defs}

\begin{defs}[Гипотетическое распределение]
  Полностью (частично) неизвестное распределение измеряемой СВ $\xi$ называется \textit{теоретическим (гипотетическим) распределением} этой СВ.
  ($E^n \subseteq R$).
\end{defs}

\begin{defs}[Случайная выборка]
  Исходный для стат.структуры вектор $\overrightarrow{x} = (x_1,\ldots,x_n)$ результатов $n$ независимых наблюдений
  за СВ $\xi$ или одного наблюдения за случайным вектором $\overrightarrow{\xi} = (\xi_1,\ldots,\xi_n)$ называется
  \textit{случайной (или статистической) выборкой} объема $n$ из выборочного пространства $E^n \ (R^n)$ с гипотетическим
  распределением $F_{\xi}(x)$ СВ $\xi$, или просто случайной выборкой из гипотетического распределения $F_{\xi}(x)$.
  ($E^n \subseteq R$).
\end{defs}

\begin{defs}[Параметрический (непараметрический) класс]
  Если о неизвестном распределении $F_{\xi}(x)$ измеряемой СВ $\xi$ известно, к какому параметрическому классу
  (семейству) распределений оно принадлежит, то есть $F_{\xi}(x)=F(x,\theta_1,\theta_2,\ldots), \
  \overrightarrow{\theta}=(\theta_1, \theta_2,\ldots,\theta_k) \subseteq R^k$, то раздел мат.статистики как науки,
  занимающийся решением различных задач в данном случае, называется \textit{параметрическим}. В противном случае, когда про
  распределение СВ $\xi$ вообще ничего не известно, раздел мат.статистики называется \textit{непараметрическим}.
\end{defs}

\begin{defs}[Статистика]
  Статистикой в мат.статистике как науке называется триединое понятие:
  \begin{enumerate}
    \item Это действительное число (из $R$), равное результату применения к случайной выборке
    $\overrightarrow{x}$ некоторой борелевской функции $T(.) \ (T(\overrightarrow{x}) \in R)$.
    \item Сама эта борелевская функция $T(\overrightarrow{x}, \ \overrightarrow{x} \in R^n)$
    \item СВ $\eta=T(\overrightarrow{\xi})$, где $\overrightarrow{\xi}=(\xi_1,\ldots,\xi_n), \ \xi_i \sim \xi_j \ \forall i,j \in \overline{1,n}$
  \end{enumerate}
\end{defs}

\begin{example}
  \begin{enumerate}
    \item $\overline{x} = \frac{x_1+\ldots+x_n}{n}$ - выборочное среднее.
    \item $S^2 = \frac{1}{n}\sum\limits_{i=1}{n}(x_i-\overline{x})^2 \
    = \ (\frac{1}{n}\sum\limits_{i=1}{n}x_i^2) - \overline{x}^2$  - выборочная дисперсия (первая, неисправленная).
    \item $S_0^2 = \frac{1}{n-1}\sum\limits_{i=1}^n(x_i-\overline{x})^2$ - вторая (исправленная) выборочная дисперсия.
    \item $\alpha_k = \frac{1}{n}\sum\limits_{i=1}^n x_i^k$ - выборочный начальный момент порядка $k, \ k \in \mathbb{N}$.
    \item $\mu_k = \frac{1}{n}\sum\limits_{i=1}^n (x_i - \overline{x})^k$ - выборочный центральный момент порядка $k$.
    \item $\S = \sqrt{S^2}$ - выборочное стандартное отклонение.
  \end{enumerate}
\end{example}

\begin{defs}[Вариационный ряд]
  Вариационным рядом случайной выборки $\overrightarrow{x}$ называется вектор $(x_{(1)},\ldots,x_{(n)})$,
  состоящий из тех же элементов случайной выборки $\overrightarrow{x}$, но упорядоченных по неубыванию:
  \begin{enumerate}
    \item $\{x_1,\ldots,x_n\} = \{x_{(1)},\ldots,x_{(n)}\}$
    \item $x_{(1)} \leqslant x_{(x)} \leqslant \ldots \leqslant x_{(n)}$
  \end{enumerate}
\end{defs}

\begin{defs}[Размах выборки]
  Статистика $R = x_{(n)} - x_{(1)}$ называется \textit{размахом выборки}.
\end{defs}

\begin{defs}[Параметрическая статистика]
  \begin{enumerate}
    \item Пусть задана непараметрическая стат.структура: имеется случайная выборка
      $\overrightarrow{x}=(x_1,\ldots,x_n)$ - реализация случайного вектора
      $\overrightarrow{\xi} = (\xi_1,\ldots,\xi_n)$ с независимыми
      одинаково распределенными координатами $\xi_1,\ldots,x_n$.
    \item Известен некоторый параметрический класс распределений, которому
    принадлежит неизвестное гипотетическое распределение СВ $\xi$, то есть
    $F_{\xi}(x) = F(x,\theta_1,\theta_2,\ldots)$, где $F$ - известный тип распределения,
    а $\overrightarrow{\theta} = (\theta_1,\ldots,\theta_k) \in H$.
  \end{enumerate}
\end{defs}

\begin{example}
    \begin{enumerate}
      \item Первая нормальная статистическая структура $\xi \sim N(\theta,\sigma^2),
      \ \theta=a=\bf{M}\xi$ - параметр с неизвестным значением, $\sigma^2$ - известное значение,
      $\theta \in \mathbb{R} \ \sledue$ - параметрическое множество.
    \end{enumerate}
\end{example}

\begin{defs}[Точечные оценки]
  Пусть $\theta = \theta_i \ i=1,\ldots,n$ - один из параметров с неизвестным значением
  распределения измеряемой СВ $\xi$. \textit{Точечной (числовой) оценкой} парамтера $\theta$
  соответствующей случайной выборки $\overrightarrow{x} \in \mathbb{R}^n$ называется произвольная
  статистика $\widehat{\theta}(\overrightarrow{x})$. Таким образом, оценка параметра - это
  триединый объект:
  \begin{enumerate}
      \item Число из $\mathbb{R}$ - значение борелевской функции $\widehat{\theta}(\overrightarrow{x})$ от
      конкретной выборки $\overrightarrow(x)$.
      \item Сама борелевская функция $\widehat{\theta}(\overrightarrow{x}), \ \overrightarrow{x} \in \mathbb{R}$
      \item СВ $\eta = \widehat{\theta}(\overrightarrow{\xi})$,
      где $\overrightarrow{\xi}=(\xi_1,\ldots,\xi_n)$.
  \end{enumerate}
\end{defs}

Произвольная статистика может оказаться "плохой". В статистике
нужны "хорошие" оценки параметроа, для этого нужеы критерии хороших оценок.
В статистике для этого используются следующие основные критерии оценок:
\begin{enumerate}
    \item Несмещенность (асимптотически несмещенность)
    \item Состоятельность
    \item Оптимальность
\end{enumerate}

\begin{defs}[Несмещенность]
  Статистика $\widehat{\theta}(\overrightarrow{x}), \ \overrightarrow{x} \in \mathbb{R}^n$
  называется \textit{несмещенной оценкой} параметра $\theta=\theta_i, \ i=1,\ldots,k$, если у
  этой статистики, понимаемой как СВ $widehat{\theta}(\overrightarrow{\xi})$ существует матожидание
  и это матожидание тожедественно равно оцениваемому параметру $\theta$, каким бы ни было значение вектора
  параметров $\overrightarrow{\theta} = (\theta_1,\ldots,\theta_k) \in \encircle{H} \subset \mathbb{R}^k$, то есть
  $\bf{M}_{\overrightarrow{\theta}}\widehat{\theta}(\overrightarrow{\xi})\equiv\theta \
  \forall \overrightarrow{\theta} = (\theta_1,\ldots,\theta_k) \in \encircle{H} \subset \mathbb{R}^k$, при этом
  индекс $\overrightarrow{\theta}$ при обозначении матожидания означает, что матожидание СВ
  $\widehat{\theta}(\overrightarrow{\xi})$ вычисляется в предположении, что координаты $\xi_1,\ldots,\xi_n$
  случайного вектора $\overrightarrow{\xi}$ независимы и одинаково распределены между собой и со СВ $\xi$.
\end{defs}

\begin{defs}[Асимптотически несмещенность]
  Статистика $\widehat{\theta}(\overrightarrow{x}), \ \overrightarrow{x}=(x_1,\ldots,x_n)$ называется
  \textit{асимптотически несмещенной оценкой} параметра $\theta = \theta_i, \ i=1,\ldots,k$, если выполняются
  следующие условия:
  \begin{enumerate}
      \item Эта статистика естесственным образом определена $\forall n \in \mathbb{N}$
      \item $\forall n \in \mathbb{N} \ \exists \bf{M}_{\overrightarrow{\theta}}\widehat{\theta}(\overrightarrow{\xi})$
      \item При $n \to \infty \bf{M}_{\overrightarrow{\theta}}\widehat{\theta}(\overrightarrow{\xi}) \to \theta \
      \forall \overrightarrow{\theta} \in \encircle{H} \subset \mathbb{R}^k$.
  \end{enumerate}
\end{defs}

\begin{defs}[Состоятельность]
  Статистика $\widehat{\theta}(\overrightarrow{x}), \ \overrightarrow{x}=(x_1,\ldots,x_n)$ называется
  \textit{состоятельной оценкой} параметра $\theta = \theta_i, \ i=1,\ldots,k$, если выполняются
  следующие условия:
  \begin{enumerate}
      \item Эта статистика естесственным образом определена $\forall n \in \mathbb{N}$
      \item Эта статистика, понимаемая как СВ $\widehat{\theta}(\overrightarrow{\xi}), \
      \overrightarrow{\xi} = (\xi_1,\ldots,\xi_n), \ \xi_i \sim \xi, \ \xi_1,\ldots,\xi_n$ - независимые,
      при $n \to \infty$ сходится по вероятности к оцениваемому параметру $\theta$, каким бы ни был вектор параметров
      $\overrightarrow{\theta}=(\theta_1,\ldots,\theta_k) \in \mathbb{R}^k$, то есть
      $\widehat{\theta}(\overrightarrow{\xi}) \xrightarrow[]{P_{\overrightarrow{\theta}}(.)}$ при $n \to \infty \
      \forall \overrightarrow{\theta} \in \encircle{H} \subset \mathbb{R}^k$, то есть
      $\forall \varepsilon > 0 \ \exists \predel{n \to \infty}P_{\overrightarrow{\theta}}\{\Modul{\widehat{\theta}(\overrightarrow{\xi}) - \theta}
      \geqslant \varepsilon\}=0 \ \forall \overrightarrow{\theta} \in \encircle{H} \subset \mathbb{R}^k$
  \end{enumerate}
\end{defs}

Свойство состоятельности оценки означает, что при большое длине выборки вероятность любого
(даже самого малого) отличия оценки от оцениваемого параметра становится ничтожно малой.

\begin{defs}[]
  Пусть в рамках некоторой параметрической стат.структуры
   $\widehat{\theta}(\overrightarrow{x})$ - некоторая статистика, используемая в качестве
  оценки параметра $\theta$, тогда следующие величины, характеризующие качество
  оценивания этого статистического параметра $\theta$, имеют следующие названия:
  \begin{enumerate}
      \item $b_n(\theta) = \bf{M}_{\overrightarrow{\theta}}\widehat{\theta}(\overrightarrow{\xi}) - \theta$ -
      \textit{смещение оценки относительно оцениваемого параметра $\theta$}
      \item $d_n(\theta) = \bf{D}_{\overrightarrow{\theta}}\widehat{\theta}(\overrightarrow{\xi})$ - \textit{дисперсия оценки}
      \item $\delta_n(\theta) = \bf{M}_{\overrightarrow{\theta}}(\widehat{\theta}(\overrightarrow{\xi}) - \theta)^2$ -
      \textit{среднее квадратичное отклонение оценки}.
  \end{enumerate}
\end{defs}

\begin{example}
    Будет после консультации..
\end{example}

\begin{proofs}[Свойства некоторых стандартных статистик]
  Пусть $\overrightarrow{x} = (x_1,\ldots,x_n), \ \xi \sim F_{\xi}(t, \overrightarrow{\theta}, \
  \overrightarrow{\theta} \in \encircle{H} \subset \mathbb{R}^k)$ - некоторая стат.структура, в которой параметр $\theta_1 = \bf{M}\xi, \
  \theta_2 = \bf{D}\xi$, тогда
  \begin{enumerate}
      \item Статистика "выборочное среднее" $\overline{x} = \frac{x_1 + \ldots + x_n}{n}$ является несмещенной и состоятельной
      оценкой параметра неизвестного матожидания СВ $\xi \ (\theta = \theta_1=\bf{M}\xi)$.
      \item Статистика "первая (неисправленная) выборочная дисперсия" $S^2 = \frac{1}{n}\sum\limits_{i=1}^n(x_i-\overline{x})^2$
      является асимптотически несмещенной и состоятельной оценкой неизвестной дисперсии.
      \item Статистика "вторая (исправленная) выборочная дисперсия" $S_0^2 = \frac{1}{n-1}\sum\limits_{i=1}^n(x_i-\overline{x})^2$
      является несмещенной и состоятельной оценкой неизвестной дисперсии.
  \end{enumerate}

  \begin{dokvo}
      \begin{enumerate}
        \item $\xi \sim F_{\xi}(t,\theta_1,\ldots,\xi_k), \ \theta = \bf{M}\xi; \
        \widehat{\theta}(\overrightarrow{x}) = \overline{x} = \frac{x_1 + \ldots + x_n}{n}$. \\
        a) Несмещенность.\\
        Представим эту оценку как СВ $\widehat{\theta}(\overrightarrow{\xi})_{\diagup \overrightarrow{x} = \overrightarrow{\xi}} =
        \skobk{\frac{x_1+\ldots+x_n}{n}}_{\diagup x_i=\xi_i} = \frac{\xi_1+\ldots+\xi_n}{n}$\\
        $\bf{M}_{\overrightarrow{\theta}}\widehat{\theta}(\overrightarrow{\xi}) =
        \bf{M}_{\overrightarrow{\theta}}\skobk{\frac{\xi_1+\ldots+\xi_n}{n}} = \bf{M}\skobk{\frac{\xi_1+\ldots+\xi_n}{n}}$ =
        /$\xi_1,\ldots,\xi_n$-независимые, $\xi_i \sim \xi \ \forall i \in \overline{1,n}$/ =
        $\frac{\bf{M}_{\overrightarrow{\theta}}\xi_1 + \ldots + \bf{M}_{\overrightarrow{\theta}}\xi_n}{n}$ = /$\bf{M}_{\overrightarrow{\theta}}\xi_i =
        \bf{M}\xi_i = \bf{M}\xi = \theta$/ = $\frac{\theta + \ldots + \theta}{n}
        = \theta \ (=\bf{M}\xi) \ \forall \overrightarrow{\theta} \in \encircle{H} \subset \mathbb{R}^k$\\
        б) Состоятельность.\\
        По определению $\widehat{\theta}(\overrightarrow{\xi})\xrightarrow[]{P_{\overrightarrow{\theta}}}\theta, \ n \to \infty \ (\forall \theta \in \encircle{H})$
        $\frac{\xi_1+\ldots+\xi_n}{n}\xrightarrow[]{{P_{\overrightarrow{\theta}}}}, \ n \to \infty$, то есть $\forall \varepsilon > 0$ нужно убедиться, что
        $\exists\predel{n \to \infty}P_{\overrightarrow{\theta}}\{\Modul{\frac{\xi_1+\ldots+\xi_n}{n} - \theta} \geqslant \varepsilon\} = 0$.
        $\forall \varepsilon > 0 \ \exists\predel{n \to \infty}P_{\overrightarrow{\theta}}\{\Modul{\frac{\xi_1+\ldots+\xi_n}{n}-\bf{M}\xi} \geqslant \varepsilon\}$
        $\ \frac{\xi_1+\ldots+\xi_n}{n}-\bf{M}\xi \ = \ \frac{(\xi_1+\ldots+\xi_n) - n\bf{M}\xi}{n} \ = \ \frac{S_n-\bf{M}S_n}{n}$\\
        $\forall \varepsilon > 0 \ \exists\predel{n \to \infty}P_{\overrightarrow{\theta}}\{\Modul{\frac{S_n-\bf{M}S_n}{n}} \geqslant \varepsilon\} = 0$
        (потому что выполняется закон больших чисел)

        \item $\theta = \theta_2 = \bf{D}\xi \ (\exists \bf{D}\xi)$\\
        $\widehat{\theta}(\overrightarrow{x}) = S^2 = \frac{1}{n}\sum\limits_{i=1}^n(x_i-\overline{x})^2 =
        \skobk{\frac{1}{n}\sum\limits_{i=1}^n x_i^2} - \overline{x}^2$. \\
        а) Несмещенность. \\
        $\widehat{\theta}(\overrightarrow{x}) = \skobk{\frac{1}{n}\sum\limits_{i=1}^n\xi_i^2} - \overline{\xi}^2$
        ($\overline{\xi}^2 = \skobk{\frac{\xi_1+\ldots\xi_n}{n}}^2$)
        $\bf{M}_{\overrightarrow{\theta}}\widehat{\theta}(\overrightarrow{\xi}) =
        \bf{M}_{\overrightarrow{\theta}}[(\frac{1}{n}\sum\limits_{i=1}^n\xi_i^2)-(\frac{1}{n}\sum\limits_{i=1}^n\xi_i)^2]
        =\frac{1}{n}\sum\limits_{i=1}^n\bf{M}\xi_i^2-\frac{1}{n^2}\bf{M}\skobk{\sum\limits_{i=1}^n\xi_i}^2\oeq$\\
        $\bf{M}_{\overrightarrow{\theta}}\xi_i^2=\bf{D}_{\overrightarrow{\theta}}\xi_i+\skobk{\bf{M}_{\overrightarrow{\theta}}\xi_i}^2=\theta + a^2 \ \forall i=1,\ldots,n$\\
        $\bf{M}_{\overrightarrow{\theta}}[\sum\limits_{i=1}^n\xi_i]^2 = \
        \bf{M}_{\overrightarrow{\theta}}\skobk{\sum\limits_{i=1}^n\xi_i^2+2\sum\limits_{i<j}\xi_i\xi_j} = \
        \sum\limits_{i=1}^n{\bf{M}_{\overrightarrow{\theta}}\xi_i^2}_{\diagup \theta + a^2} + {2\sum\limits_{i<j}\xi_i\xi_j}_{\diagup a\cdot a=a^2} =
        n(\theta + a^2) + 2a\sum\limits_{i<j}1_{\diagup C_n^2=\frac{n(n-1)}{2}} =
        n(\theta + a^2)+a^2n(n-1) = n\theta + n^2a^2$\\
        $\oeq \frac{1}{n}n(\theta + a^2) - \frac{1}{n^2}(n\theta + n^2a^2) = \theta+a^2-\frac{\theta}{n}-a^2 = \frac{n-1}{n}\theta \neq \theta \
        \sledue \ \widehat{\theta}(\overrightarrow{x})=S^2$ - смещенная. При этом $\bf{M}_{\overrightarrow{\theta}}S^2=\frac{n-1}{n} \to
        \theta \ \text{при} n \to \infty \ \forall \overrightarrow{\theta} \in \encircle{H} \subset \mathbb{R}^k \ \sledue \ S^2$ - асимптотически несмещенная.\\
        б) Состоятельность.\\
        Воспользуемся леммой о достаточном условии состоятельности:
        $\levfigurn{\exists b_n(\theta) \to 0, \ n \to \infty \\ \exists d_n(\theta) \to 0, \ n \to \infty \\ }$ \\
        $\exists b_n(\theta) = \bf{M}_{\overrightarrow{\theta}}S^2 - \theta = \frac{n-1}{n}\theta - \theta = -\frac{\theta}{n} \to 0, \ n \to \infty$ \\
        $d_n(\theta) = \bf{D}_{\overrightarrow{\theta}}S^2 = $ /без док-ва/ $O(\frac{1}{n}) \to 0, \ n \to \infty \ \sledue$ на основании
        леммы статистика $S^2$ является состоятельной оценкой параметра $\theta = \bf{D}\xi$

        \item $S_0^2 = \frac{1}{n-1}\sum\limits_{i=1}n(x_i-\overline{x})^2 = \frac{n}{n-1}S^2$\\
        а) Несмещенность.\\
        $\bf{M}_{\overrightarrow{\theta}}(\overrightarrow{\xi}) = \bf{M}_{\overrightarrow{\theta}}S_0^2(\overrightarrow{\xi}) =
        \bf{M}_{\overrightarrow{\theta}}(\frac{n}{n-1}S^2) = \frac{n}{n-1}\frac{n-1}{n}\theta = \theta \ \forall \overrightarrow{\theta} \ \sledue$
        оценка несмещенная.\\
        б) Состоятельность.\\
        Аналогично $S^2$.


      \end{enumerate}
  \end{dokvo}
\end{proofs}
