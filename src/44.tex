%!TEX root = ../report.tex$
\section{Вопрос 44:
Периодические последовательности и их параметры. Нахождение периода ЛРП над конечным полем.}

\begin{defs}
$\Omega$ - произвольное множество.
Последовательность $u \in \Omega^{\infty}$ - периодическая, если $\exists$ параметры $\lambda \in N_0, t \in N:$
$$\forall i \geq \lambda: u(i + t) = u(i)$$

Если $u$ - пос-ть над кольцом R, то определение эквивалентно
 $$x^\lambda (x^t - e) u = (0).$$
\end{defs}

\begin{proofs}
Любая периодическая пос-ть над кольцом R есть ЛРП
\end{proofs}

\begin{defs}
 $u \in \Omega^{\infty}$ - периодическая последовательность, то наименьшее $t \in N$, для которого $\exists \lambda \in N_0$, что выполяется условие определения, назовем периодом пос-ть $u$ и обозначим $T(u)$, при этом наименьший параметр наименьший параметр $\lambda \in N_0:$

 $$\forall i \geq \lambda: u(i + T(u)) = u(i)$$  
\end{defs}


\begin{proofs}
$u$ - периодическая пос-ть с параметрами $\lambda \in N_0, t \in N \Leftrightarrow:$

$$\lambda \geq \Lambda(u), T(u) | t$$
\end{proofs}

\begin{proofs}
$u$ - периодическая пос-ть над R с параметрами $\lambda \in N_0, t \in N$, то $\forall$ многочлена $H(x) \in R[x]$ - последовательность $v = H(x)$ также периодическая, причем $\Lambda(v) \leq \Lambda(u), T(v) | T(u)$
\end{proofs}

\begin{proofs}
$u, v \in R^{\infty} - периодические пос-ти, то $w=u+v$ - периодическая пос-ть:
$$\Lambda(w) \leq max \fskobk{\Lambda(u), \Lambda(v)}, T(w) | \kskobk{T(u), T(v)}$$

При этом:\\
 (a) $\Lambda(u) \neq \Lambda(v) \Rightarrow \Lambda(w) = max \fskobk{\Lambda(u), \Lambda(v)};$ \\
 (б) $(T(u), T(v)) = 1 \Rightarrow T(w) =  \kskobk{T(u), T(v)};$ \\
 (в) $u,v$ - ЛРП, для которых возможно указать взаимно простые характеристические многочлены $\Rightarrow$ справедливы пункты (а), (б).
\end{proofs}     

\begin{defs}
Периодическая пос-ть $u$ над кольцом $R$ - чисто периодическая (реверсивная), если $\Lambda(u) = 0$ и выраждающаяся, если $u = (u(0),..,u(\lambda-1), 0,..,0,..)$ для некоторго $\lambda \in N_0$.
\end{defs} 

\begin{proofs}
Любая периодическая пос-ть u над кольцом R однозначно представляется в виде суммы 
$$u = u_0 + u_1, где$$
$u_0$ - вырождающаяся, $u_1$ - чисто периодическая.
\end{proofs}

\begin{proofs}
Унитарный мноочлен $F(x) \in R[x]$ - периодический $\Leftrightarrow$ пос-ть $e^F \in L_R(F)$
\\$F(x)$ - периодический $\Rightarrow \Lambda(F) = \Lambda(e^F),  T(F) = T(e^F)$ 
\end{proofs}

\begin{proofs}
$F(x), G(x) \in R[x]$ - унитарные периодические взаимно простые многочлены $\Rightarrow H(x) = F(x)G(x)$ - периодический многочлен, причем
$$\Lambda(H) = max \fkobk{\Lambda(F), \Lambda(G)}, T(H) = [T(F), T(G)]$$
\end{proofs}

\begin{proofs}
$u$ - ЛРП над конечным полем $\Rightarrow$
$$\Lambda(u) = \Lambda(M_u(x)), T(u) = T(M_u(x))$$
\end{proofs}

\begin{defs}
Описание параметров $\Lambda(F), T(F)$ для произвольного унитарного многочлена $F(x)$ над конечным полем сводится к построению его канонического разложения и вычислению периодов неприводимых сомножителей. Пусть $P = GF(p^n), p$ - простое. Определим для $F(x) O(F)$ - НОК порядков всех ненулевых корней многочлена в его поле разложения над P, положим $O(F) = 1,$ при $F(x) = x^l, l \in N$ 
\end{defs}

\begin{proofs}
$F(x) \in P[x]$ - реверсивный неприводимый многочлен степени $m \Rightarrow T(F) = O(F),$
$$T(F) | q^m - 1; T(F) \nmid q^k - 1, k \in [1, m-1]$$ 
при этом (T(F), p) = 1.
\end{proofs}

\begin{defs}
Способ получения периода реверсивного неприводимого многочлена F(x) степени m > 0:
\\1. Переббрать все длелители $t$ числа $q^m - 1$, (где $q = p^n$), не являющиеся делителями $q-1,.., q^{m-1} - 1$ 
\\2. Для каждого $t$ проверить условие 
$$x^t \equiv e (mod F(x))$$.

Наименьшее из них, для которого выполнится условие есть T(F). 
\end{defs}

\begin{proofs}
$P = GF(q)$ - поле характеристики p. Унитарный многочлен имеет каноническое разложение
$$F(x) = x^l G_1(x)^{k_1} .. G_s(x)^{k_s}, k = max{k_1,..,k_s} \Rightarrow$$
$$\Rightarrow \Lambda(F) = l, T(F) = [T(G_1),..,T(G_s)] \cdot p^c = O(F) \cdot p^c.$$ 
$$c \in N_0: p^{c-1} < k \leq p^c \Rightarrow c = ]log_p k[$$   
\end{proofs}

\begin{defs}
Пос-ть $u$ над $P = GF(q)$ называется ЛРП максимального периода (ЛРП МП) над P, если для некоторого $m \in N$, если для некотрого $m \in N$ пос-ть u есть ЛРП ранга m и периода $q^m - 1$
\end{defs}

\begin{proofs}
u - ЛРП над P = GF(q) с реверсивным минимальным многочленом $M_u(x) = F(x)$ степени m: $q^m > 2$. Тогда эквивалентно:
\\(a) u - ЛРП МП над Р;
\\(б) любая ненулевая ЛРП $v \in L_P(F) - есть сдвиг пос-ти u, т.е. $v = x^k u, k \in N$
\\(в) F(x) неприводим над P, его корень $\alpha$ в минимальном поле разложения $Q = GF(q^m)$ над P есть примитивный элемент поля Q, т.е. F(x) - примитивный многочлен над P
\\(г) $T(F) = q^m - 1$
\end{proofs}

\begin{defs}
u - ЛРП МП периода $q^m - 1$ над $P = GF(q)$ и $P < P' = GF(q^t)$, то t > 1 последовательность u уже не является ЛРП МП над P', посколько $T(u) < q^{tm} - 1$
\end{defs}

\begin{proofs}
Пусть u - ЛРП МП $\tau = q^m - 1$ над полем $P = GF(q)$ и $0 \leq i_1 < .. < i_r \leq m-1$, тогда   
\end{proofs}   

\begin{proofs} 
Неприводимый многочлен $F(x) \in P[x]$ степени $m \geq 1$ является многочленом МП над полем $P \Leftrightarrow F(x) \neq  x.$ и для каждого собственного простого делителя $\pi q^m - 1$ выполняется:
$$x^{\frac{q^m - 1}{\pi}} \nequiv e (mod F(x)).$$
\end{proofs}

\begin{proofs}
Если $2^m - 1$ - простоt число, то любой неприводимый над GF(2) многочлен степени m - многочлен максимального периода.
\end{proofs} 