%!TEX root = ../report.tex"
\section{Вопрос 21: Группы и их основные свойства.
Смежные классы по подгруппе, порядок элемента группы, теорме Лагранжа и её следствия.
Описание циклических групп.
Прямые произведения групп и  подгрупп, критерий разложимости в прямое произведение.}


Вспомним простейшие определения: \\
\begin{defs}[Группоид]
  Множество $G$ с одной бинарной операцией $*$ называют группоидом и обозначают через $(G;*)$.
\end{defs}

\begin{defs}[Полугруппа]
  Группоид $(G;*)$ с ассоциативной операцией называется полугруппой.
\end{defs}

\begin{defs}[Полугруппа]
  Группоид $(G;*)$ называется группой, если выполнены условия:\\
  1) операция $*$ ассоциативна;\\
  2) В $(G;*)$ существует нейтральный элемент;
  3) $\forall a \in G \exists a' \in G$ - симметричный элемент. \\
  Если, кроме того, выполняется условие коммутативности операции $*$, то группа называется коммутативной или абелевой.
\end{defs}

\begin{defs}[Подгруппа]
  Непустое подмножество $H$ группы $(G;\cdot)$ называют ее подгруппой, если $H$ замкнуто относительно
  групповой операции и является группой относительно этой операции. В этом случае пишут $H < (G;\cdot)$ или $H < G$.
  И если $H \notin {G, {e}}$, то подгруппу $H$ называют собственной.
\end{defs}

\begin{claim}
 Непустое подмножество $H$ группы $(G;\cdot)$ является ее подгруппой ТИТТК $\forall g,h \in H gh^{-1} \in H$.
 \begin{dokvo}
 Если $H < (G;\cdot)$, то утверждение следует из определения подгруппы. Обратно. Так как $H \neq \void$, то существует $g \in H$ и
 в силу условия $e = gg^{-1} \in H$. Тогда для любых $g, h \in H$ справедливы соотношения $h^{-1} = eh^{-1} \in H$ и
 $gh = g(h^{-1})^{-1} \in H$. Следовательно, подмножество $H$ замкнуто относительно групповой операции на $G$, и так как эта
 операция ассоциативна, то $H$ удовлетворяет всем условям определения 1, то есть $H < (G,\cdot)$.
 \end{dokvo}
\end{claim}

\begin{defs}[Сравнимость по подгруппе]
  Говорят, что элементы $a,b$ группы $G$ сравнимы по подгруппе $H$ справа (слева), и пишут 	$a \equiv b (H)_\text{П} (a \equiv b (H)_\text{Л})$, если
  $ab^{-1} \in H (a^{-1}b \in H)$.
  Если $G$ - абелева группа, то отношения сравнимости справа и слева совпадают и пищут просто $a \equiv b (H)$.
\end{defs}

\begin{defs}[Смежный класс по подгруппе]
  Правым (левым) смежным классом группы $(G,\cdot)$ по ее подгруппе $H$ с представителем $g \in G$ называется множество
  $Hg$ (множество $gH$).
\end{defs}

\begin{defs}[Индекс подгруппы]
  Индексом подгруппы $H$ в группе $G$ называют число правых (левых) смежных классов $G$ по $H$, если это число конечно,
  и бесконечность - в противном случае. Индес $H$ в $G$ обозначают через $|G : H|$.
\end{defs}

\begin{proofs}[Теорема Лагранжа]
	Порядок подгруппы $H$ конечной группы $G$ делит порядок $G$ и $|G| = |G : H| \cdot |H|$.
	\begin{dokvo}
    Разложение $G$ на правые смежные классы по подгруппе $H$ имеет вид $G = Hg_1 \cup \ldots \cup Hg_k$,
    где $k=|G : H|$. Отсюда $|G| = |Hg_1| + \ldots + |Hg_k|$ и тогда по теореме о равномощности смежных классов
    по подгруппе $|G| = k|H|$.
	\end{dokvo}
\end{proofs}

\begin{sledsv}
  Если $G > H > K$ - цепочка подгрупп конечной группы $G$, то $|G : K| = |G : H| \cdot |H : K|$.
  Если при этом $|G : K| = p$ - простое число, то либо $H = G$ либо $H = K$.
  \begin{dokvo}
    $|G : K| = \frac{\Modul{G}}{\Modul{K}} = \frac{\Modul{G}}{\Modul{H}}\cdot \frac{\Modul{H}}{\Modul{K}} = |G : H|\cdot|H : K|$.
  \end{dokvo}
\end{sledsv}

\begin{sledsv}
  Порядок любого элемента $g$ конечной группы $G$ желить $|G|$, в частности, $g^{|G|} = e$.
  \begin{dokvo}
    Так как $G$ - конечная, то $ord g < \infty$ и подгруппа $H = \langle g \rangle$ имеет
    порядок $|H| = ord g$. Теперь соотношение $ord g | |G|$ следует из теоремы Лагранжа.
  \end{dokvo}
\end{sledsv}

\begin{sledsv}
  Если $G$ - конечная группа, то $exp G | |G|$.
  \begin{dokvo}
    Достаточно воспользоваться фактом, что экспонента группы - это наименьшее общее кратное порядков ее элементов и
    предыдущим следствием.
  \end{dokvo}
\end{sledsv}

\begin{sledsv}
  Любая группа $G$ простого порядка $p$ - циклическая.
  \begin{dokvo}
    Пусть $g \in G \setminus \{e\}$. Тогда $ord g > 1, ord g | p$, и так как $p$ - простое,
    то $ord g = p$, и $|\langle g \rangle| = p = |G|$. Следовательно, $G = \langle g \rangle$.
  \end{dokvo}
\end{sledsv}

\begin{defs}[Циклическая группа]
  Группа $(G,\cdot)$ называется циклической, если $\exists g \in G : \forall h \in G \exists n \in Z : h = g^n$.
  (При этом $g^{k+1} = g^kg, k \geqslant 0, g^0 = e_G, g^{-k} = (g^k)^{-1}, k > 0$.
\end{defs}

\begin{proofs}[О единственности бесконечной циклической группы]
	Если $G$ - бесконечная циклическая группа, то $\exists$ изоморфизм $\varphi : Z \to G$
  (бесконечная циклическая группа единственна с точностью до изоморфизма)
	\begin{dokvo}
    $\exists g \in G : \forall h \in G h = g^m, m \in Z$
    Пусть $\exists m < k : g^m = g^k$, тогда $g^{k-m} = e_G$.
    Обозначим $d=k-m > 0, \forall h \in G h = g^t, t \in Z$, делим с остатком
    $t = qd + r, 0 \leqslant r < d$, тогда $h=(g^d)^qg^r=q^r \Rightarrow g^r \in \{e, g, g^2, \ldots, g^{d-1}\}$ -
    конечное множество. Из этого следует, что отображение $\varphi : Z \to G$, где $\varphi(m)=g^m$ - биекция.
    Легко видеть, что $\varphi$ - гомоморфизм.
	\end{dokvo}
\end{proofs}

\begin{defs}[Прямое произведение]
  Прямым (внешним) произведением групп $(G_1,\cdot),$ $\ldots, (G_t, \cdot)$ называют группоидом $(G,\cdot)$, где
  $G = G_1 \times \ldots \times G_t$ - декартово произведение множеств $G_1, \ldots, G_t$, а операция $\cdot$ на $G$ задается
  условием $\forall g = (g_1, \ldots, g_t) \in G, \forall h = (h_1,$ $\ldots, h_t) \in G : g \cdot h = (g_1 \cdot h_1, \ldots, g_t \cdot h_t)$.
  Для этого группоида используют обозначение: $G = G_1 \otimes \ldots \otimes G_t = \prod_{i=1}^{t} \otimes G_i$.
\end{defs}

\begin{defs}[Разложимость]
  Группа $(G, \cdot)$ называется разложимой, если она представляется в виде прямого произведения двух собственныхподгрупп.
  В противном случае группа $G$ называется неразложимой.
\end{defs}

\begin{proofs}[Критерий разложимости]
	Циклическая группа $(G, +)$ неразложима $\tittg$ она бесконечна ил примарна. Любая конечная
  циклическая непримарная группа однозначно, с точностью до перестановки слагаемых, раскладывается в прямую сумму примарных
  циклических подгрупп.
	\begin{dokvo}
    Если $G$ - бесконечная группа, то она изоморфна группе $(Z, +)$, которая неразложима, поскольку
    любые две ее ненулевые подгруппы $mZ$ и $nZ$ имеют нулевое пересечение: $mZ \cap nZ \ni mn, mn \neq 0$.
    Если $|G| = p^m$, то в $G$ также любые две ненулевые подгруппы $A$ и $B$ имеют ненулевое пересечение.
    Действительно, по теореме (которая здесь не доказывается) $A$ и $B$ - циклические группы, и по теореме Лагранжа
    она - $p$-группы. Следовательно (по теореме) в каждой из них есть подгруппа порядка $p: A_1 < A, B_1 < B,
    |A_1| = |B_1| = p$. Но по той же теоремев $G$ есть только одна подгруппа порядка $p$. Поэтому $A_1 = B_1 \subset A \cap B$ и
    $A \cap B \neq \{0\}$. Таким образом, примарная циклическая группа неразложима.
    Пусть, наконец, $|G| = n > 1$, и каноническое разложение числа $n$ имеет вид $n = p_1^{m_1} \cdot \ldots p_t^{m_t}$, где
    $t > 1$. Тогда $\forall i \in \overline{1,t}$ в $G$ есть единственная подгруппа $H_i$ порядка $p_i^{m_i}$ (по теореме Лагранжа),
    и подгруппа $H = H_1 + \ldots + H_{e}$ удовлетворяет условию $H = H_1 \dot{+} \ldots \dot{+} H_t$. Но тогда по утверждению
    $|H| = |H_1| \cdot \ldots |H_t| = |G|$ и $G = H_1 \dot{+} \ldots \dot{+} H_t$ - искомое разложение группы G в прямую сумму примарных циклических подгрупп.
    Единственность такого разложения с точностью до перестановки слагаемых следует из теоремы Лагранжа.

	\end{dokvo}
\end{proofs}
\newpage
