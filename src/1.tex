%!TEX root = ../report.tex"
\section{Вопрос 1: Непрерывность действительных функций одного и многих действительных переменных. Свойства непрерывных функций.}

\begin{defs}[Понятие функции]
Говорят, что на множестве $X$ имеется функция со значениями в $Y$, если в силу некторого $f$ каждому элементу $x \in X$ соответствует элемент $y \in Y$. Обозначается: $f: X \to Y$

$$f(x) := \{ y \in Y \ | \ \exists \ x ((x\in X) \wedge (y = f(x))) \} $$
\end{defs}


\begin{defs}[Предел по коши]
	Пусть $E \subset \Real$ и $f: E \to \Real$. Значение $A$ функции $f(x)$ в точке $x_0$ называется \underline{пределом}, если:
	$$\lim_{x\to x_0} f(x) = A \ \tittg \ \forall \varepsilon > 0 \ \exists \delta > 0: \forall x \ 0 < \Modul{x - x_0} < \delta \ \sledue \ \Modul{f(x) - a} < \delta$$
\end{defs}

\begin{defs}[Непрерывность в точке]
	Функция $f(x)$ называется непрерывной в точке $x_0$, если $\predel{x \to x_0}f(x) = f(x_0)$
\end{defs}

\begin{defs}[Определение непрерывности по Гейне]
	Говорят, что функция действительного переменного $f(x)$ является непрерывной в точке $a \in \Real$ если для любой последовательности $\{x_n\}$, такой что
	$$\lim_{n \to \infty}x_n = a, \text{выполняется соотношение} \lim_{n \to \infty}f(x_n) = f(a)$$
\end{defs}

\textbf{На практике удобно использовать следующие 3 условия} непрерывности функции $f(x)$ в точке $x = a$ (которые должны выполняться одновременно):
\begin{enumerate}
	\item Функция $f(x)$ определена в точке $a$
	\item Предел $\predel{x \to a}f(x)$ существует
	\item Выполняется равенство $\predel{x \to a}f(x)= f(a)$
\end{enumerate}

\begin{defs}
	Элементами пространства $\Rmern{n}$ являются упорядоченные наборы $\myvect{1}{n}$, где $x_i \in \Real$
\end{defs}

\begin{defs}
	$\Rmern{n}$ векторное пространство над $\Real$ $\sledue$ $x + y = (x_1+y_1, \ldots, x_{n}+y_{n})$, $\lambda \cdot x = (\lambda \cdot x_1, \ldots, \lambda \cdot x_n)$
\end{defs}

\begin{defs}
	$e_i = (\underbrace{0, \ldots, 1,\ldots,0}_n^i)$, где $i = \overline{1,n}$ стандартый базис $\Rmern{n}$
\end{defs}

\begin{defs}
	\textbf{Скалярным произведением} называется $x,y \in \Rmern{n}$, $\skobk{x,y} = \summa{i=1}{n}x_i \tochka y_i$, \textbf{Нормой вектора} $\DModul{x} = (x,x)^{\frac{1}{2}}$, \textbf{Расстояние между элементами $\Rmern{n}$} $\rho(x,y)= \DModul{x-y}$
\end{defs}

\begin{defs}[Открытый шар]
	Пусть $x \in \Rmern{n}$, $r > 0$.

	Обозначим $U(x,r)$ = $\fskobk{y \in \Rmern{n} \ : \ \DModul{x-y} < r}$ -- открытый шар радиуса $r$
\end{defs}

\begin{defs}
	Множество $U \in \Rmern{n}$ называется открытым, если $\forall x \in U \ \exists \ r > 0 \ : \  U(x,r) \subset U$
\end{defs}

\begin{defs}
	Окрестностью точки $x \in \Rmern{n}$ называется любое открытое подмножество, содержащее данную точку $:$ $U(x)$
\end{defs}

\begin{defs}
	Пусть $f: E \to \Rmern{m}$, $E \in \Rmern{n}$, $x_0 \in \prokol{E}$. Говорят, что $\exists$ предел $f(x)$ при $x\to x_0$ по мн-ву $E$, равный $a \in \Rmern{m}$, если $\forall \ U(a) \ \exists \ U(x_0) \ \forall x \in \prokol{U}_{E}(x_0) \sledue f(x) \in U(a)$
	и обозначается $\predel{x \to x_0}f(x)= a$
\end{defs}

\begin{claim}[]
	Если $\predel{x\to x_0}f(x)=(a_1,\ldots,a_m)=a$, то $\forall k \in \overline{1,m} $ $\exists \predel{x \to x_0}f_k(x)=a_k$

	Верно и обратное, Где $f(x)=(f_1(x),\ldots,f_m(x))$ -- координатное представление функции $f(x)$

\end{claim}

\begin{defs}
	Пусть $f: E \to \Rmern{m}$, $E \in \Rmern{n}$, $x_0 \in E$ Ф-я $f(x)$ называется непрерывной в точке $x_0$, если $\forall U(f(x_0)) \ \exists U(x_0): \ \forall x\in U_E(x_0) \sledue f(x) \in U(f(x_0))$
	\begin{itemize}
		\item $x_0$ изолированная $\sledue$ $f(x)$ всегда непрерывна в точке $x_0$
		\item $x_0$ предельная точка $E$ $\sledue$ $\skobk{f(x) \text{ непрерывна в точке } x_0} \tittg \exists \predel{x \to x_0}f(x)=f(x_0)$
	\end{itemize}
\end{defs}

\begin{claim}
	Пусть $f: E \to \Rmern{m}$, $E \in \Rmern{n}$, $x_0 \in E$, $f(x)=(f_1(x),\ldots,f_m(x))$
	Тогда \fx непрерывна в точке $x_0$ $\tittg$ $\forall \ i = \overline{1,m}$ $f_i(x) \text{ непрерывна } в x_0$
	\begin{dokvo}
	Если $x_0$ - изолированная, то все доказано. Пусть $x_0 \in E$, тогда \fx непрерывна в точке $x_0 \ \tittg \ \exists \predel{x \to x_0}f(x)=f(x_0) \ \naverh{по утв}\tittg \ \exists \predel{x \to x_0}f_i(x) = f_i(x_0) \forall i = \overline{1,m}$
	\end{dokvo}
\end{claim}

\begin{claim}
	\begin{enumerate}
		\item $f_1,f_2: E \to \Rmern{m}$, $E \in \Rmern{n}$, $f_i$ непрерывна в точке $x_0 \in E$ $\sledue$ $f_1 + f_2, \lambda\cdot f_1$ непрерывны в точке $x_0$
		\item $f_1,f_2: E \to \Rmern{m}$, $E \in \Rmern{n}$, $f_i$ непрерывна в точке $x_0 \in E$ $\sledue$ $f_1 \cdot f_2, \frac{f_1}{f_2}$, если $f_2 \neq 0$
	\end{enumerate}

	\begin{dokvo}
		Следует из свойств предела функции
	\end{dokvo}
\end{claim}

\begin{defs}
	$f: E \to \Rmern{m}, \ E \subseteq \Rmern{n}$. Ф-я \fx \ называется непрерывной на $E$, если она непрерывна в любой точке множества $E$
\end{defs}

\begin{claim}
	Пусть $f(x):E \to J, E \subseteq \Rmern{n}, J \subseteq \Rmern{m}, \ g(y):J\to \Rmern{k}$

	$\left.\begin{array}{l}
		f(x) \text{ непрерывна в точке } x_0 \in E \\
		g(y) \text{ непрерывна в точке } y_0 \in f(x_0)
	\end{array}\right\} \sledue$ $g \circ f$ непрерывна в т. $x_0$

	\begin{dokvo}
		Зафиксируем любую $U(g(y_0))$ т.к $g(y)$ непрерывна в т. $y_0$, то $\exists U(y_0)=U(f(x_0)):\forall y \in U_y(y_0) \sledue g(y) \in U(g(y_0))$

		С другой стороны \fx непрерывна в точке $x_0\in E \sledue \exists U(x_0): \forall x\in U_E(x_0) \sledue g(x)\in U(f(x_0)) \sledue g(f(x))\in U(g(y_0))$

		То есть имеем: $\forall \ U(g(f(x_0))) \ \exists  \ U(x_0):\forall x\in U_E(x_0)\sledue g(f(x))\in U(g(f(x_0)))$
	\end{dokvo}
\end{claim}

\begin{defs}
	Множество $M \in \Rmern{n}$ называется компактным $\naverh{опр}\tittg$ из любого покрытия $M$ открытыми подмножествами можно выделить конечное подпокрытие.
\end{defs}

\begin{proofs}[1-я теорема Вейерштрасса про ограниченность непрерывной функции]
	Если ф-я $f: E \to \Rmern{m}$ непрерывна на $E$, $E$ - компактное подмножество $\Rmern{n}$, то $f$ -- ограничена. $f(E)$ ограниченное подмножество $\Rmern{n}$
	\begin{dokvo}
		$\forall x \in E$ $f$-непрерывна в точке $x$ $\sledue$ $\exists \ U(x,r_x) \ r_x > 0:\forall z \in E \peres U(x,r_k)$

		$\DModul{f(z)} \subseteq M_x$, $E \subset \bigcup\limits_{x \in E}U(x,r_x)$ $E$-компактно $\sledue$ можно выбрать конечное подпокрытие, т.е разбить $E \subset \bigcup\limits_{i = 1}^{k} U(x_i,r_{x_i})$, $M = \max\limits_{1 \leq i \leq k} M_{x_i}$ $\sledue \ \forall x \in E$ $\exists \ U(x_j, r_{x_j}): x \in U(x_j,r_{x_j}) \sledue \DModul{f(x)} \subseteq M_{x_{j}} \subseteq M$
	\end{dokvo}
\end{proofs}

\begin{proofs}[2-я теорема Вейерштрасса о достижении верхней и нижней границ]
	Пусть $f:E \to \Real$, $f$-непрерывна на $E$, $E$-компактное подмножество $\Rmern{n}$ Тогда $\exists \ x_1,x_2 : f(x_1)=\max\limits_{x \in E}f(x), \ f(x_2)=\min\limits_{x \in E}f(x)$
	\begin{dokvo}
		$f$-непрерывна на $E$ $\sledue$ по 1 теореме Вейерштрасса $f$ ограничена на $E$ $\sledue$ $\exists \sup\limits_{x\in E}f(x)=M\in \Real$

		Покажем, что $\exists x_1 \in E : f(x_1)=M$ Предположим противное и рассмотрим ф-ю $g(x) = \frac{1}{M-f(x)}$, $M-f(x)$ непрерывна и не $\neq 0 \ \sledue$ $g(x)$ непрерывна на $E$ и по 1-й теореме Вейерштрасса ограничена на $E$

		С другой стороны т.к. $M=\sup\limits_{x\in E}f(x)$, то $\exists \ \fskobk{x_k}: \predel{k\to \infty}f(x_k)=M \ \sledue \predel{k\to \infty}(M-f(x))=0 \ \sledue \predel{k\to \infty}\frac{1}{M-f(x)}=\infty$, \underline{противоречие} с тем что $g(x)$ ограничена

		\textbf{Для минимума также}
	\end{dokvo}
\end{proofs}

\begin{proofs}[Кантора о равномерной непрерывности]
	Пусть $f:E\to \Rmern{m}$, $f$ - непрерывна на $E$, $E$-компактое подмножество в $\Rmern{n}$. Тогда $f$ - равномерно непрерывна на $E$
	\begin{dokvo}
		$x \in E, \ \forall \epsilon > 0, \exists \delta_x > 0: \forall \ z\in U(x, \delta_x)\peres E: \DModul{f(x)-f(z)}< \frac{\epsilon}{2}$
		$E \subset \bigcup\limits_{x\in E}U\skobk{x_i,\frac{\delta_x}{2}}$ -- покрытие открытыми подмножествами $\sledue$ можно выбрать конечное подпокрытие $E \subset \bigcup\limits_{i=1}^{k}U\skobk{x_i,\frac{\delta_{x_{i}}}{2}}$

		Пусть $\delta = \min\limits_{1 \leq i \leq k}\frac{\delta_{x_{i}}}{2}$ и рассмотрим произвольные $x^{\shtrih},x^{\shtrih\shtrih} \in E: \DModul{x^{\shtrih} - x^{\shtrih\shtrih}} < \delta$ $\exists i_0: 1 \leq i_0 \leq k \ : x^{\shtrih} \in U\skobk{x_{i_0},\frac{\delta_{x_{i_0}}}{2}} \sledue$ $\DModul{x^{\shtrih} - x_{i_0}} < \delta_{x_{i_0}}$ $\sledue$ $\DModul{f(x^{\shtrih}) - f(x_{i_0})} < \frac{\epsilon}{2}$

		C другой стороны рассмотрим $x^{\shtrih\shtrih}:\DModul{x^{\shtrih\shtrih} - x_{i_0} }$
		$\leq$ $\DModul{x^{\shtrih\shtrih} - x^{\shtrih}} + \DModul{x^{\shtrih} - x^{i_0}} < \skobk{\delta + \frac{\delta_{x_{i_0}}}{2}} \leq \delta_{x_{i_0}}$ $\sledue$ $\DModul{f(x_{i_0}) - f(x^{\shtrih\shtrih})} < \frac{\epsilon}{2}$
	\end{dokvo}

\end{proofs}

\begin{defs}[равномерно непрерывная]
	Пусть $f: E \to \Rmern{m}, \ E \subset \Rmern{n}$, тогда $f$ называется равномерно непрерывной на $E \ \tittg \ \forall \epsilon > 0 \exists \ \delta > 0 : \forall x_1,x_2 \in E \ \DModul{f(x_1)-f(x_2)} < \frac{\epsilon}{2}$
\end{defs}

\begin{defs}[связное подмножество]
	Подмножество $E \in \Rmern{n}$ называется связным, если $\forall x_1, x_2 \in E \ \exists$ непрерывное отображение $\phi: \kskobk{0,1}\to E:$ $\phi(0)=x_1,\phi(1)=x_2$
\end{defs}

\begin{proofs}[Теорема Больцано-Коши]
	Пусть $f: E \to \Real$, $f$ непрерывна на $E$, $E$ -- связное подмножество в $\Rmern{n}$, $x,y \in E : f(x) < 0 < f(y)$. Тогда $\exists c \in E : f(c) = 0$
	\begin{dokvo}
		без доказательства
	\end{dokvo}
\end{proofs}

\begin{defs}[точки разрыва]
	$f: E \to \Real, E \subseteq \Real$, точка $x_0 \in \Real$ является точкой разрыва \fx, если
	\begin{enumerate}
		\item либо $\skobk{x_0 \in E} \I $ $\skobk{f - \text{не является непрерывной в точке} \ x_0}$
		\item либо $\skobk{x \notin E} \I \skobk{x \in \prokol{E}}$

	\end{enumerate}
\end{defs}

\begin{defs}[устранимая]
	$x_0 \in E$ называется устранимой точкой разрыва если:
	$$\skobk{x_0 \ \text{точка разрыва}} \I \skobk{\exists \predel{x\to x_0 \in E}f(x) = A \in \prokol{R}}$$
\end{defs}

\begin{defs}[I рода]
	Точка разрыва $x_0$ называется точкой разрыва \textbf{I рода} если:
	\begin{enumerate}
		\item $x_0 \in \prokol{E}^{+} \sledue \exists f(x_0 + 0) \naverh{обозн}\tittg \predel{x\to x_0 + 0 \ \in E}f(x) \ \in \Real$
		 \item $x_0 \in \prokol{E}^{-} \sledue \exists f(x_0 - 0) \naverh{обозн}\tittg \predel{x\to x_0 - 0 \ \in E}f(x) \ \in \Real$
	\end{enumerate}
\end{defs}

\begin{claim}
	Всякая устранимая точка разрыва является точкой разрыва \textbf{I рода}
\end{claim}

\begin{defs}
	Точка $x_0$ разрыва \fx называется точкой разрыва \textbf{II рода} если она не является точкой разрыва \textbf{I рода}\footnote{джиниусы блин}
\end{defs}

\begin{claim}[Свойства точки разрыва]
	$f(x): E \to \Real, E \subseteq \Real$

	$\left.\begin{array}{l}
		1) f(x) \text{ монотонна на E } x_0 \in E \\
		2) \skobk{a \in E} \I \skobk{\text{а точка разрыва}}
	\end{array}\right\} \sledue$ $а$ является точкой разрыва \textbf{I рода}
\end{claim}

\begin{defs}[Еще раз по Гейне, но только в терминах]
	$x_0 \in \prokol{E}$, говорят что $\predel{k\to\infty}f(x) = A$, если для $\forall \ \fskobk{x_k}: \predel{k\to \infty}x_k = x_0, \ x_k \in E, x_k \neq x_0, \forall k \in \Natural, \ \predel{k\to\infty}f(x_k) = A$
\end{defs}

\begin{example}[Пример на точку разрыва]

\end{example}
\newpage
