%!TEX root = ../report.tex"
\section{Вопрос 60. Базовая модель ролевого управления доступом. Модель мандатного ролевого управления доступом.
Теорема об информационных потоках в модели мандатного управления доступом.}

\textbf{Основные элементы базовой модели РУД:}
\begin{enumerate*}
	\item $U$ -- множество пользователей;
	\item $R$ -- множество ролей;
	\item $P$ -- множество прав доступа к объектам КС;
	\item $S$ -- множество сессий пользователей;
	\item $PA: R \to 2^P$ -- функция, задающая для каждой роли множество прав доступа. $\forall p \in P \ \exists r \in R : p \in PA(r)$;
	\item $UA: U \to 2^R$ -- функция, задающая для каждого пользователя множество ролей, на которые он может быть авторизован;
	\item $user: S \to U$ -- функция, задающая для каждой сессии пользователя, от имени которого она активизированна;
	\item $roles: S \to 2^R$ -- функция, задающая для каждого пользователя множество ролей, на которые он авторизован в данной сессии. При этом
	в каждый момент времени для каждой сессии $s \in S$ выполняется условие $roles(s) \subseteq UA(user(s))$.
\end{enumerate*}

\begin{defs}[Иерархия]
	Иерархией ролей в базовой модели РУД называется заданное на множестве ролей R отношение частичного порядка \kav{$\leqslant$};
\end{defs}

\textbf{Ограничения:}
\begin{enumerate*}
	\item Статическое взаимное исключение подразумевает выполнения следующих условий (каждый пользователь может обладать не более чем одной ролью из каждого подмножества ролей, а каждая роль не более чем одним
	правом доступа из каждого подмножества ролей):
		\begin{itemize*}
			\item $R = R_1 \cup \ldots \cup R_n, R_i \cap R_j = \O, 1 \leqslant i < j \leqslant n, |UA(u) \cap R_i| \leqslant 1$ для $u \in U, i \in \{1, \ldots, n\}$;
			\item $P = P_1 \cup \ldots \cup P_m, P_i \cap P_j = \O, 1 \leqslant i < j \leqslant m, |PA(u) \cap P_i| \leqslant 1$ для $p \in U, i \in \{1, \ldots, m\}$;
		\end{itemize*}
	\item Динамическое взаимное исключение подразумевает выполнение следующих условий (в каждой сессии пользователь может обладать не более чем одной ролью из каждого подмножества ролей):
		\begin{itemize*}
			\item $R = R_1 \cup \ldots \cup R_n, R_i \cap R_j = \O, 1 \leqslant i < j \leqslant n$;
			\item $|roles(s) \cap R_i| \leqslant 1$ для $s \in S, i \in \{1, \ldots, n\}$;
		\end{itemize*}
	\item Статические количественные ограничения на обладание ролью или правом доступа. Определены 2 функции: $\alpha: R \to \mathbb{N}_0, \beta: P \to \mathbb{N}_0$.
	Выполняются следующие условия (для каждой роли устанавливается максимальное число пользователей, которые могут быть на нее авторизованы, а для каждого права доступа устанавливается максимальное число ролей, которые
	могут им обладать):
		\begin{itemize*}
			\item $|UA^{-1}(r)| \leqslant \alpha(r)$ для $r \in R$;
			\item $|PA^{-1}(p)| \leqslant \beta(p)$ для $p \in P$;
		\end{itemize*}
	\item Динамическое количественное ограничение на обладание ролью подразумевает наличие функции $\gamma: R \to \mathbb{N}_0$ и выполнение условия (для каждой роли устанавливается максимальное число сессий пользователей
	, которые могут одновременно быть на нее авторизованы):
		\begin{itemize*}
			\item $|roles^{-1}(r)| \leqslant \gamma(r)$ для $r \in R$;
		\end{itemize*}
	\item Статическое ограничение необходимого обладания ролью и правом доступа. Тут определены 2 функции: $\alpha: R \to 2^R, \ \beta:P \to 2^P$ и выполняются следующие условия (для каждой роли для того, чтобы на нее мог
	быть авторизован пользователь, могут быть определены роли, на которые пользователь также должен быть авторизован. Для каждого права доступа для того, чтобы им обладала роль, могут быть определены права доступа,
	которыми данная роль также должна обладать):
		\begin{itemize*}
			\item для $u \in U$ если $r, r^{\shtrih} \in R, r \in UA(u)$ и $r^{\shtrih} \in \alpha(r)$ то $r^{\shtrih} \in UA(u)$;
			\item для $r \in R$ если $p, p^{\shtrih} \in P, r \in PA(r)$ и $p^{\shtrih} \in \beta(p)$ то $p^{\shtrih} \in PA(r)$;
		\end{itemize*}
	\item Динамическое ограничение необходимого обладания ролью подразумевает наличие функции $\gamma:R \to 2^R$ и выполнение условия (для каждой роли для того, чтобы на нее мог быть авторизован пользователь
	в некоторой сессии, могут быть определены роли, на которые пользователь также должен быть авторизован в данной сессии):
		\begin{itemize*}
			\item для $s \in S$ если $r, r^{\shtrih} \in R, r \in roles(s)$ и $r^{\shtrih} \in \gamma(r)$, то $r^{\shtrih} \in roles(s)$;
		\end{itemize*}
\end{enumerate*}

\textbf{Основные элементы модели мандатного ролевого управления доступом:}
\begin{enumerate*}
	\item $O$ -- множество объектов;
	\item $(L, \leqslant)$ -- решетка уровней конфиденциальности;
	\item $c_u: U \to L$ -- функция уровней доступа пользователей;
	\item $c_o: O \to L$ -- функция уровней конфиденциальности объектов;
	\item $A = \{read, write\}$ -- виды доступа.
\end{enumerate*}

\begin{defs}[Св-во либерального/строгого мандатного управления доступом]
	Доступ $(s, (o,r)) \in S \times P$ удовлетворяет свойству либерального/строгого мандатного управления доступом, когда:
		\begin{itemize*}
			\item $r = read, c_u(user(s)) \geqslant c_o(o)$ (аналог ss-свойства)
			\item $r = write$ и если $\exists (s, (o^{\shtrih}, read)) \in S \times P$ то $c_o(o) \geqslant c_o(o^{\shtrih})  \setminus c_o(o) = c_o(o^{\shtrih})$ \textbf{<-- для второго случая}
		\end{itemize*}
\end{defs}

\textbf{Зададим:}
\begin{enumerate*}
	\item $R = \{x\_read | x \in L\} \cup \{x\_write | x \in L\}$ -- множество ролей;
	\item $P = \{(o,read) | o \in O\} \cup \{(o,write) | o \in O\}$ -- множество прав доступа;
\end{enumerate*}


\begin{defs}[Иерархия на R]
	Иерархией на R в соответствии с требованиями либерального/строго мандатного управления доступом называется отношение \kav{$\leqslant$} : для $r, r^{\shtrih} \in R$ верно $r \leqslant r^{\shtrih}$, когда выполняется
	одно из условий:
		\begin{itemize*}
			\item $r = x\_read$ и $r^{\shtrih} = x^{\shtrih}\_read$, то $x \leqslant x^{\shtrih}$;
			\item $r = x\_write$ и $r^{\shtrih} = x^{\shtrih}\_write$, то $x^{\shtrih} \leqslant x \setminus x^{\shtrih} = x$;
		\end{itemize*}
\end{defs}

\begin{defs}[Соответствие требованиям]
	Модель ролевого управления доступом соответствует требованиям либерального/строго мандатного управления доступом, когда на R задана соответствующая иерархия и выполняются ограничения:
		\begin{itemize*}
			\item на UA для $u \in U \ c(u)\_read = \bigoplus (UA(u) \cap \{x\_read | x \in L\}) \in UA(u), \{x\_write(x \in L)\} \subset UA(u)$
			\item на roles: для всех сессий $s \in S$ верно $roles(s) = \{y\_read | y \leqslant x\} \cup \{x\_write\} \setminus roles(s) = \{x\_read, x\_write\}$
			\item на PA:
				\begin{itemize*}
					\item для $x \in L$ доступ $(o, read) \in PA(x\_read) \Leftrightarrow (o,write) \in PA(x\_write)$
					\item для $(o,read) \exists!$ роль $c_o(o)\_read : (o, read) \in PA(c_o(o)\_read)$
				\end{itemize*}
		\end{itemize*}
\end{defs}

\begin{defs}[Существование информационного потока]
	Будем считать, что существует информационный поток от $o \in O$ к $o^{\shtrih} \in O$, когда $\exists s \in S, r, r^{\shtrih} \in R : (o, read) \in PA(r), (o^{\shtrih}, write) \in PA(r^{\shtrih})$
	и $r,r^{\shtrih} \in roles(s)$.
\end{defs}

\begin{proofs}[Об информационных потоках]
	Если модель ролевого управления доступом соответствует требованиям либерального или строго мандатного управления доступом, то в ней для любых объектов $o, o^{\shtrih}$ таких, что
	$c(o) > c(o^{\shtrih})$, невозможно возникновение информационного потока от $o$ к $o^{\shtrih}$.
\end{proofs}
