
\section{Вопрос 52: Способы задания графов. Алгоритмы обхода графа в ширину и вглубину. Алоритмы нахождения связных компонент ориентированного графа, обоснование коррект ности и оценки их вычислительной сложности}

                                                                                                                                                                                                                                                                                                                                                                                                                                                                                                                                                                                                                                                                                                                                                                                                                                                                                                                                                                                                                                                                                                                                                                                                                                                                                                                                                                                                                                                                                                                                                                                                                                                                                                                                                                                                                                                                                                                                                                                                                                                                                                                                                                                                                                                                                                                                                                                                                                                                                                                                                                                                                                                                                                                                                                                                                                                                                                                                                                                                                                                                                                                                                                                                                                                                                                                                                                                                                                                                                                                                                                                                                                                                                                                                                                                                                                                                                                                                                                                                                                                                                                                                                                                                                                                                                                                                                                                                                                                                                                                                                                                                                                                                                                                                                                                                                                                                                                                                                                                                                                                                                                                                                                                                                                                                           \zagolovok{}
Способы задания графов:

1) Графический способ.

Вершины изображают точками на плоскости, а ребра – линиями, соединяющими соответствующие точки. Для изображения дуги используется линия со стрелкой, указывающей направление от начала к концу дуги.

2). Спимок рёбер.

Граф задают перечислением элементов множества вершин и множества ребер.


3). Матрица смежности графа.

Это квадратная матрица, размер которой определяется числом вершин в графе. Элементы этой матрицы определяются так: 

$S_{i,j} = \levfigurn{1, \text{если пара $(v_{i}, v_{j})\in E$} \\ 0,  \text{$(v_{i}, v_{j})\notin E$}} $


4). Матрица инцидентности.

Это прямоугольная матрица, число строк которой равно числу вершин, а число столбцов – числу дуг (ребер) графа. Элементы этой матрицы определяются следующим образом:

$I_{i,j} = \levfigurn{1,  \text{ если $v_{i}$ является начальной вершиной дуги $e_{j}$} \\ -1,  \text{если $v_{i}$ является конечной вершиной дуги $e_{j}$} \\ 0,  \text{если $v_{i}$ и $e_{j}$ не инцидентны}}$

\zagolovok{Поиск в глубину}

При поиске в глубину посещается первая вершина, затем необходимо идти вдоль ребер графа, до попадания в тупик. Вершина графа является тупиком, если все смежные с ней вершины уже посещены. После попадания в тупик нужно возвращаться назад вдоль пройденного пути, пока не будет обнаружена вершина, у которой есть еще не посещенная вершина, а затем необходимо двигаться в этом новом направлении. Процесс оказывается завершенным при возвращении в начальную вершину, причем все смежные с ней вершины уже должны быть посещены.

Шаг 1. Всем вершинам графа присваивается значение не посещенная. Выбирается первая вершина и помечается как посещенная.

Шаг 2. Для последней помеченной как посещенная вершины выбирается смежная вершина, являющаяся первой помеченной как не посещенная, и ей присваивается значение посещенная. Если таких вершин нет, то берется предыдущая помеченная вершина.

Шаг 3. Повторить шаг 2 до тех пор, пока все вершины не будут помечены как посещенные.

\zagolovok{Поиск в ширину}

При поиске в ширину, после посещения первой вершины, посещаются все соседние с ней вершины. Потом посещаются все вершины, находящиеся на расстоянии двух ребер от начальной. При каждом новом шаге посещаются вершины, расстояние от которых до начальной на единицу больше предыдущего. Чтобы предотвратить повторное посещение вершин, необходимо вести список посещенных вершин. Для хранения временных данных, необходимых для работы алгоритма, используется очередь – упорядоченная последовательность элементов, в которой новые элементы добавляются в конец, а старые удаляются из начала.

Шаг 1. Всем вершинам графа присваивается значение не посещенная. Выбирается первая вершина и помечается как посещенная (и заносится в очередь).

Шаг 2. Посещается первая вершина из очереди (если она не помечена как посещенная). Все ее соседние вершины заносятся в очередь. После этого она удаляется из очереди.

Шаг 3. Повторяется шаг 2 до тех пор, пока очередь не пуста.

\zagolovok{Алоритм нахождения связных компонент ориентированного графа}

Каждой вершине приписывается вес – это вес пути от начальной вершины до данной. Также каждая вершина может быть выделена. Если вершина выделена, то путь от нее до начальной вершины кратчайший, если нет – то временный. Обходя граф, алгоритм считает для каждой вершины маршрут, и, если он оказывается кратчайшим, выделяет вершину. Весом данной вершины становится вес пути. Для всех соседей данной вершины алгоритм также рассчитывает вес, при этом ни при каких условиях не выделяя их. Алгоритм заканчивает свою работу, дойдя до конечной вершины, и весом кратчайшего пути становится вес конечной вершины.

Алгоритм Дейкстры

Шаг 1. Всем вершинам, за исключением первой, присваивается вес равный бесконечности, а первой вершине – 0.

Шаг 2. Все вершины не выделены.

Шаг 3. Первая вершина объявляется текущей.

Шаг 4. Вес всех невыделенных вершин пересчитывается по формуле: вес невыделенной вершины есть минимальное число из старого веса данной вершины, суммы веса текущей вершины и веса ребра, соединяющего текущую вершину с невыделенной.

Шаг 5. Среди невыделенных вершин ищется вершина с минимальным весом. Если таковая не найдена, то есть вес всех вершин равен бесконечности, то маршрут не существует. Следовательно, выход. Иначе, текущей становится найденная вершина. Она же выделяется.

Шаг 6. Если текущей вершиной оказывается конечная, то путь найден, и его вес есть вес конечной вершины.

Шаг 7. Переход на шаг 4.


\newpage
