%!TEX root = ../report.tex"
\section{Вопрос 48: Критерий цикличности мультипликативной группы кольца вычетов}

Хорошо известно, что аддитивная группа кольца вычетов $(\mathbb{Z}_{N};+)$ является циклической. Она порождается любым обратимым элементом кольца. Вчастности, $(\mathbb{Z}_{N};+) =\ <1>$. Рассмотрим мультипликативную группу $(\mathbb{Z}^{*}_{N},\cdot)$ этого кольца.
Пусть натуральное число N имеет каноническое разложение $N=\prod_{i=1}^N {p_{i}^{k_{i}}}$. Тогда $$|\mathbb{Z}_{N}^{*}| = \phi(N) = \prod_{i=1}^s {p_{i}^{k_{i}-1}\cdot(p_{i}-1)}$$ и $$ \mathbb{Z}_{N}^{*} \cong \mathbb{Z}_{p_{1}^{k_{1}}}^{*} \otimes \cdots \otimes \mathbb{Z}_{p_{s}^{k_{s}}}^{*}, \eqno(1)$$
где $\otimes$ - операция внешнего прямого произведения групп. Значит, для описания строения группы $\mathbb{Z}_{N}^{*}$ достаточно сделать это лишь для примарных модулей $p_{i}^{k_{i}}$. Для этого нам понадобятся некоторые утверждения о сравнениях по примарным модулям.

\begin{lemma}
	Для любого простого числа $p$, любых $k, t \in \mathbb{N}_{0}$ и $a,b \in \mathbb{Z}$ справедливо следующее: $a \equiv b (mod\ p^{k}) \Rightarrow a^{p^t} \equiv b^{p^t} (mod\ p^{k+t})$
	\begin{dokvo}
		Доказательство производится индукцией по t.
		\begin{enumerate}
			\item Для $t=0$ утверждение очевидно.
			\item Пусть верно для $t>0$. По предположению индукции $a^{p^t} = b^{p^t}+p^{k+t}c, c \in \mathbb{Z}$. Тогда по формуле бинома Ньютона
			$a^{p^{t+1}}= (b^{p^t}+p^{k+t}c)^p = b^{p^{t+1}}+\binom{p}{1} b^{p^t (p-1)}p^{k+t}c + \binom{p}{2} b^{p^t (p-2)}p^{2(k+t)}c^2 +...$
			Отсюда, учитывая, что $p | \binom{p}{i}$ при $1 \leq i \leq p-1$, получим $a^{p^{t+1}} = b^{p^{t+1}} + p^{k+t+1}(b^{p^{t}(p-1)}c+\frac{p-1}{2}b^{p^{t}(p-2)}p^{k+t-1}c^2+\dots) = b^{p^{t+1}} + p^{k+t+1}c'$
		\end{enumerate}
		Лемма доказана.
	\end{dokvo}
\end{lemma}

Аналогично, индукцией по t с использованием формулы бинома Ньютона доказываются следующие две леммы.

\begin{lemma}
	Если p - нечетное простое число, $a \in \mathbb{Z}$ и $a = 1 +pc_{0}$, где $(p,c_{0}) = 1$, то $\forall t \in \mathbb{N}_{0}$ имеет место равенство
	$a^{p^{t}} = 1+p^{t+1}c_{t}$, где $(p,c_{t})=1$.
\end{lemma}

\begin{lemma}
	Если $a \in \mathbb{Z}$ и $a = 1 + 2^{2}c_{0}$, где $(2,c_{0}) = 1$, то $\forall t \in \mathbb{N}_{0}$ имеет место равенство
	$a^{2^{t}} = 1+2^{t+2}c_{t}$, где $(2,c_{t})=1$.
\end{lemma}

\begin{proofs}[О строении группы $\mathbb{Z}_{p^{k}}^{*}$ при нечетном простом p]
	$\forall$ нечетных простых p и $\forall$ $k \in \mathbb{N}$ группа $\mathbb{Z}_{p^{k}}^{*}$ является циклической.
	\begin{dokvo}
		Во-первых, заметим, что для $k=1$ группа $\mathbb{Z}_{p^{k}}^{*}$ является мультипликативной группой конечного поля из p элементов, и потому циклическая(решали когда-то на ПЗ такую задачу). При k>1 рассмотрим циклические подгруппы $\mathbb{Z}_{p^{k}}^{*}$:
		\begin{enumerate}
			\item $A=<a>$, где $a = 1 + pc, (p,c) = 1$
			\item $B=<b^{p^{k-1}}>$, где b выбрано так, что число $b_{1} = r_{p}(b)$ является образующим элементом группы $\mathbb{Z}_{p^{k}}^{*}$. Очевидно, что $(p,b)=1$.
		\end{enumerate}
		Непосредственно из леммы 1 получаем $a^{p^{k-1}} \equiv 1 (mod\ p^{k})$, $a^{p^{k-2}} \not\equiv 1 (mod\ p^{k})$. Значит, порядок элемента a в группе $\mathbb{Z}_{p^{k}}^{*}$ равен $p^{k-1}$ и, следовательно, $|A| = p^{k-1}$.
		Так как по теореме Эйлера Ферма выполняется сравнение $b^{\phi(p^{k})} \equiv 1 (mod\ p^{k})$, то $(b^{p^{k-1}})^{p-1} \equiv 1 (mod\ p^{k})$.

		Допустим, что $(b^{p^{k-1}})^t \equiv 1(mod\ p^{k})$ при $0 < t < p-1$. Тогда и $(b^{p^{k-1}})^t \equiv 1(mod\ p)$. По малой теореме Ферма $b^{p-1} \equiv 1 (mod\ p)$. Отсюда легко следует сравнение $b^{p} \equiv b (mod\ p)$, а потому и $b^{p^{k-1}} \equiv b (mod\ p)$. Значит, $b^t \equiv b^{t}_{1} \equiv 1 (mod\ p)$, что противоречит выбору b. В итоге мы доказали, что порядок элемента $b^{p^{k-1}}$ в группе $\mathbb{Z}_{p^{k}}^{*}$ равен $p-1$ и, следовательно, $|B| = p-1$.

		Заметим, что $(|A|,|B|) = 1$, и потому $|A \bigcap B|=1$. Значит, AB - прямое произведение подгрупп. При этом $|AB|=|A||B|=p^{k-1}(p-1)=|\mathbb{Z}_{p^{k}}^{*}|$, поэтому $AB = \mathbb{Z}_{p^{k}}^{*}$.
		Воспользуемся теперь тем фактом, что прямое произведение групп G, H является циклической группой $\Leftrightarrow$ G, H - циклические и их порядки взаимно просты.
		Поскольку порядки подгрупп A и B взаимно просты, то группа $AB = \mathbb{Z}_{p^{k}}^{*}$ - циклическая.
	\end{dokvo}
\end{proofs}

\begin{proofs}[О цикличности группы $\mathbb{Z}_{2^{k}}^{*}$]
	При $k \in {1,2}$ группа $\mathbb{Z}_{p^{k}}^{*}$ является циклической. При $k > 2$ группа $\mathbb{Z}_{p^{k}}^{*}$ не является циклической и раскладывается в произведение двух циклических подгрупп порядков 2 и $2^{k-2}$.
	\begin{dokvo}
		При $k \in {1,2}$ утверждение теоремы очевидно.
		Пусть $k > 2$. Обозначим $A = {1, 2^{k}-1}, B = {b \in \mathbb{Z}_{2^{k}}^{*} | b \equiv 1 (mod\ 2^{2})}$. Нетрудно заметить, что A и B - подгруппы в $\mathbb{Z}_{2^{k}}^{*}$. При этом A - циклическая подгруппа порядка 2.

		Представим теперь элементы множества B в двоичной системе счисления $b = \sum_{i=0}^{k-1}b_{i}2^{i}$. Т.к. $b \equiv 1(mod\ 2^2)$, то $b_0 = 1, b_1 = 0$. Значит, все элементы множества B представляются в виде $b = 1 + 2^{2}c, 0 \leq c < 2^{k-2}$. Отсюда, в частности, следует, что $|B| = 2^{k-2}$ и $2^{k}-1 = \sum_{i=0}^{k-1}2^{i} \notin B$.
		Итак, $|A \bigcap B| = 1$, AB - прямое произведение подгрупп, $|AB|=|A||B|=2 \cdot 2^{k-2} = |\mathbb{Z}_{2^{k}}^{*}|$ и $AB = \mathbb{Z}_{2^{k}}^{*}$.

		Докажем, что подгруппа B циклическая. Для этого выберем $b = 1 + 2^{2}c \in B, (2,c)=1$. Непосредственно из леммы 2 получаем соотношения $b^{2^{k-2}} \equiv 1 (mod\ 2^{k}), b^{2^{k-3}} \not\equiv 1(mod\ 2^{k})$.
		Значит, порядок элемента b в группе $\mathbb{Z}_{2^{k}}^{*}$ равен $2^{k-2}$ и, следовательно, $B=<b>$.

		Осталось заметить, что прямое произведение $AB = \mathbb{Z}_{2^{k}}^{*}$ не является циклической группой, поскольку порядки A и B не взаимно просты.
	\end{dokvo}
\end{proofs}

На основании этих двух теорем можно сформулировать следующий критерий.

\begin{proofs}[Критерий цикличности группы $\mathbb{Z}_{N}^{*}$]
	Группа $\mathbb{Z}_{N}^{*}$ является циклической $\Leftrightarrow N \in  \{2,4,p^{k}, 2p^{k}|p$ - нечетное простое число$\}$.
	\begin{dokvo}
		Рассмотрим каноническое представление числа $N = \prod_{i=1}^{s}p_{i}^{k_{i}}$ и разложение (1). Если $s = 1$, то утверждение теоремы следует из предыдущих теорем.
		Пусть теперь $s>1$. Если при этом среди чисел $p_{1},\dots,p_{s}$ существуют два нечетных числа $p_{i}, p_{j}$, то порядки групп $\mathbb{Z}_{p^{k_{i}}_{i}}^{*}$ и $\mathbb{Z}_{p^{k_{j}}_{j}}^{*}$ не взаимно просты (т.к. они чётны). Значит, в этом случае группа $\mathbb{Z}_{N}^{*}$ не циклическая.
		Осталось рассмотреть случай $s=2, p_{1}=2, p_{2}$ - нечетное простое число. Если $k_{1}>2$, то по предыдущей теореме группа $\mathbb{Z}_{2^{k_{1}}}^{*}$ (а, следовательно, и $\mathbb{Z}_{N}^{*}$) не циклическая.
		Если $k_{1}=2$, то группа $\mathbb{Z}_{N}^{*} \cong \mathbb{Z}_{2^{2}}^{*} \otimes \mathbb{Z}_{p_{2}^{k_{2}}}^{*}$ не является циклической, т.к. порядки групп $\mathbb{Z}_{2^{2}}^{*}$ и $\mathbb{Z}_{p_{2}^{k_{2}}}^{*}$ не взаимно просты. Если же $k_{1} = 1$, то порядок группы $\mathbb{Z}_{2}^{*}$ равен единице, и группа $\mathbb{Z}_{N}^{*}$ циклическая.
	\end{dokvo}
\end{proofs}
\newpage
