%!TEX root = ../report.tex"
\section{Вопрос 42: Неприводимые многочлены над конечными полями, существование неприводимых многочленов данной степени. Построение поля с заданным числом элементов. Описание минимального поля разложения неприводимого многочлена над конечным полем.
}

\begin{proofs}[Теорема о существовании неприводимого многочлена заданной степени]
	Пусть $P$ - конечное поле, тогда $\forall\; m \in \mathbb{N}$ над $P\; \exists\; f(x) \in P$ - неприводимый многочлен степени $degf(x) = m$
  \begin{dokvo}
    Пусть $P = GF(q) \Rightarrow$ по теореме о существовании полей с примарным числом элементов $\exists\; P' = GF(q^m)$.
    Тогда $P'$ содержит подполе, изоморфное $P$. Не ограничивая общности будем считать, что $P < P'$.
    Положим  $a$ - примитивный элемент $P'$, пусть тогда $P' = P(a)$ (типа поле $P'$ было, а теперь мы определились как оно выглядит). Тк $[P':P] = m < \infty \Rightarrow a$ -  алгебраичен над $P$
    $\Rightarrow\; \exists$ неприводимый многочлен над $P$ , аннулирующий $a$. Положим $f(x)$ - минимальный многочлен элемента $a$ над $P$.
    Тогда $f(x)$ - неприводим над $P,\; degf(x) = [P':P] = m \Rightarrow f(x)$ - требуемый неприводимый многочлен. 
	\end{dokvo}
\end{proofs}

\begin{sledsv}[]
  $\forall\; m \in \mathbb{N}\; GF(q) \cong GF(q)[x]_{/f(x)} \cong GF(q)(a), где f(x)$ - неприводимый степени $m$, $a$ - корень $f(x)$ в поле разложения.
\end{sledsv}

\begin{proofs}[Теорема о корнях неприводимого многочлена]
    Пусть $f(x)$ - неприводимый степени $m$ над $P = GF(q)$ и $P' = P(a)$ - расширение корнем $a$ многочлена $f(x)$, тогда верно:
    \begin{enumerate}
        \item $P'$ - МПР $f(x)$ над $P$ и в $P'$ многочлен $f(x)$ имеет в точности m корней вида $\{a, a^q, a^{q^2}, ..., a^{q^{m-1}}\}$
        таким образом $f(x) = (x - a)(x - a^{q})(x - a^{q})(x - a^{q^{m-1}})$
        \item $f(x) | x^{q^m} - x$ над $P$
    \end{enumerate}

  \begin{dokvo}
    \begin{enumerate}
        \item Пусть $f(x) = f_mx^m + ... + f_1x + f_0$, где $f_i \in P = GF(q)$. Заметим, что $\forall\; i \in \overline{0,m}$
        ${f_i}^q = f_i$. Тогда, если $a$ - корень $f(x) \Rightarrow f(a^q) = f_m\cdot(a^q)^m + ... + f_1\cdot a^q + f_0
        = {f_m}^q\cdot(a^m)^q + ... + {f_1}^q\cdot a^q + {f_0}^q = (f_m a^m + ... + f_0)^q = (f(a))^q = 0 \Rightarrow a$ - корень $f(x)$
        $\Rightarrow a^i$ - также корни $f(x)\; \forall\; i \in \overline{0,m-1}$. Покажем что корни различны.
        Пусть для некоторых $0 \leq k < s \leq m-1\; a^{q^k} = a^{q^s} \Leftrightarrow a^{q^s} - a^{q^k} = 0
        \Leftrightarrow (a^{q^{s-k}} - a)^{q^k} = 0 \Rightarrow a^{s-k} - a = 0 \Rightarrow a$ - корень многочлена
        $x^{q^{s-k} - x}$. Рассмотрим поле $P' = P(a)$. Тк $degm_{a,P} = degf(x) = m \Rightarrow |P'| = q^m$. Тогда рассмотрим любой многочлен
        $b \in P'[x], b = c_0 + ... + c_{m-1}a^{m-1}, c_i \in P$. Возведем $b^{q^{s-k}} = (c_0 + ... + c_{m-1}a^{m-1})^{q^{s-k}} = {c_0}^{q^{s-k}} + ... + c_{m-1}^{q^{s-k}} a^{(m-1)q^{s-k}} = c_0 + ... + c_{m-1}{a^{m-1}} = b \Rightarrow
        \; \forall\; b \in P'[x]$ - является корнем $x^{q^{s-k}} - x \Rightarrow$ любой элемент из $q^m$ элементов $P'$ - является корнем $x^{q^{s-k}} - x$,
        но $q^{s-k} < q^m$ - противоречие. Значит различны.
        \item $[P':P] = degf(x) = m \Rightarrow |P'| = q^m$. Все элементы $P$ - корни $x^{q^m} - x, f(x) | x^{q^m} - x$ в $P'$.
        Покажем, что делит и в $P$. \zagolovok{От противного:} Пусть $x^{q^m} - x = f(x)\cdot b(x) + r(x)$ над $P$, где 
        $degr(x) < m$. В силу того, что $P < P'$ верно и над $P'$, но над $P'$ $r(x) = 0 \Rightarrow r(x) = 0$ над $P$.
    \end{enumerate}
    
	\end{dokvo}
\end{proofs}

\begin{sledsv}[Полезно знать]
  Пусть $f(x) \neq x$ - неприводимый многочлен над $P$, $a,b$ - корни $f(x)$ в поле разложения $P'$. Тогда для гуппы $(P*,\cdot)$ верно:
  \begin{enumerate}
        \item $ord(a) = ord(b)$
        \item $ord(a) | q^m - 1, где m = degf(x)$
        \item $ord(a) \nmid  q^r - 1$, где $к \in \overline{0,m-1}$
    \end{enumerate}
\end{sledsv}

\begin{sledsv}[Полезно знать 2]
  Неприводимый многочлен над полем разложения взаимнопрост со своей производной.
\end{sledsv}

\zagolovok{Построение поля с заданным числом элементов}

Пусть $q$ - заданное число элеменотов. Построим поле $P = GF(q) = GF(p^m),\; p = Char(P),\; P \cong \mathbb{Z}_p[x]_{/f(x)}$, где $f(x)$ - неприводимый многочлен степени $m$ над $\mathbb{Z}_p,\;
P = \mathbb{Z}_p(a)$, где $a$ - корень многочлена $f(x)$. Сам способ: $P_0 = \mathbb{Z}_p$, элементы $\{e, a, ... , a^{m-1}\}$ - являются ЛНЗ
над $P_0$ и представляют базис $P_{P_0}$. Тогда любой элемент $b \in P$ представлен многочленом $b(a) = (b_0 + b_1a... + b_{m-1}a^{m-1}),\; b_i \in P_0$.
Операции задаются так, чтобы не выходить за рамки поля: $\forall\; b(a),c(a) \in P$: 
    \begin{enumerate}
        \item $b(a) + c(a) = ((b_0+c_0) + (b_1+c_1)a... + (b_{m-1}+c_{m-1})a^{m-1})$
        \item $b(a) \cdot c(a) = [b(a) \cdot c(a)]_{f(x)}$
    \end{enumerate}

\begin{example}[Пример построения поля с заданным числом элементов]
    Пусть $q = 9 = 3^2$. Тогда возьмем $P_0 = \mathbb{Z}_3$. Неприводимый многочлен степени 2 - $f(x) = x^2 + 1 \Rightarrow P = GF(3^2) = P_0(a)$, где $а$ - корень $f(x)$ в поле МПР. 
    Таким образом элементы $P$ представляют собой ЛНЗ многочлены вида $b(a) = b_{m-1}a^{m-1} + ... + b_0 = b_1a + b_0,\; b_i \in P_0$. То есть $P  = \{0, 1, 2, a, a+1, a+2, 2a, 2a+1, 2a+2\}$.
\end{example}

\newpage
