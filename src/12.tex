%!TEX root = ../report.tex"
\section{Вопрос 12}

\subsection{Системы линейных уравнений над полем.
Алгоритм Гаусса.
Критерий совместности системы линейных уравнений (теорема Кронекера-Капелли).
Теорема Крамера.
Фундаментальная система реений системы линейных однородных уравнений.
Общее решение системы линейных уравнений.
}

\begin{defs}[СЛУ над полем]
  Системой линейных уравнений над полем $P$ называется система вида
  \begin{equation*}
    \begin{cases}
      u_{11}x_1 + u_{12}x_2 + \ldots + u_{1n}x_n = v_{11}x_1 + \ldots + v_{1n}x_n + b_1 \\
      u_{21}x_1 + u_{22}x_2 + \ldots + u_{2n}x_n = v_{21}x_1 + \ldots + v_{2n}x_n + b_2 \\
      \ldots \\
      u_{m1}x_1 + u_{m2}x_2 + \ldots + u_{mn}x_n = v_{m1}x_1 + \ldots + v_{mn}x_n + b_m \\
    \end{cases}
  \end{equation*}
  $m,n \geqslant 1, u_{ij}, v_{ij} \in P$ - известные; \\
  $x_1, \ldots, x_n$ - неизвестные.
\end{defs}

\begin{defs}[Решение системы]
  Последовательность $(a_1, \ldots, a_n) \in P^n$ называется решением системы, если
  при подстановке $a_1, \ldots, a_n$ вместо $x_1, \ldots, x_n$ получается верное равенство.
\end{defs}

\begin{defs}[Некоторые определения]
\underline{Решить систему} означает найти множество \underline{всех} решений. \\
Система называется \underline{совместной}, если множество решений непустое. \\
Система называется \underline{несовместной}, если множество решений пустое. \\
Система называется \underline{определенной}, если она имеет ровно одно решение. \\
Система называется \underline{неопределенной}, если она совместна и имеет более одного решения. \\
Две СЛУ называются \underline{равносильными}, если множества их решений совпадают. \\
\end{defs}

Основные задачи: \\
  Определить, является ли система совместной; \\
  Найти хотя бы одно решение; \\
  Найти множество всех решений системы. \\

Введем матричную форму записи системы: \\
$U = (u_{ij})_{i \in \overline{1, m}, j \in \overline{1, n}}$ \\
$М = (v_{ij})_{i \in \overline{1, m}, j \in \overline{1, n}}$ \\
$b^\downarrow = \begin{pmatrix} b_1 \\ \vdots \\ b_m \end{pmatrix}$ \\
$U, V \in P_{m,n}, b^\downarrow \in P^{(m)}$ \\
$x^\downarrow = \begin{pmatrix} x_1 \\ \vdots \\ x_m \end{pmatrix}$ - столбец неизвестных. \\

Вектор $a^\downarrow = \begin{pmatrix}a_1 \\ \vdots \\ a_n \end{pmatrix}$ называется
решением системы, если верно равенство $Ua^\downarrow = Va^\downarrow + b^\downarrow$

\begin{defs}[Система в стандартной форме]
  Система называется системой в стандартной форме, если $V = (0)_{mxn}$, то есть $Ux^\downarrow = b^\downarrow$.
\end{defs}

Ступенчатой $m$x$(n+1)$ матрицей ранга $r \in \overline{1, m}$ называется
матрица $S_{mx(n+1)}$, для которой верно: \\
(a) Последние $m-r$ ее строк - нулевые; \\
(б) Любая из $r$ первых строк - ненулевая; \\
(в) Если $1 \leqslant r \leqslant, \overrightarrow{S_k} = (0, \ldots, 0, s_{kj_k}, *, \ldots, *)$, где$s_{kj_k} \neq 0$\\
$\begin{pmatrix}
    0 & \ldots & 0 & e & * & \ldots & * & * & * \\
    0 & \ldots & 0 & 0 & 0 & \ldots & 0 & e & *
\end{pmatrix}$
\newpage
