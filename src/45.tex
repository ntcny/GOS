%!TEX root = ../report.tex"
\section{Вопрос 45: Замкнутые классы функций. Критерий полноты для булевых функций и функций многозначной логики.}

\begin{defs}[Замкнутые классы]
Класс булевых функций $K$ называется замкнутым, если он совпадает со своим замыканием , т.е. $K=[K]$.
\end{defs}

\zagolovok{Замечание}
 Пост доказал, что множество всех замкнутых классов булевых функций счётно и в каждом замкнутом классе $K$ можно выделить конечную подсистему $K^{'}$, порождающую $К$, т.е. имеющую своим замыкание класс $K$.

\zagolovok {Пример 1.} Класс $T_{0}$ всех функций, сохраняющих константу $0$, т.е. удовлетворяющих условию: $f(0,\ldots,0)=0$.

\zagolovok {Пример 2.} Класс $T_{1}$ всех функций, сохраняющих константу $1$, т.е. удовлетворяющих условию: $f(1,\ldots,1)=1$.

\zagolovok {Пример 3.} Класс $L$ всех линейных функций, т.е. функций представимых в  виде: $a_{1} x_{1} \oplus \ldots \oplus a_{n} x_{n}, a_{i} \in \Omega  $.

\zagolovok {Пример 4.} Класс $L_{1}$ всех афинных функций, т.е. функций представимых в  виде: $a_{1} x_{1} \oplus \ldots \oplus a_{n} x_{n} \oplus a_{n+1}, a_{i} \in \Omega  $.

\zagolovok {Пример 5.} Класс $S$ самодвойственных функций, т.е. функций удовлетворяющих условию: $f(a_{1}, \ldots, a_{n}) = \overline{f}(\overline{a_{1}}, \ldots, \overline{a_{n}})$ при любых $a_{1}, \ldots, a_{n} \in \Omega$.

\zagolovok {Пример 6.} Класс $M$ монотонных функций.
Для того чтобыопределить класс $M$, , введем на множестве $\Omega_{n}$ отношение частичного порядка $\leq$ следующим образом: для наборов $\alpha = (a_{1}, \ldots, a_{n})$ и $\beta= (b_{1}, \ldots, b_{n})$ : $\alpha \leq \beta \tittg \forall i \in \overline{1,n}   a_{i} \leq b_{i}.$      Здесь мы подразумеваем, что в $\Omega$ выполняется неравенство $0 \leq 1$.



\begin{proofs}[Э. Пост]
Система булевых функций K полна тогда и только тогда, когда она содержит хотя бы по одной функции каждого из классов $F_{2} \mathbin{/} T_{0}, F_{2} \mathbin{/} T_{1}, F_{2} \mathbin{/} L, F_{2} \mathbin{/} S, F_{2} \mathbin{/} M$.
	\begin{dokvo}
		Пусть G - произвольный замкнутый класс, не совпадающий с $F_{2}$ и $K \cap (F_{2}\mathbin{/} G) = \oslash$. Тогда $K \subseteq G$, и поэтому $[K] \subseteq [G] = G \neq F_{2}$.
		Следовательно, система K не полна. Этим теорема доказана в одну сторону, поскольку $T_{0}, T_{1}, L, S, M$ — замкнутые классы, отличные от $F_{2}$.

		Обратно, пусть K не лежит полностью ни в одном из классов $T_{0}, T_{1}, L, S, M$. Покажем, что тогда замыкание системы K содержит функции $x_{1} * x_{2}$ и $\overline{x}$. Тогда полнота системы K будет следовать из полноты $ K_{1} = { x_{1}x_{2}, \overline{x_{1}}};$. Пусть $f_{1}, f_{2}, f_{3}, f_{4}, f_{5}$ — функции (не обязательно различные) из системы K, причем $ f_{1} \nsubseteq T_{0}, f_{2} \nsubseteq T_{2}, f_{3} \nsubseteq L, f_{4} \nsubseteq S, f_{5} \nsubseteq M$.

		Рассмотрим два случая в зависимости от значения $f_{1} (1,\ldots , 1)$. Покажем, что в обоих случаях система $[K]$ содержит функции $0, 1, \overline{x}$.

		1. $f_{1} (1,\ldots , 1) = 1$. Так как $f_{1} (0,\ldots , 0) = 1$ то в этом случае формула $f_{1} (x,\ldots , x)$ представляет константу 1, т. е. $ 1 \in [K]$. Тогда и константа $0 \equiv f_{2} (1,\ldots , 1)$ также лежит в $[K]$.
		Поскольку функция $f_{5}$ - не монотонна (Булева функция не является монотонной тогда и только тогда, когда существуют соседние наборы $\alpha$ и $\beta$ , т.что $ \alpha \leq \beta$ и $ f(\alpha)$ \textgreater $f(\beta)$) найдутся такие соседние наборы $\alpha$ и $\beta$, что $\alpha \leq \beta$ и $f_{5}(\alpha) = 1, f_5(\beta) = 0$. Пусть $f_{5}$ - функция от $n$ переменных и $ \alpha = (a_{1}, \ldots , a_{k-1}, 0,a_{k+1},  \ldots , a_{n}),  \beta = (a_{1}, \ldots , a_{k-1}, 1, a_{k+1} \ldots , a_{n})$.

		Тогда, очевидно, формула $f_{5}(a_{1}, \ldots, a_{k-1}, x, a_{k+1}, \ldots, a_{n})$  представляет функцию $\overline{x}$.  А так как $x, 0 ,1$ и $ f_{5}$ содержатся в $[K]$, то и $\overline{x} \in [K]$.


		2.  $f_{1} (1,\ldots , 1) = 0$. Так как  $f_{1} (0,\ldots , 0) = 1$, то в этом случае формула $f_{1} (x,\ldots , x)$ представляет функцию  $\overline {x}$. Таким образом, $\overline{x} \in [K]$.

		Так как функция $f_{4}$ не является самодвойственной, то найдется набор $\alpha = (a_{1}, \ldots , a_{n}) : f_{4}(a_{1}, \ldots, a_{n}) = f_{4}(\overline{a_{1}}, \ldots,\overline{ a_{n}})=C$, $ C \in \Omega$. Рассмотрим функцию $f_{4}(x^{a_{1}}, \ldots, x^{a_{n}})$. При $x =0$ она принимает значение $f_{4}(0^{a_{1}}, \ldots, 0^{a_{n}}) = f_{4}(\overline{a_{1}}, \ldots, \overline{a_{n}}) = C$, следовательно, эта функция — константа C. Значит, $C \in [K]$. Константа  $\overline{C}$  также принадлежит [K], так как уже показано, что в $[K]$ содержится $\overline{x}$.

		Таким образом, показано, что в любом случае $ {0, 1, \overline{x}} \subset [K]$.  Остается показать, что  $x_{1}*x_{2} \in [K]$.

		Рассмотрим функцию $f_{3}$ . Так как она не лежит в L, то она представляется многочленом Жегалкина степени выше 1, т.е. в этот многочлен входит член, содержащий произведение по крайней мере двух переменных. Пусть это переменные $x_{1}$ и $x_{2}$. Тогда $f_{3}$ можно представить следующим образом:

		$f_{3}(x_{1}, \ldots, x_{n}) = x_{1}x_{2}\varphi_{1}(x_{3}, \ldots, x_{n})\oplus x_{1}\varphi_{2}(x_{3}, \ldots, x_{n})\oplus x_{2}\varphi_{3}(x_{3}, \ldots, x_{n})\oplus \varphi_{3}(x_{3}, \ldots, x_{n})$,

		причем функция $\varphi_{1}$  отлична от константы 0, т.е. существует набор $(a_{3}, \ldots, a_{n})$, такой что $\varphi_{1} (a_{3}, \ldots, a_{n} = 1)$. Функция $ g(x_{1},x_{2})$,  представимая формулой

		$ g(x_{1},x_{2}) \equiv f_{3}(x_{1}, x_{2}, a_{3}, \ldots, a_{n}) \equiv x_{1}x_{2} \oplus C_{1}x_{1} \oplus C_{2}x_{2} \oplus C_{3}, C_{i} \in \Omega, i \in \overline{1,3}$,

		принадлежит $[K]$, так как ранее доказано, что в этом классе лежат константы $0$ и $1$. Тогда

		$ g(x^{\overline{C_{2}}},x^{\overline{C_{1}}}) \equiv (x_{1} \oplus C_{2})(x_{2} \oplus C_{1}) \oplus C_{1}(x_{1} \oplus C_{2}) \oplus C_{2}(x_{2} \oplus C_{1}) \oplus C_{3} \equiv x_{1}x_{2} \oplus C_{3} \oplus C_{1}C_{2}.$

		Таким образом, классу $[K]$ принадлежит либо функция $ x_{1}x_{2}$, либо $\overline{x_{1}x_{2}}$ , а поскольку уже известно, что в этом классе содержится «отрицание», то в любом случае $ x_{1}x_{2} \in [K]$.

	\end{dokvo}
\end{proofs}

\begin{defs}[Функция k-значной логики ]
Функцией k-значной логики от n переменных (n-местной функцией) называется любое отображение $f : \Omega_{k}^{n} \rightarrow  \Omega_{k}, n \in N $. При $n =0$ функциями k-значной логики называют константы $0,1,\ldots, k-1 $.
\end{defs}

\begin{proofs}[А. А. Нечаев]
Система функций $k$-значной логики $K$ полна тогда и только тогда, когда одновременно выполняются условия:

$1)$  Cистема $K$ не сохраняет бинарного отношения сравнимости по каждому из собственных делителей числа $k$;

$2)$ Замыкание системы $K$ содержит функции $x_{1} + x_{2}, x_{1}  * x_{2}$ и $1$.
	\begin{dokvo}
		 Необходимость условия 2 для полноты системыK очевидна по определению полной системы. Предположим, условие 1 не выполнено, т.е. каждая функция из K сохраняет отношение сравнимости по некоторому собственному делителю числа k. Индукцией по рангу формулы нетрудно доказать, что таким свойством будет обладать любая функция из замыкания K. Однако, как было показано в утверждении 1.12, в Fk всегда найдется функция, не сохраняющая этого отношения. Таким образом, условие 1 также выполняется.

		Для доказательства обратного утверждения нам понадобится следующая лемма ( Если система K удовлетворяет условиям 1 и 2, то для любых различных a и b из $\Omega_{k}  (рассматриваемого как кольцо Z_{k})$  множество $M_{a,b}$ определяемое равенством $ M_{a,b} = { f(a) - f(b) | f(x_{1} \in [K])}$, совпадает со всем $\Omega_{k}$.

		Из леммы следует, что для любых различных a и b в [K] найдется функция одного переменного $\varphi$, такая что $\varphi(a) - \varphi (b) = 1$.  Рассмотрим функцию $h_{a,b}(x) = \varphi(x) - \varphi(b)$. Очевидно, $h_{a,b}(a) = 1, h_{a,b}(b) = 0 и h_{a,b}(x) \in [K]$.  Но тогда в $[K]$ лежит функция $\delta_{a}(x)$,   поскольку $\delta_{a}(x) \equiv \sqcap_{b \in \Omega_{k} \backslash {a}} h_{a,b}(x)$. Таким образом, [K] содержит в себе систему $({0, 1, \ldots , k-1, \delta_{0}(x), \ldots , \delta_{k-1}(x), x_{1}*x_{2}, x_{1}+x_{2}})$, следовательно [K] - полная система.
	\end{dokvo}
\end{proofs}
