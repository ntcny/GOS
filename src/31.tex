%!TEX root = ../report.tex"
\section{Вопрос 31: Конечные однородные марковские цепи.
Пример: выяснить, является ли цепью Маркова последовательность положений частицы, двигающейся по
вершинам правильного N-угольника так, что в начальный момент времени частица получает толчок по или против часовой стрелки,
и затем постоянно движется в направлении толчка.
Переходные вероятности за один и за n шагов. Уравнение Колмогорова-Чепмэна и их следствие: $\text{П}^{( \,n) \,} \ = \ \text{П}^{n}$
}

Сначала вспомним некоторые определения, которые не относятся непосредственно к данному вопросу, но их знание подразумевается.

\begin{defs}[Случайный процесс]
  Пусть $( \,\Omega, \mathfrak{A}, P( \,.) \,) \,$ - некоторое ВП. $T \subseteq R$ - некоторое числовое множество
  и пусть $\forall t \in T$ определена СВ $\xi \ : \ \Omega \to R$ (t - номера СВ).
  Тогда совокупность ($\xi_t, t \in T$) этих СВ, пронумерованных (проиндексированных числами) $t \in T$ называется
  \textit{случайным процессом}, определенным на множестве индексов $T$.
\end{defs}

\begin{defs}[Сечение]
  Пусть ($\xi_t, t \in T \subseteq R$) - СП, то индексы $t$ называются \textit{моментами времени}, а сама СВ $\xi_t$ называется
  \textit{сечением СП в момент времени $t$}.
\end{defs}

\begin{defs}[Фазовое пространство]
  Совокупное множество возможных значений всех сечений СП ($\xi_t, t \in T \subseteq R$)
  называется \textit{множеством состояний (или фазовым пространством)} этого СП.
  $E = \bigcup\limits_{t \in T}\xi_t( \, \Omega) \, \subseteq R$\\
  Соответственно, если для некоторых $t \in T, x \in R, \omega \in \Omega$ выполняется
  $\xi_t( \,\omega) \,=x$, то говорят, что в момнент времени $t$ СП ($\xi_t, t \in T \subseteq R$)
  находится в состоянии $x$ (при реализации элементарного события $\omega$).
\end{defs}

\begin{defs}[Марковский СП]
  СП ($\xi_t, t \in T \subseteq R$) называется \textit{марковским}, если он обладает марковским свойством
\end{defs}

\begin{defs}[Настоящий момент времени]
  Произвольный фиксированный $t_0 \in T \subseteq R$) называется \textit{настоящим моментом времени}.
\end{defs}

\begin{defs}[$\sigma$-алгебра настоящего]
  Для всякого настоящего момента времени $t_0$ порожденный сечением $\xi_{t_0}$ класс событий $\mathfrak{A}_{t_0} = \mathfrak{A}_{\xi_{t_0}} =
  \xi_{t_0}^{-1}( \, B( \,R) \,) \, = \{A \in \mathfrak{A} | \exists B \in B( \, R) \,\}$ является, очевидно, $\sigma$-алгеброй и называется
  \textit{$\sigma$-алгеброй настоящего}, а входящие в него события - \textit{настоящими событиями}.
\end{defs}

\begin{defs}[Простое настоящее событие]
  Пусть $E = \bigcup\limits_{t \in T}\xi_t( \, \Omega) \,$ - множество состояний (фазовое пространство) СП.
  Тогда $\forall i \in E$ настоящее событие $\{\xi_{t_0} = i\}$ называется \textit{простым настоящий событием}.
\end{defs}

\begin{defs}[Прошлые события]
  Для любых прошлых моментов времени $\ldots < s_n < \ldots < s_1 < t_0$ порожденные сечениями $\xi_{s_1}, \xi_{s_2}, \ldots$
  $\sigma$-алгебры $\mathfrak{A}_{s_l} = \mathfrak{A}_{s_1} \cap \mathfrak{A}_{s_2}, \ \ldots = \xi_{t_0}^{-1}( \, B( \,R) \,) \, \cap \ldots$
  называются \textit{$\sigma$-алгебрами прошлого}, а входящиев них события вида $A = \{\xi_{s_1}, \ \xi_{s_2}, \ \ldots\}$ называются
  \textit{прошлыми событиями}.
\end{defs}

\begin{defs}[Будущие события]
  Аналогично определяется $\sigma$-алгебра будущего и будущие события, то есть $\forall t_0 < t_1 < \ldots \mathfrak{A}_{t_1,t_2} =
  \mathfrak{A}_{t_1} \cap \mathfrak{A}_{t_2} \cap \ldots$ - $\sigma$-алгеюра будущего, а
  $A = \{\xi_{t_1}^{-1}( \, B( \,R) \,) \,, \xi_{t_2}^{-1}( \, B( \,R) \,) \,\}$ - будущие события.
\end{defs}

\begin{defs}[Марковское свойство]
  Говорят, что СП  ($\xi_t, t \in T \subseteq R$) \textit{обладает маровским свойством} (и, соответственно, является марковским процессом), если
  для этого СП выполняется следующее свойство (которое и называется \textit{марковским}): $\forall t_0 \in T, \ \forall$ прошлых моментов
  времени  $\ldots < s_n < \ldots < s_1 < t_0, \ \forall$ будущих моментов времени $t_0 < t_1 < t_2 < \ldots \, \forall$ состояния $i \in E$, то есть
  для любого простого настоящего события $H^{(\ ,i)\ ,} = \{\xi_{t_0} = i\}, \ \forall$ прошлого события $\text{П} \in \mathfrak{A}_{s_1,s_2,\ldots}, \
  \forall$ будущего события $\text{Б} \in \mathfrak{A}_{t_1, t_2, \ldots}$ верно соотношение $P\{\text{Б} \ / \ H^{( \,i) \, \cap \text{П}}\}=
  P\{\text{Б} \ / \ H^{( \,i) \,}\}$ всегда, когда участвующие условные вероятности корректно определены. Читается: при известном простом настоящем
  событии будущее не зависит от прошлого.
\end{defs}

\begin{defs}[Марковская цепь]
  \textit{Марковская цепь} - всякий СП, обладающий свойствами:\\
  1) Это маровский СП (выполняется марковское свойство); \\
  2) Этот процесс определен на дискретном временном промежутке ($T = \{t_1, t_2, \ldots\}$).\\
  Таким образом, марковская цепь - это марковская случайная последовательность ($\xi_t, \ t \in T = \{t_1, t_2, \ldots\}$) =
  ($\xi_{t_1}, \xi_{t_2}, \ldots$).
\end{defs}

\begin{defs}[Конечная марковская цепь]
  Марковская цепь называется \textit{конечной}, если конечным является её фазовое пространство
  (множество состояний $E = \bigcup\limits_{t_n}\xi_{t_n}(\, \omega) \, = \{x_1, \ldots, x_m\}$).
\end{defs}

Как правило, в теории конечных однородных марковских цепей (далее КОМЦ) время считается натуральным
($T \subset N_0$). Кроме того, в качестве конечного фазового пространства $E$ обычно рассматривается множество первых $N$ натуральных чисел.
Поэтому далее будут рассматриваться марковские цепи ($\xi_1, \xi_2, \ldots$) с временным промежутком $T \subset N_0$ и фазовым
пространством $E = \{1, 2, \ldots\}$. \\

Поскольку фазовое пространство конечное, то все сечения КМЦ имеют дискретное распределение. \\

\begin{example}[]
  Будет после консультации.
\end{example}

\begin{defs}[Конечная однородная марковская цепь]
  Конечная цепь Маркова ($\xi_0, \xi_1$) называется \textit{однородной}(по времени), если выполняются следующие свойства:\\
  1)$\forall n \in N_0, \ \forall i,j \in E$ условные вероятности $P\{\xi_{n+1}=j \ / \ \xi_n\}$ корректно определены.
  Это, в частности, означает, что функция $P( \,. \ / \ \xi_n=i) \, = P_{\{\xi_n=i\}}( \,.) \,$ является полноценной
  вероятностной функцией от одного аргумента при любом фиксированном другом, какой бы ни была вероятность события $\{\xi_n=i\}$, а
  также эта функция должна обладать дополнительными свойствами: согласованность, расширенная нормированность, сужение на ацидент. \\
  2) $\forall i,j \in E$ условная вероятность $P\{\xi_{n+1}=j \ / \ \xi_n=i\}$ не зависит от $n \in N_0$, то есть
  $P\{\xi_{n+1}=j \ / \ \xi_n=i\} = P\{\xi_{n}=j \ / \ \xi_{n-1}=i\} = \ldots = P\{\xi_1=j \ / \ \xi_0=i\}$.
\end{defs}

\begin{defs}[Переходные вероятности и матрица переходных вероятностей]
  Пусть ($\xi_0, \xi_1, \ldots$)- КОМЦ. Тогда $\forall i,j \in E$ не зависящие от $n$ условные вероятности называются
  \textit{переходными вероятностями} из состояния $i$ в состояние $j$ за 1 шаг (или за единицу времени).
  $p_{ij} = P\{\xi_{n+1}=j \ / \ \xi_n=i\} = P\{\xi_{n}=j \ / \ \xi_{n-1}=i\} = \ldots = P\{\xi_1=j \ / \ \xi_0=i\}$.
  Составленная из этих вероятностей матрица $\text{П} = ( \,p_{ij}) \, =
  \begin{pmatrix}
    p_{11} & \ldots & p_{1N} \\
    \ldots & \ldots & \ldots \\
    p_{N1} & \ldots & p_{NN} \\
  \end{pmatrix}$ называется \textit{матрицей переходных вероятностей} цепи за 1 шаг (за единицу времени).
\end{defs}

\begin{defs}[Переходные вероятности и матрица переходных вероятностей]
  Если ($\xi_0, \xi_1, \ldots$) - КОМЦ с ФП $E = \{1,2,\ldots,N\}$, то $\forall n \in N_0, \forall i,j \in E$ независящие от $n$ условные вероятности
  $P\{\xi_{n+m}=j \ / \ \xi_n=i\}$ называются \textit{переходными вероятностями} из состояния $i$ в состояние $j$ за $m$ шагов (единиц времени)
  и обозначается $p_{ij}^{( \,n) \,} = P\{\xi_{n+m}=j \ / \ \xi_n=i\} = \ldots = P\{\xi_m=j \ / \ \xi_0=i\}$.
  Составленная из этих вероятностей матрица $\text{П}^{( \,n) \,} = ( \,p_{ij}) \,_{N \times N}$
  называется \textit{матрицей переходных вероятностей цепи за $m$ шагов (единиц времени)}.
\end{defs}

\begin{proofs}[Уравнения Колмогорова-Чепмэна]
	Пусть ($\xi_0, \xi_1, \ldots$)- произвольная КОМЦ с ФП $E = \{1,2,\ldots,N\}$. Тогда $\forall n,m \in N_0, \forall i,j \in E$ верны
  следующие соотношения (уравнения Колмогорова-Чепмэна): $p_{ij}^{ \,(n+m) \,} = \sum\limits_{k=1}^N p_{ik}^{( \,n) \,}p_{kj}^{( \,m) \,}$,
  или в матричной форме: $\text{П}^{( \,n + m) \,} = \text{П}^{( \,n) \,} \cdot \text{П}^{( \,m) \,}$
	\begin{dokvo}
    Случай 1. $n = 0$ или $m = 0$. Пусть $m = 0$. Тогда $\text{П}^{( \,n + 0) \,} = \text{П}^{( \,n) \,} \cdot E_{N \times N}$.\\
    Случай 2. $n \neq 0$ и $m \neq 0$. $\forall i,j \in E = \{1,2,\ldots,N\} \ p_{ij}^{(n+m)} = P\{'xi_{n+m} \ / \ \xi_0=i\} =
    P_{\{\xi_0 = i\}}\{\xi_{n+m} = j\} \oeq$\\
    По условию $P_{\{\xi_0=i\}}(.)$ корректно определены $\sledue$\\
    1)$P_{\{\xi_0=i\}}(.)$ - полноценная вероятностная функция (в частности верна формула полной вероятности); \\
    2)Выполняются свойства согласованности, расширенной нормированности, сужения на ХУЙ.\\
    Рассмотрим события $C_1=\{\xi_n=1\}, \ldots, C_N=\{\xi_n=N\}$. Очевидно, что {$C_1,\ldots,C_N$} - ПГПНС.
    Тогда по формулам полной вероятности для условной вероятности $P_{\{\xi_0 = i\}}(.)$ имеем
    $\oeq \sum\limits_{k=1}^N P_{\{\xi_0 = i\}}(\xi_{n+m} = j \ / \ C_k)\cdot P_{\{\xi_0 = i\}} = /\text{по свойству согласованности}/
    \sum\limits_{k=1}^N P(\xi_{n+m} = j \ / \ \xi_0 = i, C_k)\cdot P(C_k \ / \ \xi_0 = i) =
    \sum\limits_{k=1}^N P(\xi_{n+m} = j \ / \ \xi_0 = i, \xi_n=k)\cdot P(\xi_n = k \ / \ \xi_0 = i) \oeq$ \\
    Пусть $n$ - настоящий момент времени $\sledue$ \\
    \begin{equation*}
      \begin{cases}
        0 < n \sledue 0 - \text{прошлый момент времени;} \\
        m > 0 \sledue n+m - \text{будущий момент времени;} \\
        \{\xi_n=k\} - \text{простое настоящее событие.}
      \end{cases}
    \end{equation*}

    $\sledue$ /по свойству марковости/ $\oeq \sum\limits_{k=1}^N P(\xi_{n+m} = j \ / \ \xi_n = k)\cdot P(\xi_n=k \ / \ \xi_0 = i) = $
    /из однородности процесса/ $= \sum\limits_{k=1}^N P(\xi_m = j \ / \ \xi_n = k)\cdot P(\xi_n=k \ / \ \xi_0 = i) =
    \sum\limits_{k=1}^N p_ik^{(n)}p_{kj}^{(m)}$ - уравнение Колмогорова-Чепмэна.\\
    Далее $\forall i,j \in E \ (\text{П}^{(n+m)})_{ij} = p_{ij}^{(n+m)} =
    \sum\limits_{k=1}^N p_{ik}^{(n)}p_{kj}^{(m)} = \overrightarrow{(\text{П}^{(n)})_i}\cdot(\text{П}^{(m)}_j)^{\downarrow} \sledue
    \text{П}^{(n+m)} = \text{П}^{(n)}\cdot \text{П}^{(m)}$
	\end{dokvo}
\end{proofs}

\begin{sledsv}
  Для всякой КОМЦ её матрица переходных вероятностей за $n$ шагов является степенью матрицы переходных
  вероятностей за 1 шаг: $\text{П}^{(n)} = \text{П}^n$
  \begin{dokvo}
    Если $n \geqslant 1 \text{П}^{(n)} = \text{П}^{(n-1+1)} = \text{П}^{(n-1)}\cdot\text{П} = \ldots = \text{П}^n$
  \end{dokvo}
\end{sledsv}
\newpage
