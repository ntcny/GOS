%!TEX root = ../report.tex"
\section{Вопрос 11: Матрицы над полем. Ранг матрицы над полем.
Эквивалентные матрицы и их ранги.
Приведение матрицы к ступенчатому и каноническому видам.
Теорема о ранге матрицы.
Нахождение ранга и олратной матрицы с помощью элементарных преобразований.}


\begin{defs}[Матрица над полем]
  Матрицей размеров $mxn$ над полем $P$ называют прямоугольную таблицу
  элементов поля $P$, состоящую из $m$ строк и $n$ столбцов.
\end{defs}

\begin{defs}[Ранг матрицы]
  Рангом ненулевой матрицы  $A$ называют
  наибольший из порядков отличных от нуля миноров матрицы A.
  Ранг нулевой матрицы равен 0.
\end{defs}

\begin{proofs}[О рангах эквивалентных матриц]
	Если матрицы $A$ и $B$ эквивалентны, то их ранги равны.
	\begin{dokvo}
    Пусть матрицы $A$ и $B$ эквивалентны и $\rang{A} = k$.
    По определению ранга в матрице $A$ $\forall l > k$ или совсем нет миноров порядка $l$, или все они равны нулю.
    Тогда по теореме о минорах эквивалентных матриц (Если $A,B \in R_{m,n}, A~B$ и все миноры $k$-го порядка матрицы $A$ кратны элементу $c$
    кольца $R$, то все миноры $k$-го порядка матрицы $B$ также кратны $c$) то же самое верно и для матрицы $B \sledue \rang{B} \leqslant k$, то есть $\rang{B} \leqslant \rang{A}$.
    Так как отношение эквивалентности матриц симметрично, то аналогичными рассуждениями имеем: $\rang{A} \leqslant \rang{B} \sledue \rang{A} = \rang{B}$.
	\end{dokvo}
\end{proofs}

\begin{defs}[Ступенчатая матрица]
  Ненулевая матрица $S=(s_{i,j})$ называется ступенчатой матрицей типа $S(i_1, \ldots, i_r)$,
  где $r \in \overline{1,m}, 1 \leqslant i_1 < \ldots < i_r \leqslant n$, если:
  \begin{enumerate}
  	\item $s_{1i_1}, s_{2i_2}, \ldots, s_{ri_r} \neq 0$
  	\item $s_{l_t} = 0$ при $l > r, t \in \overline{1,n}$ и при $l \in \overline{1,r}, t < i_l$
  \end{enumerate}
\end{defs}

Подробно:\\
$$S =
\begin{pmatrix}
  0\ldots0 & s_{1i_1}\ldots* & *\ldots* & * & *\ldots* \\
  0\ldots0 & 0\ldots0 & s_{2i_2}\ldots* & * & *\ldots* \\
  \ldots & \ldots & \ldots & \ldots & \ldots \\
  0\ldots0 & 0\ldots0 & 0\ldots0 & s_{ri_r} & *\ldots* \\
  0\ldots0 & 0\ldots0 & 0\ldots0 & 0 & 0\ldots0 \\
  \ldots & \ldots & \ldots & \ldots & \ldots \\
  0\ldots0 & 0\ldots0 & 0\ldots0 & 0 & 0\ldots0 \\
\end{pmatrix}$$

\begin{proofs}[Об ступенчатой матрице]
	Любую матрицу $A$ над полем $P$ можно элементарными преобразованиями строк привести к ступенчатой матрице.
	\begin{dokvo}
    Индукция по числу $m$ строк матрицы.
    \begin{enumerate}
    \item База индукции. $m = 1$. Матрица $A$ и есть ступенчатая, утверждение верно.
    \item Предположим, что утверждение верно для $m$.
    \item Докажем для $m + 1$.
      Если $A$ - нулевая матрица, то она ступенчатая и утверждение верно.
      Пусть $A \neq 0$ и $A^\downarrow_{i_1}$ - самый левый ненулевой столбец матрицы $A$.
      Переставляя, если нужно, строки матрицы $A$ мы получим строчно-эквивалентную матрицу $B$: \\
      $$B =
      \begin{pmatrix}
        0\ldots0 & b_{1i_1} & *\ldots* \\
        0\ldots0 & b_{2i_2} & *\ldots* \\
        \ldots & \ldots & \ldots \\
        0\ldots0 & b_{m+1i_{m+1}} & *\ldots* \\
      \end{pmatrix}$$\\
      в которой $b_{1i_1} \neq 0$. Прибавляя к $l$-ой строке матрицы $B$ для каждого
      $l \in \overline{2, m + 1}$ ее $l$-ую строку, умноженную на $-b_{li_1}b_{1i_1}$ получим матрицу $B'$: \\
      $$B' =
      \begin{pmatrix}
        \begin{matrix}
          0\ldots0 & b_{1i_1} & *\ldots*
        \end{matrix}\\

        \begin{matrix}
          0\ldots0 & 0 \\
          \ldots & \ldots \\
          0\ldots0 & 0 \\
        \end{matrix}
        \begin{bmatrix}
          *\ldots* \\
          A_1 \\
          *\ldots* \\
        \end{bmatrix}
      \end{pmatrix}$$\\

      Так как чсло строк матрицы $A_1$ равно $m$, то по предположению индукции она строчно
      эквивалентна ступенчатой матрице, то есть $B'$ можно преобразовать в искомую ступенчатую матрицу.
    \end{enumerate}
	\end{dokvo}
\end{proofs}

\begin{defs}[Каноническая матрица]
	Каноническими матрицами над полем $P$ называются нулевая матрица и все матрицы видам
  $diag(e,\ldots,e,0,\ldots,0)_{mxn}$
\end{defs}

\begin{proofs}[О каноничской форме матрицы]
	Для любой матрицы $A$ над полем $P$ существует единственная эквивалентная ей каноничская матрица.
  Она называется каноничской формой матрицы $A$ и обозначается $K(A)$.
	\begin{dokvo}
    Если $A$ - нулевая матрица, то она уже каноничская.\\
    Пусть теперь $A \neq 0$. Приведем сначала матрицу $A$ элементарными преобразованиями строк к ступенчатой матрице.
    Умножив её $l$-ую строку на $s^{-1}_{li_1} \forall l \in \overline{1, r}$, получим матрицу
    с единицами на местах $(1, i_1), (2, i_2), \ldots, (r, i_r)$. Вычитая последовательноеё строки с номерами $2, \ldots, r$, умноженными
    на подходящие элементы, получим матрицу $C$:


    $$C =
    \begin{pmatrix}
      & i_1 & & i_2 & & i_r & \\
      0\ldots0 & e & *\ldots* & 0 & *\ldots* & 0 & *\ldots* \\
      0\ldots0 & 0 & 0\ldots0 & e & *\ldots* & 0 & *\ldots* \\
      \ldots & \ldots & \ldots & \ldots & \ldots & \ldots & \ldots \\
      0\ldots0 & 0 & 0\ldots0 & 0 & 0\ldots0 & e & *\ldots* \\
      0\ldots0 & 0 & 0\ldots0 & 0 & 0\ldots0 & 0 & 0\ldots0 \\
    \end{pmatrix}$$\\

    Теперь, вычитая столбцы с номерами $i_1, \ldots, i_r$, умноженные на подходящие элементы,
    все $*$ можно обнулить. После перестановки столбцов получается матрица вида $diag(e, \ldots, e, 0, \ldots, e)$.
	\end{dokvo}
\end{proofs}

\begin{defs}[Специальная ступенчатая матрица]
	Матрица вида $C$из теоремы называется специальной ступенчатой матрицей.
\end{defs}

\begin{proofs}[О эквивалентных матрицах (без доказательства)]
	$\forall A,B \in P_{m,n}$ равносильны утверждения:
  \begin{enumerate}
  	\item $A~B$;
  	\item $\exists U \in P_{m,n}, V \in P_{n,n}$ такие, что $B = UAV$;
    \item $\rang{A} = \rang{B}$;
    \item $K(A) = K(B)$.
  \end{enumerate}
	\begin{dokvo}
    Достаточно легко доказывается цепочка $1 \sledue 2 \sledue 3 \sledue 4 \sledue 1$
	\end{dokvo}
\end{proofs}
\newpage
