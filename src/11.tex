%!TEX root = ../report.tex"
\section{Вопрос 11}

\subsection{Матрицы над полем. Ранг матрицы над полем.
Эквивалентные матрицы и их ранги.
Приведение матрицы к ступенчатому и каноническому видам.
Теорема о ранге матрицы.
Нахождение ранга и олратной матрицы с помощью элементарных преобразований.}

\begin{defs}[Матрица над полем]
  Матрицей размеров $mxn$ над полем $P$ называют прямоугольную таблицу
  элементов поля $P$, состоящую из $m$ строк и $n$ столбцов.
\end{defs}

\begin{defs}[Ранг матрицы]
  Рангом ненулевой матрицы  $A$ называют
  наибольший из порядков отличных от нуля миноров матрицы A.
  Ранг нулевой матрицы равен 0.
\end{defs}

\begin{proofs}[О рангах эквивалентных матриц]
	Если матрицы $A$ и $B$ эквивалентны, то их ранги равны.
	\begin{dokvo}
    Пусть матрицы $A$ и $B$ эквивалентны и $\rang{A} = k$.
    По определению ранга в матрице $A$ $\forall l > k$ или совсем нет миноров порядка $l$, или все они равны нулю.
    Тогда по теореме о минорах эквивалентных матриц (Если $A,B \in R_{m,n}, A~B$ и все миноры $k$-го порядка матрицы $A$ кратны элементу $c$
    кольца $R$, то все миноры $k$-го порядка матрицы $B$ также кратны $c$) то же самое верно и для матрицы $B \sledue \rang{B} \leqslant k$, то есть $\rang{B}  \leqslant \rang{A}$.
    Так как отношение эквивалентности матриц симметрично, то аналогичными рассуждениями имеем: $\rang{A} \leqslant \rang{B} \sledue \rang{A} = \rang{B}$.
	\end{dokvo}
\end{proofs}

\begin{defs}[Ступенчатая матрица]
  Ненулевая матрица $S=(s_{i,j})$ называется ступенчатой матрицей типа $S(i_1, \ldots, i_r)$,
  где $r \in \overline{1,m}, 1 \leqslant i_1 < \ldots < i_r \leqslant n$, если:
  \begin{enumerate}
  	\item $s_{1i_1}, s_{2i_2}, \ldots, s_{ri_r} \neq 0$
  	\item $s_{l_t} = 0$ при $l > r, t \in \overline{1,n}$ и при $l \in \overline{1,r}, t < i_l$
  \end{enumerate}
\end{defs}

Подробно:\\
$$\begin{pmatrix}
  0\ldots0 & s_{1i_1}\ldots* & *\ldots* & * & *\ldots* \\
  0\ldots0 & 0\ldots0 & s_{2i_2}\ldots* & * & *\ldots* \\
  \ldots & \ldots & \ldots & \ldots & \ldots \\
  0\ldots0 & 0\ldots0 & 0\ldots0 & s_{ri_r} & *\ldots* \\
  0\ldots0 & 0\ldots0 & 0\ldots0 & 0 & 0\ldots0 \\
  \ldots & \ldots & \ldots & \ldots & \ldots \\
  0\ldots0 & 0\ldots0 & 0\ldots0 & 0 & 0\ldots0 \\
\end{pmatrix}$$%$

\begin{proofs}[Об ступенчатой матрице]
	Любую матрицу $A$ над полем $P$ можео элементарными преобразованиями строк привести к ступенчатой матрице.
	\begin{dokvo}
    Индукция по числу $m$ строк матрицы.
    1. База индукции. $m = 1$. Матрица $A$ и есть ступенчатая, утверждение верно.
    2. Предположим, что утверждение верно для $m$.
    3. Докажем для $m + 1$.
    Если $A$ - нулевая матрица, то она ступенчатая и утверждение верно.
    Пусть $A \neq 0$ и $A_{i_1}\downarrow$
	\end{dokvo}
\end{proofs}
