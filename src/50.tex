
\section{Вопрос 50: Криптосистема RSA. Выбор параметров.}

\zagolovok RSA (аббревиатура от фамилий Rivest, Shamir и Adleman) — криптографический алгоритм с открытым ключом, основывающийся на вычислительной сложности задачи факторизации больших целых чисел.

В основе RSA лежит задача факторизации произведения двух простых больших чисел. Для шифрования используется простая операция возведения в степень по модулю N. Для расшифрования же необходимо вычислить функцию Эйлера от числа N, для этого необходимо знать разложение числа n на простые множители (В этом и состоит задача факторизации).

\zagolovok{Генерация ключей RSA}
Всё начинается с генерации ключевой пары (открытый, закрытый ключ). Генерация ключей в RSA осуществляется следующим образом:

1.Выбираются два простых числа p и q (такие что p неравно q).

2. Вычисляется модуль $N=p*q$.

3. Вычисляется значение функции Эйлера от модуля $N: \varphi(N)=(p-1)(q-1)$.

4. Выбирается число e, называемое открытой экспонентой, число e должно лежать в интервале $q<e<\varphi(N)$, а так же быть взаимно простым со значением функции $\varphi(N)$.

5.Вычисляется число d, называемое секретной экспонентой, такое, что $d*e=1(mod\varphi(N))$, то есть является мультипликативно обратное к числу e по модулю $\varphi(N)$.

Итак, мы получили пару ключей:

Пара (e,N) - открытый ключ.

Пара (d,N) - закрытый ключ.

\zagolovok{Шифрование и расшифровывание в RSA}

Есть следующий сценарий: Боб и Алиса переписываются в интернете, но хотят использовать шифрование, чтобы поддерживать переписку в секрете :). Алиса заранее сгенерировала закрытый и открытый ключ, а затем отправила открытый ключ Бобу. Боб хочет послать зашифрованное сообщение Алисе:

\zagolovok{Шифрование:}  Боб шифрует сообщение m, используя открытый ключ Алисы $(e,N) : C = E(M) = M^{e}mod(N)$, и отправляет с Алисе.

\zagolovok{Расшифровывание:} Алиса принимает зашифрованное сообщение c. Используя закрытый ключ $(d,N)$, расшифровывает сообщение $ M = D(C) = C^{d}mod(N)$.

\zagolovok{Выбор параметров} 
Чтобы алгоритм был стойким, необходимо:

1. Выбрать два больших простых случайных числа p и q (к примеру, >= 1024 бита каждое), должны быть не слишком различными и быть не слишком близкими

2. Наибольший общий делитель $(p-1) и (q-1)$ должен быть небольшим, в лучшем случае равен двум.

3. Выбрать большое значение открытой экспоненты e, как правило, выбирают простые числа Ферма: 17, 257, 65537...

4. Сохранение в секрете закрытого ключа.