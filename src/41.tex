%!TEX root = ../report.tex"
\section{Вопрос 41: Конечные поля, характеристика, число элементов. Существование и единственность поля с заданным примарным числом элементов. Описание подполей конечного поля. Теорема о примитивном элементе.
}

\begin{defs}[Конечное поле]
  Пусть $(P,+,\cdot)$ - поле. Оно называется \textit{конечным}, если $|P| < \infty$
\end{defs}

\begin{defs}[Характеристика поля]
  Пусть $(P,+,\cdot)$ - поле. Наименьшее из всех $n \in \mathbb{N}$, т.что $\forall\; a \in P$ выполняется:
  $n \cdot a = \underbrace{a + ... + a}_n$ называется \textit{характеристикой поля $P$}. Если такого $n$ не существует,
  то характеристика считается равной нулю. Обозначается $Char(P) = n$.
\end{defs}

\begin{proofs}[Теорема об описании простых полей]
	Пусть $P$ - простое поле, тогда ,если $Char(P) = 0$, то $P \cong \mathbb{Q}$, если $Char(P) = p$, где $p$ - простое, 
  то $P \cong \mathbb{Z}_{/p}$.
\end{proofs}

\begin{proofs}[Теорема о строении конечных полей]
	Пусть $P$ - конечное поле, тогда $|P| = p^t$, где $p$ - простое, $t \in \mathbb{N} $ и верно:
  \begin{enumerate}
    \item $p = Char(P), t = [P : P_o]$, где $P_o$ - простое подполе.
    \item $P$ - минимальное поле разложения $f(x) \in P_o[x]$, где $f(x) = x^{p^t} - x$.
  \end{enumerate}
  \begin{dokvo}
    \begin{enumerate}
    \item Пусть $P_o$ - простое подполе $P, p = Char(P) \neq 0\; \Rightarrow p$ - простое (теорема об описании простых полей). Положим $t = [P:P_o]$,
    тогда поле $P_o$ изоморфно $\mathbb{Z}_p$ (согласно какой-то теореме). $|P| = |P_{P_o}| = |P_o|^{dimP_{P_o}} = |P_o|^{P:P_o} = p^t$
    \item Рассмотрим группу $(P^*, \cdot) = G;\; |G| = p^t - 1 \Rightarrow \forall\; a \in G$ верно $a^{p^t - 1} = e \Rightarrow a^{p^t} = a
    \Rightarrow \forall\; a \in P$ - корень многочлена $f(x) = x^{p^t} - x$. Всего $p^t$ корней, значит $P$ - минимальное поле разложения. 
  \end{enumerate}
	\end{dokvo}
\end{proofs}

\begin{sledsv}
  Если $|P| = p^t$, то $\forall\; a \in P$ верно: $a^{p^t} = a$
\end{sledsv}

\begin{proofs}[Теорема о существовании и единственности поля с заданным примарным числом элементов]
	$\forall\; p$ - простого и $t \in \mathbb{N} \exists$ единственное с точностью до изоморфизма поле и $p^t$ элементов.

  \begin{dokvo}
    \zagolovok{Существование:} Пусть $f(x) = x^{p^t} - x \in \mathbb{Z}_{p}[x]\; \Rightarrow \exists$ МПР $P$. Покажем, что $|P| = p^t$.
    Пусть $M$ - множество корней $f(x)$ в $P$. Заметим, что $f'(x) = p^t x^{p^t - 1} - e \Rightarrow$ НОД$(f(x),f'(x)) = e \Rightarrow$
    у $f(x)$ нет кратных корней. Тогда $|M| = p^t$. Пусть $a,b \in M \Rightarrow a^{p^t} = a; b^{p^t} = b$. Так $(a+b)^{p^t} = a^{p^t} + b^{p^t} = a+b$
    (по формуле бинома для полей простой характеристики) $\Rightarrow a+b \in M.\; (a b)^{p^t} = a^{p^t}  b^{p^t} = ab \Rightarrow ab \in M \Rightarrow M < P$.
    Но $P$ - МПР $\Rightarrow M = P$.
    \zagolovok{Единственность:} Пусть $P, Q$ - конечные поля, $|P| = |Q| = p^t.\; P_o, Q_o$ - их простые подполя. Тогда, по теореме
    о строении конечных полей, $P, Q$ - МПР многчлена $f(x) = x^{p^t} - x$ над $P_o$ и $Q_o$ соответственно. Заметим, что $P_o \cong Q_o \cong \mathbb{Z}_p$,
    значит существует изоморфизм $\sigma : P_o \rightarrow Q_o \Rightarrow$ (по теореме об описании простых полей) существует изоморфизм $\tau : P \rightarrow Q$
    т.что $\tau | P_o = \sigma$ (ограниченное множеством $P_o$).
	\end{dokvo}
\end{proofs}

\begin{proofs}[Теорема об описании подполей конечного поля]
	Пусть $P, Q$ - конечные поля. Тогда поле $P$ содержит подполе изоморфное полю $Q$ ТИТТК $\exists\; m \in \mathbb{N} : |P| = |Q|^m$. В этом случае подполе единственное.

  \begin{dokvo}
    \zagolovok{Необходимость:} Пусть $K < P, m = [P:K], K \cong Q$, тогда $|P| = |K|^{[P:K]} = |K|^m = |Q|^m$.
    \zagolovok{Достаточность:} Пусть $\exists\; m \in \mathbb{N} : |P| = |Q|^m, p = Char(Q)$, тогда $|Q| = p^t, |P| = p^{mt} \Rightarrow$
     по теореме строении простых полей $P$ - МПР многочлена $f(x) = x^{p^t} - x$ над простым подполем $P_o$. Заметим, что $x^{p^t} - x | x^{p^mt} - x$.
     В поле $P$ многочлен $f(x)$ раскладывается на линейные множители. Его корни образуют поле $K < P, |K| = p^t$. По теореме о существовании поля с примарным числом элементов
     $K \cong Q$. Оно является также единственным, тк, если существет поле $K_1 < P, |K| = |K_1| \Rightarrow$ элементы $K_1$ это корни $f(x)$, но они лежат в $K \Rightarrow K = K_1$  
	\end{dokvo}
\end{proofs}

\begin{sledsv}
  В поле $GF(p^t)\; \forall\; d/t\; \exists\;!$ подполе из $p^d$ элементов.
\end{sledsv}

\begin{defs}[Примитивный элемент поля]
  Пусть $P = GF(q), (P*, \cdot)$ - группа и $|P*| = q-1$, тогда элемент $a \in P$ называется \textit{примитивным элементом поля $P$}, если $<a> = (P*, \cdot)$ (является порождающим элементом мультипликативной группы).
\end{defs}

\begin{proofs}[Теорема об примитивном элементе]
	В любом конечном поле существует примитивный элемент.

  \begin{dokvo}
    Пусть $(P*, \cdot)$ -  конечная абелева группа. Если $a \in P$, тогда $ord(a) | q-1$. Считаем, что $q = p^t \Rightarrow ord(a) | p^t - 1$. Положим $m = exp(P*, \cdot) =$ 
    НОК$\{ord(a) |a \in P*\}$. Тк $ord(a) | p^t - 1 \Rightarrow m | p^t - 1 \Rightarrow m \leq p^t - 1 \Rightarrow\; \forall\; a \in P*$ - является корнем
    многочлена $f(x) = x^m - e \Rightarrow у f(x) q-1$ корней в $P$, тогда $m = q - 1 \Rightarrow exp(P*) = ord(P*)$. Из курса алгебры известно, что
    $\exists a \in P*$ т.что $ord(a) = exp(P*) = q - 1 \Rightarrow a$ - примитивный элемент.

	\end{dokvo}
\end{proofs}

\begin{sledsv}
  Группа $(P*, \cdot)$ - циклическая и содержит $\phi(q-1)$ примитивных элементов (функция Эйлера).
\end{sledsv}

\newpage
