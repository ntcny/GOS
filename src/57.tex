%!TEX root = ../report.tex"
\section{Вопрос 57: Модель распространения проав доступа Take-Grant. Теорема о передаче прав в графе доступов, состоящем только из субъектов, и в произвольном графе доступов. Сведение систем Take-Grant к системам ХРУ и ТМД.}

\textbf{Модель $Take-Grant$} ориентирована на анализ путей распространения прав доступа в системах дискреционного управления доступом.

\begin{defs}[Модель $Take-Grant$]
	\begin{enumerate}
		\item $ O $ -- множество объектов.
		\item $S \subseteq O$ -- множество субъектов.
		\item $ R = \{ r_1, \ldots, r_m \} \cup \{ t, g \} $ -- множество видов прав доступа, где t -- право брать права доступа, g -- давать.
		\item $ G = (S, O, E) $ -- конечный, ориентированный без петель граф доступов, $E \subseteq O \times O \times R $ -- ребра графа.
	\end{enumerate}
\end{defs}

\textbf{Цель модели} -- определение и обоснование условий проверки возможности утечки права доступа по исходному графу.

\begin{defs}[Предикат $can-share$]
	$ \sqsupset x,y \in O_0, G_0 = (S_0, O_0, E_0), \alpha   \subseteq R$. Определим предикат
	$can-share(\alpha, x, y, G_0)$, который истинен ТИТТК $ \exists G_1 = (S_1, O_1, E_1), \ldots,
	G_N = (S_N, O_N, E_N)$ и правила $op_1, \ldots, op_N, N \geqslant 0$, т.что
	$G_0 \vdash_{op_1} \ldots \vdash_{op_N} G_N $ и $(x, y, \alpha) \subset E_N$.
\end{defs}

\begin{defs}[$tg$-связность]
	Пусть $G = (S, S, E)$ -- граф доступов, вершины которого являются субъектами. Вершины являются $tg$-связными, или
	соединены $tg$-путем, если между ними существует путь, каждое ребро которого помечено t или g.
\end{defs}

\begin{proofs}[О передаче прав в графе только из субъектов]
	$\sqsupset  G_0 = (S_0, S_0, E_0), x,y \in S_0, x \neq y$. Тогда предикат $can-share(\alpha, x, y, G_0)$
	истинен ТИТТК выполняются условия:
	\begin{enumerate}
		\item $ \sqsupset s_1, \ldots, s_m \in S_0 : (s_i, y, \gamma_i) \subset E_0 $, где
		$ i = 1, \ldots, m $ и $ \alpha = \gamma_1 \cup \ldots \cup \gamma_m $.
		\item Субъекты $ x $ и $ s_i $ являются $tg$-связными в графе $G_0, i = 1, \ldots, m $.
	\end{enumerate}
\end{proofs}

\begin{defs}[Остров]
	Островом в произвольном графе доступов $G_0$ называется его максимальный tg-связный подграф, состоящий только из вершин субъектов.
\end{defs}

\begin{defs}[Мост]
	Мостом в графе доступов $G_0$ называется tg-путь, концами которого являются вершины субъекты, проходящий через вершины объекты.
\end{defs}

\begin{defs}[Начальный пролет]
	Начальным пролетом моста в графе доступов называется tg-путь, началом которого является вершина субъект, концом -- объект, проходящий через ввершины объекты, словарная запись которого имеет вид
	 $ \overrightarrow{t}^* \overrightarrow{g} $.
\end{defs}

\begin{defs}[Конечный пролет]
	Конечным пролетом моста в графе доступов называется tg-путь, началом которого является вершина субъект, концом -- объект, проходящий через ввершины объекты, словарная запись которого имеет вид
	$\overrightarrow{t}^*$.
\end{defs}

\begin{proofs}[О передаче прав в произвольном графе]
	$\sqsupset  G_0 = (S_0, O_0, E_0), x,y \in O_0, x \neq y$. Тогда предикат $can-share(\alpha, x, y, G_0)$
	истинен ТИТТК или $(x, y, \alpha ) \subset E_0$, или выполняются условия:
	\begin{enumerate}
		\item $ \sqsupset s_1, \ldots, s_m \in O_0 : (s_i, y, \gamma_i) \subset E_0 $, где
		$ i = 1, \ldots, m $ и $ \alpha = \gamma_1 \cup \ldots \cup \gamma_m $.
		\item $ \exists x_1^{\shtrih}, \ldots, x_m^{\shtrih}, s_1^{\shtrih}, \ldots, s_m^{\shtrih} \in S_0$ :
			\begin{itemize}
				\item $x = x_i^{\shtrih}$ или $x_i^{\shtrih}$ соединен с $x$ начальным пролетом моста в графе $G_0$
				\item $s_i = s_i^{\shtrih}$ или $s_i^{\shtrih}$ соединен с $s_i$ конечным пролетом моста в графе $G_0$
			\end{itemize}
		\item В графе $G_0$ для каждой пары $(x_i^{\shtrih}, s_i^{\shtrih}) \exists $ острова $I_{i,1}, \ldots, I_{i, u_{i}}$, где $u_i \geqslant 1$, т.что
		$x_i^{\shtrih} \in I_{i,1}, s_i^{\shtrih} \in I_{i, u_{i}}$, и существуют мосты между островами $I_{i,j}$ и $I_{i, j+1}, j = 1, \ldots, u_i - 1$.
	\end{enumerate}
\end{proofs}

\textbf{Take-Grant в ХРУ}

Состояние системы $Take-Grant$ описыается:
\begin{enumerate}
	\item $G=(S_{tg}, O_{tg}, E)$
	\item $R_{tg}$
\end{enumerate}

\textbf{ХРУ:}
\begin{enumerate}
	\item $R = R_{tg} \cup \{own\}$
	\item $S = O = O_{tg}$
	\item $M_{|S| \times |S|}$ -- матрица доступов, если $(x,y,r) \in E$, то $r \in M[x,y]$, для $s \in S_{tg}$ выполняется условие $own \in M[s,s]$
\end{enumerate}

\begin{multicols}{2}
	\begin{itemize}
		\item $\mathbf{command} \  Take_{\alpha}(x,y,z)$

		$if (own \in M[x,x]) and (t \in M[x,y])and (r_1 \in M[y,z])and  \ldots and (r_k \in M[y,z]) then $

		\kav{внести} право $r_1$ в $M[x,z]$

		$\ldots$

		\kav{внести} право $r_k$ в $M[x,z]$

		$endif$

		$end$

		\item 		$\mathbf{command} \  Grant_{\alpha}(x,y,z)$

				$if (own \in M[x,x]) and (g \in M[x,y])and (r_1 \in M[x,z])and  \ldots and (r_k \in M[x,z]) then $

				\kav{внести} право $r_1$ в $M[y,z]$

				$\ldots$

				\kav{внести} право $r_k$ в $M[y,z]$

				$endif$

				$end$
		\end{itemize}
		\begin{itemize}
		\item 		$\mathbf{command} \ CreateObjectAlpha(x,y)$

				$if (own \in M[x,x]) then $

				\kav{создать} субъект $y$

				\kav{внести} право $r_1$ в $M[x,y]$

				$\ldots$

				\kav{внести} право $r_k$ в $M[x,y]$

				$endif$

				$end$

		\item 		$\mathbf{command} \ CreateSubjectAlpha(x,y)$

				$if (own \in M[x,x]) then $

				\kav{создать} субъект $y$

				\kav{внести} право $own$ в $M[y,y]$

				\kav{внести} право $r_1$ в $M[x,y]$

				$\ldots$

				\kav{внести} право $r_k$ в $M[x,y]$

				$endif$

				$end$
			\end{itemize}
			\begin{itemize}

			\item 		$\mathbf{command} \ RemoveAlpha(x,y)$
					$if (own \in M[x,x]) then $

					\kav{удалить} право $r_1$ в $M[x,y]$
					$\ldots$
					\kav{удалить} право $r_k$ в $M[x,y]$

					$endif$

					$end$

	\end{itemize}

\end{multicols}

\textbf{Take-Grant в ТМД}

	\textbf{Take-Grant:}
	\begin{multicols}{2}
		\begin{enumerate}
			\item $G=(S_{tg}, O_{tg}, E)$
			\item $R_{tg}$
		\end{enumerate}
	\end{multicols}

	\textbf{ТМД:}

	\begin{multicols}{2}
		\begin{enumerate}
			\item $R = R_{tg}$
			\item $S = O = O_{tg}$
			\item $M_{|S| \times |S|}$ -- матрица доступов, где для $x,y \in O_{tg}$,если $(x, y, r) \in E$ то $r \in M[x,y]$
			\item $T = \{subject, object\}$ -- множество типов
			\item $t: O \to T : t(s) = subject, t(o) = object, o \in O_{tg} \backslash S_{tg}$
			\item $q = (S, O, t, M)$ -- состояние ТМД
		\end{enumerate}
	\end{multicols}



	Построим отображение правил преобразования графов доступов в множество команд системы ТМД.
\begin{multicols}{2}
	\begin{itemize}
		\item 	$\mathbf{command} \ \mathit{Take}_{r_1, \ldots, r_k}os(x: subject,y:object,z:subject)$

			$if (t \in M[x,y]) \ and (r_1 \in M[y,z])and  \ldots \ and \ (r_k \in M[y,z]) then $

			\kav{внести} право $r_1$ в $M[x,z]$

			$\ldots$

			\kav{внести} право $r_k$ в $M[x,z]$

			$endif$

			$end$

			\item $\mathbf{command} \  \mathit{Grant}_{r_1, \ldots, r_k}so(x: subject,y:subject,z:object)$

			$if (g \in M[x,y]) \ and  \ (r_1 \in M[x,z]) and  \ldots and (r_k \in M[x,z]) \ then $

			\kav{внести} право $r_1$ в $M[y,z]$

			$\ldots$

			\kav{внести} право $r_k$ в $M[y,z]$

			$endif$

			$end$
	\end{itemize}

	\begin{itemize}
		\item 		$\mathbf{command} \  \mathit{CreateObjectAlpha}_{r_1, \ldots, r_k}(x: subject,y:object)$


				\kav{создать} субъект $y$ с типом $object$

				\kav{внести} право $r_1$ в $M[x,y]$

				$\ldots$

				\kav{внести} право $r_k$ в $M[x,y]$

				$endif$

				$end$
				\item		$\mathbf{command} \ \mathit{CreateSubjectAlpha}_{r_1, \ldots, r_k}(x: subject,y:subject)$

						$if (own \in M[x,x]) then $

						\kav{создать} субъект $y$ с типом $subject$

						\kav{внести} право $r_1$ в $M[x,y]$

						$\ldots$

						\kav{внести} право $r_k$ в $M[x,y]$

						$endif$

						$end$
	\end{itemize}
\end{multicols}


		$\mathbf{command} \ \mathit{RemoveAlpha}_{r_1, \ldots, r_k}(x: subject,y:object)$

		$if (own \in M[x,x]) \ then $

		\kav{удалить} право $r_1$ в $M[x,y]$

		\kav{удалить} право $r_k$ в $M[x,y]$

		$endif$

		$end$
\newpage
