
\section{Вопрос 54: Алгоритмы внутренней сортировки. Теорема о средней высоте двоичного дерева с p листьями. Теорема об оценке средней вычислительной сложности алгоритмов сортировки сравнениями. Примеры алгоритмов сортировки сравнениями: сортировка вставками в дерево, пирамидальная сортировка, быстрая сортировка. Лексикографическая сортировка как пример распределяющей сортировки. Оценки трудоемкости.}

\zagolovok{Внутренняя сортировка}

Внутренняя сортировка – это алгоритм сортировки, который в процессе упорядочивания данных использует только оперативную память (ОЗУ) компьютера. То есть оперативной памяти достаточно для помещения в нее сортируемого массива данных с произвольным доступом к любой ячейке и собственно для выполнения алгоритма. Внутренняя сортировка применяется во всех случаях, за исключением однопроходного считывания данных и однопроходной записи отсортированных данных. В зависимости от конкретного алгоритма и его реализации данные могут сортироваться в той же области памяти, либо использовать дополнительную оперативную память.




\newpage
