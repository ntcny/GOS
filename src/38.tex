%!TEX root = ../report.tex"
\section{Вопрос 32:
Теорема о стационарном распределении конечной однородной марковской цепи.
Пример: найти все стационарные распределения КОМЦ с переходными вероятностями
$p_{11} = 1 - a, \ p_{12} = a, \ p_{21} = b, \ p_{22} = 1 - b$, где $a \in [0,1], b \in [0,1]$.
Эргодическая конечная однородная цепь Маркова и эргодическое распределение.
Теорема о связи эргодического и стационарного распределения КОМЦ.
}

\begin{defs}[Стационарное распределение]
  Числовой вектор $\overrightarrow{q} = (q_1, \ldots, q_N)$ называется \textit{стационарным распределением КОМЦ}
  $(\xi_t, t \in N_0)$ с ФП $E = \{1, \ldots, N\}$, если оно обладает следующими свойствами:\\
  1) Является стохастическим: $\overrightarrow{q} \geqslant \overrightarrow{0}, \overrightarrow{q}1^\downarrow = 1$;\\
  2) Выполняется свойство стационарности: если в качестве $\overrightarrow{p_0}$ взять распределение $\overrightarrow{q}$, то распределение
  в оставшиеся моменты времени не изменится ($\overrightarrow{p_0} = \overrightarrow{q} \ \sledue \
  \overrightarrow{p_n} = \overrightarrow{q} \ \forall n = 1,2,\ldots$)
\end{defs}

\begin{proofs}[Теорема о стационарном распределении]
	А) Для любой КОМЦ стационарное распределение всегда существует и может быть не единственным;\\
  Б) Критерий стационарного распределения: распределение вероятностей $\overrightarrow{q} = (q_1, \ldots, q_N)$ является
  стационарным распределением ТИТТК этот вектор $\overrightarrow{q}$ удовлетворяет следующей системе: $\levfigurn{\overrightarrow{q} = \overrightarrow\cdot\text{П};\\
	\overrightarrow{q} \geqslant 0; \ \ \ \ \ \ (*) \\
	\overrightarrow{q}\cdot1^\downarrow = 1.\\}$

  В) Всякому несущественному состоянию КОМЦ в любом стационарном распределении соответствует нулевая координата:
  $\forall i \in E \ - \ \text{несущественного} \ q_i=0$.
	\begin{dokvo}
    А) Существование без доказательства.\\
    Б) $\sledue$  Пусть $\overrightarrow{q}$ - произвольное стационарное распределение, то есть \\
       1) $\overrightarrow{q}$ - распределение вероятностей ($\overrightarrow{q} \geqslant 0, \overrightarrow{q}1^\downarrow = 1$);\\
       2) $\overrightarrow{p_0} = \overrightarrow{q} \ \sledue \ \overrightarrow{p_n} = \overrightarrow{q} \forall n = 1,2,\ldots$;\\
    По доказанному ранее соотношению $\overrightarrow{p_n} = \overrightarrow{p_0}\cdot\text{П}^n \ \forall n = 1,2,\ldots$. Тогда из этих
    соотношений получаем, что $\overrightarrow{q} = \overrightarrow{q}\cdot\text{П}^n \ \forall n = 1,2,\ldots \ \sledue $ при $n=1$
    получаем $\overrightarrow{q} = \overrightarrow{q}\cdot\text{П}$. \\
      $\Leftarrow$ Пусть теперь $\overrightarrow{q}$ - произвольное решение уравнения (*), тогда:\\
      1) $\overrightarrow{q}$ - распределение вероятностей ($\overrightarrow{q} \geqslant 0, \overrightarrow{q}1^\downarrow = 1$); \\
      2) Докажем, что для $\overrightarrow{q}$ выполняется основное свойство стационарного распределения.
      Положим $\overrightarrow{p_0}=\overrightarrow{q}$. Тогда на основании известного соотношения берем $\overrightarrow{p_n} = p_0 \cdot \text{П}^n=
      \overrightarrow{q}\cdot\text{П}^n=(\overrightarrow{q}\cdot\text{П})\cdot\text{П}^{n-1}=\ldots=\overrightarrow{q}\cdot\text{П}=\overrightarrow{q}
      \ \forall n = 0,1,2,\ldots \ \sledue$ по определению $q$ - стационарное распределение. \\
    В) Без доказательства.
	\end{dokvo}
\end{proofs}

\begin{defs}[Эргодическая цепь Маркова]
  КОМЦ $(\xi_t, n \in N_0)$ с ФП $E = \{1, \ldots, N\}$, матрицей $\text{П}=(p_{ij})$ называется \textit{эргодической},
  если $\forall i,j \in E \ \exists \predel{n \to \infty}p_{ij}^{(n)}$ и этот предел не зависит от исходного состояния $i$
   ($\forall i,j \in E \ \exists \ \predel{n \to \infty}p_{ij}^{(n)}=\pi_j$), или в матричном виде
   $\exists \ \predel{n \to \infty}\text{П}^{(n)} = \predel{n \to \infty}\text{П}^{n} =
    \begin{pmatrix}
      \pi_1 & \ldots & \pi_n \\
      \ldots & \ldots & \ldots \\
      \pi_1 & \ldots & \pi_n \\
    \end{pmatrix} = 1^\downarrow\cdot\overrightarrow{\pi}$
\end{defs}

\begin{proofs}[Теорема о связи стационарных и эргодических распределений]
	Пусть $(\xi_n, \ n \in N_0)$ - КОМЦ с ФП $E = \{1,\ldots,N\}$, матрицей $\text{П}$ и пусть эта цепь является эргодической
  с единственным эргодическим распределением $\pi$. Тогда:\\
  1) Стационарное распределение этой цепи единственно; \\
  2) Это единственное стационарное распределение совпадает с эргодическим; \\
  3) $\overrightarrow{\pi}$ является решением системы $(*)$, то есть $\levfigurn{  \overrightarrow{\text{П}} = \overrightarrow{\text{П}}\cdot\text{П};\\
		\overrightarrow{\text{П}} \geqslant 0;\\
		\overrightarrow{\text{П}}\cdot 1^\downarrow = 1.}$

	\begin{dokvo}
    Пусть $\overrightarrow{q}$ - произвольное стационарное распределение цепи.
    Тогда по определению если $\overrightarrow{p_0} = \overrightarrow{q}$, то $\overrightarrow{p_n}= \overrightarrow{p_0} = \overrightarrow{q}
    \forall i = 1,2,\ldots \ \sledue \ \exists \predel{n \to \inf}\overrightarrow{p_n} = \overrightarrow{q}$. \\
    С другой стороны, если существует эргодическое распределение $\pi$, то $\exists \ \predel{n \to \inf}\overrightarrow{p_n}
    = \overrightarrow{\pi} \ \forall \overrightarrow{p_0}$.
    В силу единственности предела $\overrightarrow{q} = \overrightarrow{\pi} \ \sledue \ \overrightarrow{q}$ - единственно.
	\end{dokvo}
\end{proofs}
\newpage
