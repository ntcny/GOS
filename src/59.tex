%!TEX root = ../report.tex"
\section{Вопрос 59: Модель Белла-ЛаПадулы. Базовая теорема безопасности. Безопасность переходов. Теорема о безопасности системы Белла-ЛаПадулы с безопасной функцией переходов. 
Проблемы практического приминения модели Белла-ЛаПадулы. Модель мандатной политики целостности Биба.}

\textbf{Элементы классической модели Белла-ЛаПадулы:}
\begin{enumerate*}
		\item $ O $ -- множество объектов.
		\item $S$ -- множество субъектов.
		\item $ R = \{ read, write, append$ (доступ на запись в конец объекта) $,execute\}$ -- множество видов доступа и видов прав доступа.
		\item $B = \{b \subseteq S \times O \times R\}$ -- множество возможных множеств текущих доступов в системе.
		\item $(L, \leqslant)$ -- решетка уровней конфиденциальности.
		\item $M=\{m_{|S| \times |O|}\}$ -- множество возможных матриц доступов, где $m_{|S| \times |O|}$ -- матрица доступов, $m[s,o] \subseteq R$ -- права доступа субъекта к объекту.
		\item $(f_s, f_o, f_c) \in F = L^S \times L^O \times L^S$ -- тройка функций, задающих: $f_s : S \to L$ -- уровень доступа субъектов, $f_o : O \to L$ -- уровень конфиденциальности объектов, 
			$f_c : S \to L$ -- текущий уровень доступа субъектов. При этом для любого $s$ верно $f_c(s) \leqslant f_s(s)$.
		\item $V = B \times M \times F$ -- множество состояний системы.
		\item $Q$ -- множество запросов системе.
		\item $D$ -- множество ответов по запросам, например $({yes, no, error})$.
		\item $W \subseteq Q \times D \times V \times V$ -- множество действий системы, где $(q, d, v^*, v) \in W$ означает, что система по запросу $q$ с ответом $d$ перешла из состояния $v$ в состояние $v^*$.
		\item $\mathbb{N}_0 = \{0, 1, 2, \ldots\}$ -- множество значений времени.
		\item $X$ -- множество функций $x:\mathbb{N}_0 \to Q$, задающих все возможные последовательности запросов системе.
		\item $Y$ -- множество функций $y:\mathbb{N}_0 \to D$, задающих все возможные последовательности ответов системы по запросам.
		\item $Z$ -- множество функций $z:\mathbb{N}_0 \to V$, задающих все возможные последовательности состояний системы.
\end{enumerate*}

\begin{defs}[Система]
		$\sum (Q,D,W,z_0) \subseteq X \times Y \times Z$ называется системой, когда для каждого $(x,y,z) \in \sum (Q,D,W,z_0)$ выполняется условие: для $t \in \mathbb{N}_0, (x_t, y_t, z_{t+1}, z_t) \in W$, где
		$z_0$ -- начальное состояние системы. При этом каждый набор $(x,y,z) \in \sum (Q,D,W,z_0)$ называется реализацией системы, а $(x_t, y_t, z_{t+1}, z_t) \in W$ -- действием системы в момент времени $t$.
\end{defs}

\textbf{Безопасность системы определяется с помощью трех свойств: $ss, *, ds$}.

\begin{defs}[доступ и ss-свойство]
	Доступ $(s,o,r) \in S \times O \times R$ обладает ss-свойством относительно $f = (f_s, f_o, f_c) \in F$, когда выполняется одно из условий:
	\begin{itemize*}
		\item $r \in \{execute, append\}$;
		\item $r \in \{read, write\}$ и $f_s(s) \geqslant f_o(o)$.
	\end{itemize*}
\end{defs}

\begin{defs}[состояние и ss-свойство]
	Состояние системы $(b,m,f) \in V$ обладает ss-свойством, когда каждый элемент $(s,o,r) \in b$ обладает ss-свойством относительно $f$.
\end{defs}

\begin{defs}[доступ и *-свойство]
	Доступ $(s,o,r) \in S \times O \times R$ обладает *-свойством относительно $f = (f_s, f_o, f_c) \in F$, когда выполняется одно из условий:
	\begin{itemize*}
		\item $r = execute$;
		\item $r = append$ и $f_o(o) \geqslant f_c(s)$.
		\item $r = read$ и $f_c(s) \geqslant f_o(o)$.
		\item $r = write$ и $f_c(s) = f_o(o)$.
	\end{itemize*}
\end{defs}

\begin{defs}[состояние и *-свойство]
	Состояние системы $(b,m,f) \in V$ обладает *-свойством, когда каждый элемент $(s,o,r) \in b$ обладает *-свойством относительно $f$.
\end{defs}

\begin{defs}[состояние, подмножество и *-свойство]
	Состояние системы $(b,m,f) \in V$ обладает *-свойством относительно подмножества $S^{\shtrih} \subseteq S$, когда каждый элемент $(s,o,r) \in b$, где $s \in S^{\shtrih}$, обладает *-свойством относительно $f$.
	При этом $S \setminus S^{\shtrih}$ называется множеством доверенных субъектов (имеющих право нарушать *-свойство).
\end{defs}

\begin{defs}[состояние и ds-свойство]
	Состояние системы $(b,m,f) \in V$ обладает ds-свойством, когда для каждого доступа $(s,o,r) \in b$ выполняется условие $r \in m[s,o]$.
\end{defs}

\begin{defs}[безопасное состояние]
	Состояние системы $(b,m,f)$ называется безопасным, когда оно обладает *-свойством относительно $S^{\shtrih}$, ss-свойством и ds-свойством.
\end{defs}

\begin{defs}[реализация системы и все свойства]
	Реализация системы $(x,y,z) \in \sum (Q,D,W,z_0)$ обладает ss-свойством(* и ds), когда в последовательности $(z_0, z_1, \ldots)$ каждое состояние обладает ss-свойством (*, ds).
\end{defs}

\begin{defs}[система и все свойства]
	Система $\sum (Q,D,W,z_0)$ обладает ss-свойством(* и ds), когда каждая ее реализация обладает ss-свойством (*, ds).
\end{defs}

\begin{defs}[безопасная система]
	Система $\sum (Q,D,W,z_0)$ называется безопасной, когда она обладает ss,*,ds-свойствами одновременно.
\end{defs}

\textbf{Справочка:}
	\begin{itemize*}
		\item из обладания ss-свойства выполняется запрет на чтение вверх.
		\item *-свойство исключает появление в системе запрещенного информационного потока сверху-вниз.
	\end{itemize*}

\begin{proofs}[А1]
	Система обладает ss-свойством для любого начального состояния $z_0$, обладающего ss-свойством, ТИТТК для каждого действия $(q,d,(b^*,m^*,f^*), (b,m,f)) \in W$ верно 1-2:
	\begin{enumerate*}
		\item Каждый $(s,o,r) \in b^* \setminus b$ обладает ss-свойством относительно $f^*$.
		\item Если $(s,o,r) \in b$ не обладает ss-свойством относительно $f^*$, то $(s,o,r) \notin b^*$.
	\end{enumerate*}
\end{proofs}

\begin{proofs}[А2]
	Система обладает *-свойством относительно $S^{\shtrih} \subseteq S$ для любого начального состояния $z_0$, обладающего *-свойством относительно $S^{\shtrih}$, ТИТТК для каждого действия 
	$(q,d,(b^*,m^*,f^*), (b,m,f)) \in W$ верно 1-2:
	\begin{enumerate*}
		\item Для $s \in S^{\shtrih}$ доступ $(s,o,r) \in b^* \setminus b$ обладает *-свойством относительно $f^*$.
		\item Для $s \in S^{\shtrih}$, если доступ $(s,o,r) \in b$ не обладает *-свойством относительно $f^*$, то $(s,o,r) \notin b^*$.
	\end{enumerate*}
\end{proofs}

\begin{proofs}[А3]
	Система обладает ds-свойством для любого начального состояния $z_0$, обладающего ds-свойством, ТИТТК для каждого действия $(q,d,(b^*,m^*,f^*), (b,m,f)) \in W$ верно 1-2:
	\begin{enumerate*}
		\item Для каждого $(s,o,r) \in b^* \setminus b$ выполняется условие $r \in m^*[s,o]$.
		\item Если $(s,o,r) \in b$ и $r \notin m^*[s,o]$, то $(s,o,r) \notin b^*$.
	\end{enumerate*}
\end{proofs}

\begin{proofs}[БТБ]
	Система $\sum (Q,D,W,z_0)$ безопасна для безопасного $z_0$ ТИТТК множество действий системы W удовлетворяет условиям теорем А1--А3.
\end{proofs}

\textbf{Безопасность переходов.}

В классической модели Белла-ЛаПадулы не описывается точный порядок действий системы при переходе из состояния в состояние. Частично этот недостаток был устранен с использованием функции переходов:
$T: Q \times V \to V$, где $T(q,v) = v^*$ -- переход из состояния $v$ по запросу $q$ в состояние $v^*$. В этом случае будем обозначать системы через $\sum (V,T,z_0)$. Далее переопределяются ss и *-свойства.

\begin{defs}[доступ и ss-свойство]
	Доступ $(s,o,r) \in b$ обладает ss-свойством относительно $f$, когда выполняется одно из условий:
	\begin{itemize*}
		\item $r = write$;
		\item $r = read$ и $f_s(s) \geqslant f_o(o)$.
	\end{itemize*}
\end{defs}

\begin{defs}[доступ и *-свойство]
	Доступ $(s,x,r) \in b$ обладает *-свойством относительно $f$, когда выполняется одно из условий:
	\begin{itemize*}
		\item $r = read$ и $\nexists \ y \in O: (s,y,write) \in b$ и $f_o(x) > f_o(y)$;
		\item $r = write$ и $\nexists \ y \in O: (s,y,read) \in b$ и $f_o(y) > f_o(x)$.
	\end{itemize*}
\end{defs}

\begin{defs}[функция переходов обладает ss-свойством]
	Функция переходов $T(q, (b,f)) = (b^*,f^*)$ обладает ss-свойством, когда выполнены следующие условия:
	\begin{itemize*}
		\item если $(s,o,read) \in b^* \setminus b$, то $f_s(s) \geqslant f_o(o)$ и $f^* = f$
		\item если $f_s(s) \neq f_s^*(s)$, то $f_o^* = f_o, b^* = b$, для $s^{\shtrih} \neq s$ справедливо равенство $f_s^*(s^{\shtrih}) = f_s(s^{\shtrih})$, и если $(s,o,read) \in b$, то $f_s^*(s) \geqslant f_o(o)$;
		\item если $f_o(o) \neq f_o^*(o)$, то $f_s^* = f_s, b^* = b$, для $o^{\shtrih} \neq o$ справедливо равенство $f_o^*(o^{\shtrih}) = f_o(o^{\shtrih})$, и если $(s,o,read) \in b$, то $f_s(s) \geqslant f_o^*(o)$;
	\end{itemize*}
\end{defs}

\begin{defs}[функция переходов обладает *-свойством]
	Функция переходов $T(q, (b,f)) = (b^*,f^*)$ обладает *-свойством, когда выполнены следующие условия:
	\begin{itemize*}
		\item если $\{(s,x,read), (s,y,write)\} \subseteq b^*$, и $\{(s,x,read), (s,y,write)\} \not\subset b$, то $f_o(y) \geqslant f_o(x)$ и $f^* = f$;
		\item если $f_o(y) \neq f_o^*(y)$, то $f_s^* = f_s, b^* = b$, для $z \neq y$ справедливо равенство $f_o^*(z) = f_o(z)$, и если $\{(s,x,read),(s,y,write)\} \subseteq b$, то $f_o^*(y) \geqslant f_o(x)$, или
		если $\{(s,y,read), (s,x,write)\} \subseteq b$, то $f_o(x) \geqslant f_o^*(y)$.
	\end{itemize*}
\end{defs}

\begin{defs}[Безопасность Т]
	Функция переходов $T$ безопасна, когда она обладает ss и *-свойством.
\end{defs}

\begin{proofs}[БТБ]
	Система $\sum (V,T,z_0)$ безопасна для безопасного начального состояния $z_0$, когда ее функция переходов безопасна.
\end{proofs}

\textbf{Модель мандатной политики целостности информации Биба.}

Классическая модель Белла-ЛаПадулы в первую очередь ориентирована на обеспечение защиты от угрозы конфиденциальности информации. В то же время ее математическая основа используется в модели Биба, 
реализующей мандатную политику целостности. Элементами модели Биба являются:
\begin{enumerate*}
	\item $modify$ -- доступ субъекта на модификацию объекта.
	\item $invoke$ -- доступ на обращение(запись) субъекта к субъекту.
	\item $observe$ -- доступ на чтение субъекта к объекту.
	\item $execute$ -- доступ на чтение субъекта к объекту.
	\item $S$ -- множество субъектов.
	\item $O$ -- множество объектов.
	\item $(LI, \leqslant)$ -- решетка уровней целостности.
	\item $RI = \{modify, invoke, observe, execute\}$ -- множество видов доступа и видов прав доступа.
	\item $B = \{b \subseteq S \times O \times RI\}$ -- множество возможных множеств текущих доступов в системе.
	\item $(i_s, i_o, i_c) \in I = LI^S \times LI^O \times LI^S$ -- тройка функций, задающих: $i_s : S \to LI$ -- уровень целостности субъектов, $i_o : O \to LI$ -- уровень целостности объектов, 
			$i_c : S \to LI$ -- текущий уровень целостности субъектов. При этом для любого $s$ верно $i_c(s) \leqslant i_s(s)$.
	\item $V=B \times I$ -- множество состояний системы.
\end{enumerate*}
\newpage









